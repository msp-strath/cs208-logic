% -*- TeX-engine: xetex -*-

\documentclass[xetex,aspectratio=169,14pt,hyperref={pdfpagelabels=true,pdflang={en-GB}}]{beamer}

\usepackage[sci,noslidestrathidentity]{strathclyde}
\strathsetidentity{Department of}{Computer \& Information Sciences}

\usepackage{lmodern}
\usepackage{subscript}
\usepackage{url}
\usepackage{pifont}
\usepackage{csquotes}
\usepackage{mathpartir}
\usepackage{stmaryrd}
\usepackage{multirow}
\usepackage{euler}
\usepackage[normalem]{ulem}

% 2013-09-08 SU adding xetex specifics
% http://www.woggie.net/2008/07/16/beamer-pdftex-and-xetex/
\usepackage{xltxtra}

% 2013-09-08 SU http://robjhyndman.com/hyndsight/xelatex/
\defaultfontfeatures{Ligatures=TeX}

% 2015-12-09 SU table beautified
\renewcommand{\arraystretch}{1.2}
\usepackage{booktabs}

%MGK compatibility
\newcommand{\hh}[1]
  {\medskip\textbf{\large #1}}
\newcommand{\pdu}[3]
  {#1\rightarrow #2 : &\ & \makebox[70mm][l]{$#3$}}



\setlength{\marginparwidth}{2cm}
\usepackage{todonotes}%[disable]
\let\OldTodo\todo
\renewcommand{\todo}{\OldTodo[inline]}%
\newcommand{\todolater}[1]{}% Things to do for next year


\DeclareTextCommandDefault{\nobreakspace}{\leavevmode\nobreak\ }
\usepackage{rotating}

\usepackage{pdfcomment}% To add alt text for images using \pdftooltip{}

\usepackage{appendixnumberbeamer}

\newcommand{\messageframe}[1]{\begin{frame}\begin{center}\Huge #1\end{center}\end{frame}}
\newcommand{\sechead}[1]{{\bf #1} \\}
\newcommand{\examplehead}[1]{{\bf Example:} {\it\textcolor{red!90}{#1}} \\}
\newcommand{\eqnote}[1]{\hspace{3cm}\textit{#1}}
\newcommand{\sidenote}[1]{\qquad {\footnotesize \textcolor{black!60}{(#1)}}}
\newcommand{\sem}[1]{\llbracket #1 \rrbracket}
\newcommand{\true}{\mathsf{T}}
\newcommand{\false}{\mathsf{F}}

\def\strikeafter<#1>#2{\temporal<#1>{#2}{\sout{#2}}{\sout{#2}}}


%\newcommand{\rhighlight}{\textcolor{titlered}}
\newcommand{\rhighlight}{\textbf}
\newcommand{\highlight}{\textbf}

% \setmainfont{Linux Biolinum O}
\setmainfont{LinBiolinum}[
Path=,
UprightFont = *_R.otf ,
BoldFont = *_RB.otf ,
ItalicFont = *_RI.otf
]

\setbeamertemplate{navigation symbols}{}
%\usecolortheme[rgb={0.8,0,0}]{structure}
\usefonttheme{serif}
\usefonttheme{structurebold}
\setbeamercolor{description item}{fg=black}


\author[Atkey]{Dr.~Robert Atkey}
\institute[Strathclyde]{Computer \& Information Sciences}
\date[]{}

\newcommand{\weeksection}[1]{%
  \section{\thetitle{}, Part~\thesection : #1}
  \begin{frame}
    \begin{center}
      \textcolor{black!60}{\thetitle{}, Part \thesection}\\
      {\Huge #1}
    \end{center}
  \end{frame}}

\newcommand{\weektitle}[2]{\def\thetitle{#2}
\title[CS208 - Topic #1]{CS208 (Semester 1) Topic #1 : #2}}

\newcommand{\assigned}{:}

\newcommand{\forcedto}{\assigned_f}
\newcommand{\decideto}{\assigned_d}


\weektitle{10}{Undecidability}

\begin{document}

\frame{\titlepage}

\weeksection{Metatheory}

\begin{frame}
  \begin{enumerate}
  \item Is our proof system sound and complete?
  \item Are our axiomatisations complete enough?
  \item Can we automate mathematics?
  \end{enumerate}
\end{frame}

\begin{frame}
  {Soundness}

  The proof system we have seen so far is \emph{sound}:
  \begin{displaymath}
    \Gamma \vdash Q \quad \Rightarrow \quad \Gamma \models Q
  \end{displaymath}
  \textcolor{black!60}{``Every provable judgement is valid.''}\\
  \textcolor{black!60}{Can be checked by checking that every rule preserves validity.}
\end{frame}

\begin{frame}
  {Completeness}

  If we add a rule for excluded middle ($P \lor \lnot P$ for any formula $P$), then it is \emph{complete}:
  \begin{displaymath}
    \Gamma \models Q \quad \Rightarrow \quad \Gamma \vdash Q
  \end{displaymath}
  \textcolor{black!60}{``Every valid judgement is provable.''}\\
  \textcolor{black!60}{This is not a simple fact. ``G{\"o}del's Completeness Theorem''}
\end{frame}

\begin{frame}
  {Automating Mathematics?}

  If our proof system is sound and complete, then we should be able to
  automatically prove things by searching for proofs?

  \bigskip

  This is one of the oldest branches of AI.
\end{frame}

\begin{frame}
  {Automating Mathematics?}

  We have seen so far that there are many axiomatisations for
  describing certain bits of mathematics:
  \begin{enumerate}
  \item Monoid axioms: addition with a zero.
  \item Peano's axioms: arithmetic
  \item Zermelo-Frankel axioms: set theory
  \end{enumerate}

  \bigskip

  Are these axiomatisations suitable for finding proofs?
\end{frame}

\begin{frame}
  {Syntactic Completeness}

  An axiomatisation $\mathrm{Ax}$ is \emph{syntactically complete} if
  for all formulas $P$, we can prove one of:
  \begin{displaymath}
    \mathrm{Ax} \vdash P
  \end{displaymath}
  or
  \begin{displaymath}
    \mathrm{Ax} \vdash \lnot P
  \end{displaymath}
  if we can prove both, then the theory is \emph{inconsistent}.
\end{frame}

\begin{frame}
  {Effectively Generatable}

  An axiomatisation $\mathrm{Ax}$ is \emph{effectively generatable} if
  we can write a computer program that generates all the valid axioms.

  \bigskip

  There may be infinitely many axioms, but each one will eventually be
  generated.
\end{frame}

\begin{frame}
  {Automation}

  If an axiomatisation $\mathrm{Ax}$ is syntactically complete and
  effectively generatable, then we can (in principle) write a program
  to search for a proof of some $P$:

  \bigskip

  \begin{enumerate}
  \item Search for a proof $\mathrm{Ax} \vdash P$ \\
    try proofs of size 1, then proofs of size 2, then proofs of size 3...
  \item Interleaved with this: search for a proof of $\mathrm{Ax} \vdash \lnot P$
  \end{enumerate}

  \bigskip

  Since one of them is provable, we will eventually terminate.
\end{frame}

\begin{frame}
  \begin{center}
    Is every interesting axiomatisation syntactically complete?
  \end{center}
\end{frame}

\begin{frame}
  {Peano's axioms ($\mathrm{PA}$)}

  \begin{enumerate}
  \item $\forall x.~\lnot(0 = S(x))$
  \item $\forall x. \forall y.~S(x) = S(y) \to x = y$
  \item $\forall x.~add(0,x) = x$
  \item $\forall x.\forall y.~add(S(x),y) = S(add(x,y))$
  \item $\forall x.~mul(0,x) = 0$
  \item $\forall x. \forall y.~mul(S(x),y) = add(y,mul(x,y))$
  \end{enumerate}
  $+$ induction

  \bigskip

  is effectively generatable.
\end{frame}

\begin{frame}
  {Robinson's axioms ($\mathrm{Q}$)}

  \begin{enumerate}
  \item $\forall x.~\lnot(0 = S(x))$
  \item $\forall x. \forall y.~S(x) = S(y) \to x = y$
  \item $\forall x.~add(0,x) = x$
  \item $\forall x.\forall y.~add(S(x),y) = S(add(x,y))$
  \item $\forall x.~mul(0,x) = 0$
  \item $\forall x. \forall y.~mul(S(x),y) = add(y,mul(x,y))$
  \item $\forall x. (x = 0) \lor (\exists y. x = S(y))$\qquad {\bf (instead of induction)}
  \end{enumerate}
\end{frame}

\begin{frame}
  {G{\"o}del's 1st Incompleteness Theorem}

  For any \emph{effectively generatable} consistent set of axioms
  $\mathrm{Ax}$ that imply those of Robinson arithmetic, there exists
  a formula $P$ such that it is \textbf{not} possible to prove either
  of
  \begin{displaymath}
    \mathrm{Ax} \vdash P
  \end{displaymath}
  or
  \begin{displaymath}
    \mathrm{Ax} \vdash \lnot P
  \end{displaymath}
\end{frame}

\begin{frame}
  {Consequences}

  $\mathrm{PA}$ is not syntactically complete, so our attempt to use
  it to automate mathematics fails.

  \bigskip

  In fact, provability in $\mathrm{PA}$ is undecidable, so all attempts
  are doomed.
\end{frame}

\begin{frame}
  {Consequences}

  \begin{enumerate}
  \item $\mathrm{PA}$ is incomplete, so there is a formula $P$ such that neither of:
    \begin{mathpar}
      \mathrm{PA} \vdash P

      \textrm{and}

      \mathrm{PA} \vdash \lnot P
    \end{mathpar}
    are provable.
  \item Inspection of G{\"o}del's proof shows that the formula $P$ it
    generates is actually true in ``the'' natural numbers.
  \item So we could use the axioms $\mathrm{PA} + P$, but then goto 1.
  \end{enumerate}
\end{frame}

% https://stopa.io/post/269

\begin{frame}
  {Consequences}

  So:
  \begin{enumerate}
  \item $\mathrm{PA}$ does not cover everything that is ``true'' about arithmetic
  \item Every attempt to fix it is doomed
  \end{enumerate}
\end{frame}

\begin{frame}
  {Consequences}

  \emph{Some people} have said that G{\"o}del's Incompleteness theorem
  shows that there are fundamental limitations to what computers can
  reason about.

  \pause
  \bigskip

  The reasoning (roughly) goes:
  \begin{enumerate}
  \item Computers can only use effectively generatable axioms
  \item This means that there are truths they cannot prove
  \item Humans can perceive ``real'' truth to see these truths
  \item Therefore, Humans are better than computers, and AI is impossible.
  \end{enumerate}
\end{frame}

\begin{frame}
  {Consequences}

  Two problems with this:

  \begin{enumerate}
  \item Humans only know that the formula generated by G{\"o}del's
    Incompleteness Theorem is true by some \emph{larger}
    axiomatisation we are (maybe implicitly) using. Computers can use
    this axiomatisation.
  \item The theorem depends on the theory being consistent. How do we
    \emph{know} this? Definitely not obvious for Zermelo-Frankel set
    theory.
  \end{enumerate}
\end{frame}

\begin{frame}
  {(In)Completeness?}

  G{\"o}del proved:
  \begin{enumerate}
  \item Completeness ``Everything that is true is provable''
  \item Incompleteness ``There exist true things that are not provable''
  \end{enumerate}

  \bigskip

  Surely a contradiction?
\end{frame}

\begin{frame}
  {(In)Completeness?}

  There is no contradiction.

  \pause
  \bigskip

  Completeness Theorem says that if something is true in \emph{every}
  model of the axioms, it is provable.

  \pause
  \bigskip

  Incompleteness Theorem only gives something that is true for
  \emph{``the''} natural numbers. It might be false in other models.
\end{frame}

\begin{frame}
  {Automating Mathematics?}

  If we can't completely automate arithmetic, then what can we do?

  \begin{enumerate}
  \item Do proof search with a timeout
  \item Restrict to weaker systems to gain decidability, e.g.:
    \begin{itemize}
    \item Pure equality
    \item Linear Arithmetic: only addition, no multiplication
    \end{itemize}
  \end{enumerate}

  \bigskip

  Automated proof for fragments of logic is a large and ongoing topic
  of research, with applications in software engineering, computer
  security, optimisation, ...
\end{frame}

\begin{frame}
  {Summary}

  \begin{enumerate}
  \item Our proof system is sound
  \item If we add excluded middle, it is complete
  \item G{\"o}del's Incompleteness theorem:
    \begin{itemize}
    \item If some axioms can prove basic facts about arithmetic, then
      there are statements that it can neither prove nor disprove.
    \end{itemize}
  \item Not every theory is decidable, but some useful ones are.
  \end{enumerate}
\end{frame}


\end{document}
