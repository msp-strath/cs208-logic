% -*- TeX-engine: xetex -*-

\documentclass[xetex,aspectratio=169,14pt,hyperref={pdfpagelabels=true,pdflang={en-GB}}]{beamer}

\usepackage[sci,noslidestrathidentity]{strathclyde}
\strathsetidentity{Department of}{Computer \& Information Sciences}

\usepackage{lmodern}
\usepackage{subscript}
\usepackage{url}
\usepackage{pifont}
\usepackage{csquotes}
\usepackage{mathpartir}
\usepackage{stmaryrd}
\usepackage{multirow}
\usepackage{euler}
\usepackage[normalem]{ulem}

% 2013-09-08 SU adding xetex specifics
% http://www.woggie.net/2008/07/16/beamer-pdftex-and-xetex/
\usepackage{xltxtra}

% 2013-09-08 SU http://robjhyndman.com/hyndsight/xelatex/
\defaultfontfeatures{Ligatures=TeX}

% 2015-12-09 SU table beautified
\renewcommand{\arraystretch}{1.2}
\usepackage{booktabs}

%MGK compatibility
\newcommand{\hh}[1]
  {\medskip\textbf{\large #1}}
\newcommand{\pdu}[3]
  {#1\rightarrow #2 : &\ & \makebox[70mm][l]{$#3$}}



\setlength{\marginparwidth}{2cm}
\usepackage{todonotes}%[disable]
\let\OldTodo\todo
\renewcommand{\todo}{\OldTodo[inline]}%
\newcommand{\todolater}[1]{}% Things to do for next year


\DeclareTextCommandDefault{\nobreakspace}{\leavevmode\nobreak\ }
\usepackage{rotating}

\usepackage{pdfcomment}% To add alt text for images using \pdftooltip{}

\usepackage{appendixnumberbeamer}

\newcommand{\messageframe}[1]{\begin{frame}\begin{center}\Huge #1\end{center}\end{frame}}
\newcommand{\sechead}[1]{{\bf #1} \\}
\newcommand{\examplehead}[1]{{\bf Example:} {\it\textcolor{red!90}{#1}} \\}
\newcommand{\eqnote}[1]{\hspace{3cm}\textit{#1}}
\newcommand{\sidenote}[1]{\qquad {\footnotesize \textcolor{black!60}{(#1)}}}
\newcommand{\sem}[1]{\llbracket #1 \rrbracket}
\newcommand{\true}{\mathsf{T}}
\newcommand{\false}{\mathsf{F}}

\def\strikeafter<#1>#2{\temporal<#1>{#2}{\sout{#2}}{\sout{#2}}}


%\newcommand{\rhighlight}{\textcolor{titlered}}
\newcommand{\rhighlight}{\textbf}
\newcommand{\highlight}{\textbf}

% \setmainfont{Linux Biolinum O}
\setmainfont{LinBiolinum}[
Path=,
UprightFont = *_R.otf ,
BoldFont = *_RB.otf ,
ItalicFont = *_RI.otf
]

\setbeamertemplate{navigation symbols}{}
%\usecolortheme[rgb={0.8,0,0}]{structure}
\usefonttheme{serif}
\usefonttheme{structurebold}
\setbeamercolor{description item}{fg=black}


\author[Atkey]{Dr.~Robert Atkey}
\institute[Strathclyde]{Computer \& Information Sciences}
\date[]{}

\newcommand{\weeksection}[1]{%
  \section{\thetitle{}, Part~\thesection : #1}
  \begin{frame}
    \begin{center}
      \textcolor{black!60}{\thetitle{}, Part \thesection}\\
      {\Huge #1}
    \end{center}
  \end{frame}}

\newcommand{\weektitle}[2]{\def\thetitle{#2}
\title[CS208 - Topic #1]{CS208 (Semester 1) Topic #1 : #2}}

\newcommand{\assigned}{:}

\newcommand{\forcedto}{\assigned_f}
\newcommand{\decideto}{\assigned_d}


\weektitle{9}{Automating Logic}

\begin{document}

\frame{\titlepage}

\weeksection{Automating Logic}

\begin{frame}
  {Introduction}

  TODO: Some introductory text
\end{frame}

\weeksection{SAT Solvers}

\begin{frame}
  {SAT solvers}

  SATisfiability solvers.

  \bigskip

  The problem they solve:
  \begin{itemize}
  \item Given a formula $P$ (in \emph{conjunctive normal form}), find
    a valuation $v$ that makes it $\true$ and return
    $\mathsf{SAT}(v)$, or if there is no such valuation, return
    $\mathsf{UNSAT}$.
  \end{itemize}
\end{frame}

\begin{frame}
  {Solving SAT}

  \begin{itemize}
  \item In the worst case, there are $2^n$ cases to check, where $n$
    is the number of atomic propositions.
    \begin{itemize}
    \item Checking each case is quick ... but there are a lot of cases.
    \end{itemize}
  \item This is the archetypal NP problem:
    \begin{itemize}
    \item If we knew the answer, it would be easy to check \\
      \qquad (\textbf{P}olynomial time)
    \item But there are exponentially many to check \\
      \qquad (\textbf{N}ondeterminism)
    \end{itemize}
  \item It is unknown if there is a better way. Does P = NP?
  \end{itemize}
\end{frame}

\begin{frame}
  {But SAT is useful: Solving Problems}

  \begin{enumerate}
  \item Package installations (last lecture) \\
    \sidenote{satisfying valuation = good package installation}
  \item Solving Sudoku \\
    \sidenote{satisfying valuation = correct solution}
  \item Solving Resource allocations \\
    \sidenote{satisfying valuation = feasible resource allocation}
  \end{enumerate}

\end{frame}

\begin{frame}
  {SAT is Useful: Finding Bugs}

  \sidenote{Recall: $P_1 \to P_2 \to Q$ is valid if $\lnot(P_1 \to P_2 \to Q)$ is not satisfiable}
  \begin{enumerate}
  \item Finding faults in systems \\
    \sidenote{satisfying valuation = path to a bad state}
  \item Finding flaws in Access Control rules \\
    \sidenote{satisfying valuation = unexpectedly permitted request}
  \item Verifying hardware \\
    \sidenote{satisfying valuation = counterexample to correctness}
  \end{enumerate}
\end{frame}


\begin{frame}
  {An alluring proposition}

  Instead of writing custom solvers for all these problems, we:
  \begin{enumerate}
  \item translate into propositional logic; and
  \item use an off the shelf SAT solver.
  \end{enumerate}
\end{frame}

\begin{frame}
  {Solving the problem in practice}

  Despite the $2^n$ worst case time, practical SAT solvers are possible:
  \begin{enumerate}
  \item Solvers don't blindly check all cases:
    \begin{itemize}
    \item Use the formula to guide the search;
    \item Analyse dead ends to avoid finding them more than once;
    \item Very efficient data structures.
    \end{itemize}
  \item Human-made problems tend to be quite regular.
  \item Modern SAT solvers can handle
    \begin{itemize}
    \item 10s of thousands of variables
    \item millions of clauses
    \end{itemize}
  \item Practical tools for solving real-world problems.
  \end{enumerate}
\end{frame}

\begin{frame}
  {Input for SAT solvers}

  SAT solvers take input in \emph{Conjunctive Normal Form} (CNF):
  \begin{displaymath}
    \begin{array}{cl}
      &(\lnot a \lor \lnot b \lor \lnot c) \\
      \land&(\lnot b \lor \lnot c \lor \lnot d) \\
      \land&(\lnot a \lor \lnot b \lor c) \\
      \land&b
    \end{array}
  \end{displaymath}
  \begin{enumerate}
  \item Entire formula is a conjunction $C_1 \land C_2 \land \cdots \land C_n$
  \item where each \emph{clause} $C_i = L_{i,1} \lor L_{i,2} \lor \cdots \lor L_{i,k}$
  \item where each \emph{literal} $L_{i,j} = x_{i,j}$ or $L_{i,j} = \lnot x_{i,j}$
  \end{enumerate}

  \raggedleft
  {\footnotesize \textcolor{black!60}{Every formula can be put into CNF (later)}}

  \raggedright

\end{frame}

\begin{frame}
  {Conjunctive Normal Form}
  For the package installation problems, we already have CNF:
  \begin{displaymath}
    \begin{array}{ll}
      \left.
      \begin{array}{cl}
        &(\lnot \mathit{libD}_1 \lor \lnot \mathit{libD}_2) \\
        \land&(\lnot \mathit{libC}_1 \lor \lnot \mathit{libC}_2) \\
        \land&(\lnot \mathit{progA}_1 \lor \lnot \mathit{progA}_2)
      \end{array}
      \right\}
      &
        \textsc{Incompat}
      \\
      \left.
      \begin{array}{cl}
        \land&(\lnot \mathit{progA}_1 \lor \mathit{libC}_1) \\
        \land&(\lnot \mathit{progA}_2 \lor \mathit{libC}_2) \\
        \land&(\lnot \mathit{libC}_1 \lor \mathit{libD}_2) \\
        \land&(\lnot \mathit{libC}_2 \lor \mathit{libD}_2)
      \end{array}
               \right\} & \textsc{Dep}
      \\
      \left.\begin{array}{cl}
        \land &(\mathit{progA}_1 \lor \mathit{progA}_2)
      \end{array}\right.
    \end{array}
  \end{displaymath}

\end{frame}

\begin{frame}
  {A SAT Solver's job}

  Given clauses that look like:
  \begin{displaymath}
    \begin{array}{cl}
      &(\lnot a \lor \lnot b \lor \lnot c) \\
      \land&(\lnot b \lor \lnot c \lor \lnot d) \\
      \land&(\lnot a \lor \lnot b \lor c) \\
      \land&b
    \end{array}
  \end{displaymath}

  To find a valuation $v$ for the $a, ...$ such that at least one
  literal in every clause is true.
  \begin{displaymath}
    \textrm{Returns either:} \quad \mathsf{SAT}(v) \qquad \textrm{or} \qquad \mathsf{UNSAT}.
  \end{displaymath}
\end{frame}

\begin{frame}
  {Basic idea of the algorithm}

  \begin{enumerate}
  \item The clauses $C_1, \dots, C_n$ to be satisfied are fixed;
  \item The state is a partial valuation \textcolor{black!60}{(next slide)};
  \item At each step we pick a way to modify the current partial
    valuation by choosing from a collection of rules;
  \item Algorithm terminates when either a satisfying valuation is
    constructed, or it is clear that this is not possible.
  \end{enumerate}

  \bigskip

  This is known as the \emph{DPLL Algorithm}.
\end{frame}

\begin{frame}
  {Partial Valuations}

  To describe what a SAT solver does, we need \emph{partial valuations}.

  \bigskip

  A \rhighlight{partial valuation} $v^?$ is a:
  \begin{itemize}
  \item \emph{sequence} of assignments to atoms; with each one marked
    \begin{enumerate}
    \item $d$ecision point, if we guessed this value.
    \item $f$orced, if we were forced to have this value.
    \end{enumerate}
  \end{itemize}

\medskip

  Examples:
  $\begin{array}[t]{l}
      v_1^? = [ a \decideto \true, b \decideto \false, c \forcedto \true ] \\
      v_2^? = [ a \forcedto \false, b \decideto \false ] \\
    \end{array}$
\end{frame}

\begin{frame}
  {Differences with Valuations}

  \begin{enumerate}
  \item The order matters\\
    \sidenote{we keep track of what decisions we make during the search}
  \item Not all atoms need an assignment\\
    \sidenote{we want to represent partial solutions during the search}
  \item We mark decision points and forced decisions.
  \end{enumerate}
\end{frame}

\begin{frame}
  {Notation}
  We write
  \begin{displaymath}
    v_1^?, a \decideto x, v_2^?
  \end{displaymath}
  for a partial valuation with $a \decideto x$ somewhere in
  the middle.

  \bigskip

  We write
  \begin{displaymath}
    \mathit{decisionfree}(v^?)
  \end{displaymath}
  if none of the assignments in $v^?$ are marked $d$ \\
  \sidenote{i.e., all decisions in $v^?$ are $f$orced}
\end{frame}

\begin{frame}
  {1. Initialisation}

  We start with the \emph{empty partial valuation} $v^? = []$.\\
  \sidenote{We make no commitments}

  \medskip

  We must extend this guess to a valuation that satisfies all the clauses.

\end{frame}

\begin{frame}
  {2. Guessing}

  If there is an atom $a$ in the clauses that is not in the current
  partial valuation $v^?$, then we can make a guess. We pick one of:
  \begin{displaymath}
    v^?, a \decideto \true \qquad \textrm{or} \qquad v^?, a \decideto \false
  \end{displaymath}
  \sidenote{Note: we have marked this as a $d$ecision point}

\end{frame}

\begin{frame}
  {3. Success}

  If the current $v^?$ makes all the clauses true (for all $i$,
  $\sem{C_i}v^? = \true$), then stop with $\mathsf{SAT}(v^?)$.
\end{frame}

\def\doneafter<#1>#2{%
  \temporal<#1>%
  {\colorbox{white}{\color{black}$\displaystyle#2$}}%
  {\colorbox{green!50}{\color{black}$\displaystyle#2$}}%
  {\colorbox{green!50}{\color{black}$\displaystyle#2$}}}

\def\stuck<#1>#2{%
  \temporal<#1>%
  {\colorbox{white}{\color{black}$\displaystyle#2$}}%
  {\colorbox{red!50}{\color{black}$\displaystyle#2$}}%
  {\colorbox{white}{\color{black}$\displaystyle#2$}}}

\def\done<#1>#2{%
  \temporal<#1>%
  {\colorbox{white}{\color{black}$\displaystyle#2$}}%
  {\colorbox{green!50}{\color{black}$\displaystyle#2$}}%
  {\colorbox{white}{\color{black}$\displaystyle#2$}}}

\newcommand{\undet}[1]{\stackrel{\textcolor{white}{\checkmark}}{\colorbox{white}{$#1$}}}
\newcommand{\failed}[1]{\stackrel{\times}{\colorbox{red!90}{\textcolor{white}{$#1$}}}}
\newcommand{\satd}[1]{\stackrel{\checkmark}{\colorbox{green!50}{{$#1$}}}}

\begin{frame}
  {Example}
  \begin{displaymath}
    \only<1>{(\undet{\lnot a} \lor \undet{\lnot b} \lor \undet{\lnot c}) \land (\undet{\lnot b} \lor \undet{\lnot c} \lor \undet{\lnot d}) \land (\undet{\lnot a} \lor \undet{\lnot b} \lor \undet{c}) \land \undet{b}}%
    \only<2>{(\satd{\lnot a} \lor \undet{\lnot b} \lor \undet{\lnot c}) \land (\undet{\lnot b} \lor \undet{\lnot c} \lor \undet{\lnot d}) \land (\satd{\lnot a} \lor \undet{\lnot b} \lor \undet{c}) \land \undet{b}}%
    \only<3>{(\satd{\lnot a} \lor \failed{\lnot b} \lor \undet{\lnot c}) \land (\failed{\lnot b} \lor \undet{\lnot c} \lor \undet{\lnot d}) \land (\satd{\lnot a} \lor \failed{\lnot b} \lor \undet{c}) \land \satd{b}}%
    \only<4>{(\satd{\lnot a} \lor \failed{\lnot b} \lor \satd{\lnot c}) \land (\failed{\lnot b} \lor \satd{\lnot c} \lor \undet{\lnot d}) \land (\satd{\lnot a} \lor \failed{\lnot b} \lor \failed{c}) \land \satd{b}}%
    \only<5->{(\satd{\lnot a} \lor \failed{\lnot b} \lor \satd{\lnot c}) \land (\failed{\lnot b} \lor \satd{\lnot c} \lor \satd{\lnot d}) \land (\satd{\lnot a} \lor \failed{\lnot b} \lor \failed{c}) \land \satd{b}}%
  \end{displaymath}
  \sidenote{Need at least one green in every clause}

  \sechead{Sequence of (lucky) guesses}
  \begin{enumerate}
  \item $[]$
  \item<2-> $[a \decideto \false]$
  \item<3-> $[a \decideto \false, b \decideto \true]$
  \item<4-> $[a \decideto \false, b \decideto \true, c \decideto \false]$
  \item<5-> $[a \decideto \false, b \decideto \true, c \decideto \false, d \decideto \false]$, a satisfying valuation.
  \end{enumerate}

  % \pause\pause\pause\pause\pause
  % Line 5 is a satisfying valuation.

  % \pause
  % \bigskip

  % But we can't program ``luck''!
\end{frame}

\begin{frame}
  But we can't program ``luck''!
\end{frame}

\begin{frame}
  {4. Backtracking}

  If we have a partial valuation:
  \begin{displaymath}
    v_1^?, a \decideto x, v_2^?
  \end{displaymath}
  and $\mathit{decisionfree}(v_2^?)$ (so $a \assigned x$ was our most recent guess).

  \medskip

  Then we backtrack (throw away $v_2^?$) and change our mind:
  \begin{displaymath}
    v_1^?, a \forcedto \lnot x
  \end{displaymath}
  marking the assignment as $f$orced.

\end{frame}

\begin{frame}
  {5. Failure}

  If all decisions are forced ($\mathit{decisionfree}(v^?)$), and
  there is at least one clause $C_i$ such that $\sem{C}v^? = \false$,
  then return $\mathsf{UNSAT}$.

\end{frame}

\begin{frame}[t]
  {\footnotesize
    $\only<1>{(\undet{\lnot a} \lor \undet{\lnot b} \lor \undet{\lnot c}) \land (\undet{\lnot b} \lor \undet{\lnot c} \lor \undet{\lnot d}) \land (\undet{\lnot a} \lor \undet{\lnot b} \lor \undet{c}) \land \undet{b}}%
    \only<2>{(\failed{\lnot a} \lor \undet{\lnot b} \lor \undet{\lnot c}) \land (\undet{\lnot b} \lor \undet{\lnot c} \lor \undet{\lnot d}) \land (\failed{\lnot a} \lor \undet{\lnot b} \lor \undet{c}) \land \undet{b}}%
    \only<3>{(\failed{\lnot a} \lor \failed{\lnot b} \lor \undet{\lnot c}) \land (\failed{\lnot b} \lor \undet{\lnot c} \lor \undet{\lnot d}) \land (\failed{\lnot a} \lor \failed{\lnot b} \lor \undet{c}) \land \satd{b}}%
    \only<4>{(\failed{\lnot a} \lor \failed{\lnot b} \lor \failed{\lnot c}) \land (\failed{\lnot b} \lor \failed{\lnot c} \lor \undet{\lnot d}) \land (\failed{\lnot a} \lor \failed{\lnot b} \lor \satd{c}) \land \satd{b}}%
    \only<5>{(\failed{\lnot a} \lor \failed{\lnot b} \lor \satd{\lnot c}) \land (\failed{\lnot b} \lor \satd{\lnot c} \lor \undet{\lnot d}) \land (\failed{\lnot a} \lor \failed{\lnot b} \lor \failed{c}) \land \satd{b}}%
    \only<6>{(\failed{\lnot a} \lor \satd{\lnot b} \lor \undet{\lnot c}) \land (\satd{\lnot b} \lor \undet{\lnot c} \lor \undet{\lnot d}) \land (\failed{\lnot a} \lor \satd{\lnot b} \lor \undet{c}) \land \failed{b}}%
    \only<7>{(\satd{\lnot a} \lor \undet{\lnot b} \lor \undet{\lnot c}) \land (\undet{\lnot b} \lor \undet{\lnot c} \lor \undet{\lnot d}) \land (\satd{\lnot a} \lor \undet{\lnot b} \lor \undet{c}) \land \undet{b}}%
    \only<8>{(\satd{\lnot a} \lor \failed{\lnot b} \lor \undet{\lnot c}) \land (\failed{\lnot b} \lor \undet{\lnot c} \lor \undet{\lnot d}) \land (\satd{\lnot a} \lor \failed{\lnot b} \lor \undet{c}) \land \satd{b}}%
    \only<9>{(\satd{\lnot a} \lor \failed{\lnot b} \lor \failed{\lnot c}) \land (\failed{\lnot b} \lor \failed{\lnot c} \lor \undet{\lnot d}) \land (\satd{\lnot a} \lor \failed{\lnot b} \lor \satd{c}) \land \satd{b}}%
    \only<10>{(\satd{\lnot a} \lor \failed{\lnot b} \lor \failed{\lnot c}) \land (\failed{\lnot b} \lor \failed{\lnot c} \lor \failed{\lnot d}) \land (\satd{\lnot a} \lor \failed{\lnot b} \lor \satd{c}) \land \satd{b}}%
    \only<11>{(\satd{\lnot a} \lor \failed{\lnot b} \lor \failed{\lnot c}) \land (\failed{\lnot b} \lor \failed{\lnot c} \lor \satd{\lnot d}) \land (\satd{\lnot a} \lor \failed{\lnot b} \lor \satd{c}) \land \satd{b}}%
  $}

{\footnotesize
  \begin{enumerate}
  \item $[]$
    \vspace{-0.5em}
  \item<2-> $[a \decideto \true]$
    \vspace{-0.5em}
  \item<3-> $[a \decideto \true, b \decideto \true]$
    \vspace{-0.5em}
  \item<4-> $[a \decideto \true, b \decideto \true, c \decideto \true]$ \qquad \emph{clause 1 failed, backtrack...}
    \vspace{-0.5em}
  \item<5-> $[a \decideto \true, b \decideto \true, c \forcedto \false]$ \qquad \emph{clause 3 failed, backtrack...}
    \vspace{-0.5em}
  \item<6-> $[a \decideto \true, b \forcedto \false]$ \qquad \emph{clause 4 failed, backtrack...}
    \vspace{-0.5em}
  \item<7-> $[a \forcedto \false]$
    \vspace{-0.5em}
  \item<8-> $[a \forcedto \false, b \decideto \true]$
    \vspace{-0.5em}
  \item<9-> $[a \forcedto \false, b \decideto \true, c \decideto \true]$
    \vspace{-0.5em}
  \item<10-> $[a \forcedto \false, b \decideto \true, c \decideto \true, d \decideto \true]$ \qquad \emph{clause 2 failed, backtrack}
    \vspace{-0.5em}
  \item<11-> $[a \forcedto \false, b \decideto \true, c \decideto \true, d \decideto \false]$ \qquad $\mathsf{SAT}$
  \end{enumerate}
  ~}
\end{frame}

\begin{frame}
  {Summary}

  \begin{enumerate}
  \item SAT solvers are tools that find satisfying valuations for formulas in CNF.
  \item Having a SAT solver enables solving of problems modelled using logic.
  \item The core algorithm is a backtracking search.
  \end{enumerate}
\end{frame}

\weeksection{Faster SAT by Unit Propagation}

% \begin{frame}
%   \begin{center}
%     {\Huge \textcolor{black!60}{Part 4 : }Faster SAT with Unit Propagation}
%   \end{center}
% \end{frame}

\begin{frame}
  {Backtracking is Oblivious}

  The example:
  \begin{displaymath}
    ({\lnot a} \lor {\lnot b} \lor {\lnot c}) \land ({\lnot b} \lor {\lnot c} \lor {\lnot d}) \land ({\lnot a} \lor {\lnot b} \lor {c}) \land {b}
  \end{displaymath}

  Backtracking tries the atoms in some order.

  \bigskip

  But we can see immediately that $b$ must be true.

  \bigskip

  Other forced assignments occur during the search.
\end{frame}

\begin{frame}
  {Making the Search less naive}

  If we are in a situation like:
  \begin{displaymath}
    (\failed{\lnot b} \lor \failed{\lnot c} \lor \undet{\lnot d})
  \end{displaymath}
  then if the current valuation is to succeed in any way, it must be
  the case that $d \assigned \false$.\\
  \sidenote{because we need at least one literal in every clause to be true.}

  % \bigskip

  % Similarly, in a situation like:
  % \begin{displaymath}
  %   (\failed{\lnot b} \lor \failed{\lnot c} \lor \undet{d})
  % \end{displaymath}
  % then if the current valuation is to succeed in any way, it must be
  % the case that $d \assigned \true$.\\

  \bigskip

  Using this, we can make the search a little less naive.
\end{frame}

\begin{frame}
  {6. Unit Propagation Step}

  (a) If there is a clause $C \lor a$and $\sem{C}v^? = \false$, then we
  extend $v^?$ to:
  \begin{displaymath}
    v^?, a \forcedto \true
  \end{displaymath}

  \medskip

  (b) If there is a clause $C \lor \lnot a$ and $\sem{C}v^? = \false$,
  then we extend $v^?$ to:
  \begin{displaymath}
    v^?, a \forcedto \false
  \end{displaymath}

  \sidenote{Note: the $a$ needn't necessarily appear at the end of the clause}

\end{frame}

\begin{frame}
  {\footnotesize
  \begin{displaymath}
    \only<1>{(\undet{\lnot a} \lor \undet{\lnot b} \lor \undet{\lnot c}) \land (\undet{\lnot b} \lor \undet{\lnot c} \lor \undet{\lnot d}) \land (\undet{\lnot a} \lor \undet{\lnot b} \lor \undet{c}) \land \undet{b}}%
    \only<2>{(\undet{\lnot a} \lor \failed{\lnot b} \lor \undet{\lnot c}) \land (\failed{\lnot b} \lor \undet{\lnot c} \lor \undet{\lnot d}) \land (\undet{\lnot a} \lor \failed{\lnot b} \lor \undet{c}) \land \satd{b}}%
    \only<3>{(\failed{\lnot a} \lor \failed{\lnot b} \lor \undet{\lnot c}) \land (\failed{\lnot b} \lor \undet{\lnot c} \lor \undet{\lnot d}) \land (\failed{\lnot a} \lor \failed{\lnot b} \lor \undet{c}) \land \satd{b}}%
    \only<4>{(\failed{\lnot a} \lor \failed{\lnot b} \lor \satd{\lnot c}) \land (\failed{\lnot b} \lor \satd{\lnot c} \lor \undet{\lnot d}) \land (\failed{\lnot a} \lor \failed{\lnot b} \lor \failed{c}) \land \satd{b}}%
    \only<5>{(\satd{\lnot a} \lor \failed{\lnot b} \lor \undet{\lnot c}) \land (\failed{\lnot b} \lor \undet{\lnot c} \lor \undet{\lnot d}) \land (\satd{\lnot a} \lor \failed{\lnot b} \lor \undet{c}) \land \satd{b}}%
    \only<6>{(\satd{\lnot a} \lor \failed{\lnot b} \lor \failed{\lnot c}) \land (\failed{\lnot b} \lor \failed{\lnot c} \lor \undet{\lnot d}) \land (\satd{\lnot a} \lor \failed{\lnot b} \lor \satd{c}) \land \satd{b}}%
    \only<7-8>{(\satd{\lnot a} \lor \failed{\lnot b} \lor \failed{\lnot c}) \land (\failed{\lnot b} \lor \failed{\lnot c} \lor \satd{\lnot d}) \land (\satd{\lnot a} \lor \failed{\lnot b} \lor \satd{c}) \land \satd{b}}%
  \end{displaymath}}

  \begin{enumerate}
  \item $[]$ \qquad \emph{do unit propagation...}
  \item<2-> $[b \forcedto \true]$
  \item<3-> $[b \forcedto \true, a \decideto \true]$ \qquad \emph{do unit propagation...}
  \item<4-> $[b \forcedto \true, a \decideto \true, c \forcedto \false]$ \qquad \emph{clause 3 failed, backtrack...}
  \item<5-> $[b \forcedto \true, a \forcedto \false]$
  \item<6-> $[b \forcedto \true, a \forcedto \false, c \decideto \true]$ \qquad \emph{do unit propagation...}
  \item<7-> $[b \forcedto \true, a \forcedto \false, c \decideto \true, d \forcedto \false]$ \qquad $\mathsf{SAT}$
  \end{enumerate}

  \pause\pause\pause\pause\pause\pause\pause

  \bigskip

  One backtrack vs. four without unit propagation.
\end{frame}

\begin{frame}[t]
  {2-SAT}
  If every clause has at most two literals, UP means less backtracking:

  \begin{displaymath}
    \only<1>{
      \begin{array}{clcl}
        &(\undet{\lnot \mathit{libD}_1} \lor \undet{\lnot \mathit{libD}_2})
        &\land& (\undet{\lnot \mathit{libC}_1} \lor \undet{\lnot \mathit{libC}_2}) \\
        \land&(\undet{\lnot \mathit{progA}_1} \lor \undet{\lnot \mathit{progA}_2})
        &\land& (\undet{\lnot \mathit{progA}_1} \lor \undet{\mathit{libC}_1}) \\
        \land&(\undet{\lnot \mathit{progA}_2} \lor \undet{\mathit{libC}_2})
        &\land& (\undet{\lnot \mathit{libC}_1} \lor \undet{\mathit{libD}_2}) \\
        \land&(\undet{\lnot \mathit{libC}_2} \lor \undet{\mathit{libD}_2})
        &\land&(\undet{\mathit{progA}_1} \lor \undet{\mathit{progA}_2})
      \end{array}}
    \only<2>{
      \begin{array}{clcl}
        &(\undet{\lnot \mathit{libD}_1} \lor \undet{\lnot \mathit{libD}_2})
        &\land& (\undet{\lnot \mathit{libC}_1} \lor \undet{\lnot \mathit{libC}_2}) \\
        \land&(\failed{\lnot \mathit{progA}_1} \lor \undet{\lnot \mathit{progA}_2})
        &\land& (\failed{\lnot \mathit{progA}_1} \lor \undet{\mathit{libC}_1}) \\
        \land&(\undet{\lnot \mathit{progA}_2} \lor \undet{\mathit{libC}_2})
        &\land& (\undet{\lnot \mathit{libC}_1} \lor \undet{\mathit{libD}_2}) \\
        \land&(\undet{\lnot \mathit{libC}_2} \lor \undet{\mathit{libD}_2})
        &\land& (\satd{\mathit{progA}_1} \lor \undet{\mathit{progA}_2})
      \end{array}}
    \only<3>{
      \begin{array}{clcl}
        &(\undet{\lnot \mathit{libD}_1} \lor \undet{\lnot \mathit{libD}_2})
        &\land& (\undet{\lnot \mathit{libC}_1} \lor \undet{\lnot \mathit{libC}_2}) \\
        \land&(\failed{\lnot \mathit{progA}_1} \lor \satd{\lnot \mathit{progA}_2})
        &\land& (\failed{\lnot \mathit{progA}_1} \lor \undet{\mathit{libC}_1}) \\
        \land&(\satd{\lnot \mathit{progA}_2} \lor \undet{\mathit{libC}_2})
        &\land& (\undet{\lnot \mathit{libC}_1} \lor \undet{\mathit{libD}_2}) \\
        \land&(\undet{\lnot \mathit{libC}_2} \lor \undet{\mathit{libD}_2})
        &\land& (\satd{\mathit{progA}_1} \lor \failed{\mathit{progA}_2})
      \end{array}}
    \only<4>{
      \begin{array}{clcl}
        &(\undet{\lnot \mathit{libD}_1} \lor \undet{\lnot \mathit{libD}_2})
        &\land& (\failed{\lnot \mathit{libC}_1} \lor \undet{\lnot \mathit{libC}_2}) \\
        \land&(\failed{\lnot \mathit{progA}_1} \lor \satd{\lnot \mathit{progA}_2})
        &\land& (\failed{\lnot \mathit{progA}_1} \lor \satd{\mathit{libC}_1}) \\
        \land&(\satd{\lnot \mathit{progA}_2} \lor \undet{\mathit{libC}_2})
        &\land& (\failed{\lnot \mathit{libC}_1} \lor \undet{\mathit{libD}_2}) \\
        \land&(\undet{\lnot \mathit{libC}_2} \lor \undet{\mathit{libD}_2})
        &\land& (\satd{\mathit{progA}_1} \lor \failed{\mathit{progA}_2})
      \end{array}}
    \only<5>{
      \begin{array}{clcl}
        &(\undet{\lnot \mathit{libD}_1} \lor \undet{\lnot \mathit{libD}_2})
        &\land& (\failed{\lnot \mathit{libC}_1} \lor \satd{\lnot \mathit{libC}_2}) \\
        \land&(\failed{\lnot \mathit{progA}_1} \lor \satd{\lnot \mathit{progA}_2})
        &\land& (\failed{\lnot \mathit{progA}_1} \lor \satd{\mathit{libC}_1}) \\
        \land&(\satd{\lnot \mathit{progA}_2} \lor \failed{\mathit{libC}_2})
        &\land& (\failed{\lnot \mathit{libC}_1} \lor \undet{\mathit{libD}_2}) \\
        \land&(\satd{\lnot \mathit{libC}_2} \lor \undet{\mathit{libD}_2})
        &\land& (\satd{\mathit{progA}_1} \lor \failed{\mathit{progA}_2})
      \end{array}}
    \only<6>{
      \begin{array}{clcl}
        &(\undet{\lnot \mathit{libD}_1} \lor \failed{\lnot \mathit{libD}_2})
        &\land& (\failed{\lnot \mathit{libC}_1} \lor \satd{\lnot \mathit{libC}_2}) \\
        \land&(\failed{\lnot \mathit{progA}_1} \lor \satd{\lnot \mathit{progA}_2})
        &\land& (\failed{\lnot \mathit{progA}_1} \lor \satd{\mathit{libC}_1}) \\
        \land&(\satd{\lnot \mathit{progA}_2} \lor \failed{\mathit{libC}_2})
        &\land& (\failed{\lnot \mathit{libC}_1} \lor \satd{\mathit{libD}_2}) \\
        \land&(\satd{\lnot \mathit{libC}_2} \lor \satd{\mathit{libD}_2})
        &\land& (\satd{\mathit{progA}_1} \lor \failed{\mathit{progA}_2})
      \end{array}}
    \only<7>{
      \begin{array}{clcl}
        &(\satd{\lnot \mathit{libD}_1} \lor \failed{\lnot \mathit{libD}_2})
        &\land& (\failed{\lnot \mathit{libC}_1} \lor \satd{\lnot \mathit{libC}_2}) \\
        \land&(\failed{\lnot \mathit{progA}_1} \lor \satd{\lnot \mathit{progA}_2})
        &\land& (\failed{\lnot \mathit{progA}_1} \lor \satd{\mathit{libC}_1}) \\
        \land&(\satd{\lnot \mathit{progA}_2} \lor \failed{\mathit{libC}_2})
        &\land& (\failed{\lnot \mathit{libC}_1} \lor \satd{\mathit{libD}_2}) \\
        \land&(\satd{\lnot \mathit{libC}_2} \lor \satd{\mathit{libD}_2})
        &\land& (\satd{\mathit{progA}_1} \lor \failed{\mathit{progA}_2})
      \end{array}}
  \end{displaymath}

  \begin{displaymath}
    \only<1>{[]}
    \only<2>{[\mathit{progA}_1 \decideto \true]}
    \only<3>{[\mathit{progA}_1 \decideto \true, \mathit{progA}_2 \forcedto \false]}
    \only<4>{[\mathit{progA}_1 \decideto \true, \mathit{progA}_2 \forcedto \false, \mathit{libC}_1 \forcedto \true]}
    \only<5>{[\mathit{progA}_1 \decideto \true, \mathit{progA}_2 \forcedto \false, \mathit{libC}_1 \forcedto \true, \mathit{libC}_2 \forcedto \false]}
    \only<6>{[\mathit{progA}_1 \decideto \true, \mathit{progA}_2 \forcedto \false, \mathit{libC}_1 \forcedto \true, \mathit{libC}_2 \forcedto \false, \mathit{libD}_2 \forcedto \true]}
    \only<7>{[\mathit{progA}_1 \decideto \true, \mathit{progA}_2 \forcedto \false, \mathit{libC}_1 \forcedto \true, \mathit{libC}_2 \forcedto \false, \mathit{libD}_2 \forcedto \true, \mathit{libD}_1 \forcedto \false]}
  \end{displaymath}
\end{frame}

\begin{frame}
  {2-SAT}

  If every clause has at most two literals,

  \begin{itemize}
  \item UP means at most one backtrack
  \item Means that we can solve the problem in polynomial time
  \item So for the $n$-SAT problem:
    \begin{itemize}
    \item If $n \leq 2$, there is a fast polynomial time algorithm
    \item If $n \geq 3$, no known general fast algorithm
    \end{itemize}
  \end{itemize}
\end{frame}

\begin{frame}
  {Summary of the Rules 1}

  \begin{displaymath}
    \begin{array}{@{}llcl@{\qquad}l}
      \TirName{DecideTrue}    & v^? & \Longrightarrow & v^?, a \decideto \true & \textit{if $a$ is not assigned in $v^?$} \\
      \\
      \TirName{DecideFalse}   & v^? & \Longrightarrow & v^?, a \decideto \false & \textit{if $a$ is not assigned in $v^?$} \\
      \\
      \TirName{Success}       & v^? & \Longrightarrow & \mathsf{SAT}(v^?) & \textit{if $v^?$ makes all the} \\
      &&&&\textit{clauses true.} \\
  \end{array}
\end{displaymath}
\end{frame}

\begin{frame}
  {Summary of the Rules 2}

  \begin{displaymath}
    \begin{array}{@{}llcl@{\qquad}l}
      \TirName{BackTrack}     & v_1^?, a \decideto x, v_2^? & \Longrightarrow & v_1^?, a \forcedto \lnot x & \textit{if $v_2^?$ is decision free} \\
      \\
      \TirName{Fail}          & v^? & \Longrightarrow & \mathsf{UNSAT} & \textit{if $v^?$ is decision free, and}\\
                              &&&&\textit{makes at least one clause}\\
      &&&&\textit{false.} \\
  \end{array}
\end{displaymath}
\end{frame}

\begin{frame}
  {Summary of the Rules 3}

  \begin{displaymath}
    \begin{array}{@{}llcl@{\qquad}l}
      \TirName{UnitPropTrue}  & v^? & \Longrightarrow & v^?, a \forcedto \true & \textit{if there is a clause $C \lor a$}\\
      &&&&\textit{and $\sem{C}(v^?) = \false$} \\
      \\
      \TirName{UnitPropFalse} & v^? & \Longrightarrow & v^?, a \forcedto \false & \textit{if there is a clause $C \lor \lnot a$}\\
      &&&&\textit{and $\sem{C}(v^?) = \false$} \\
  \end{array}
\end{displaymath}
\end{frame}

\begin{frame}[t]
  {Real SAT solvers}
  Use very efficient data structures.
  \sidenote{Key is very fast unit propagation}

  \medskip

  Use heuristics to guide the search:
  \begin{itemize}
  \item Which atom to try next? (not just $a, b, c, ...$)
  \item Whether to try $\true$ or $\false$ first?
  \end{itemize}

  \bigskip

  Incorporate additional rules:
  \begin{itemize}
  \item Non-chronological backjumping \\
    \sidenote{skip several decision points by analysing conflicts}
  \item Clause learning to avoid doing the same work over again.
  \item “CDCL” (Conflict Driven Clause Learning)
  \item Random walk between possible valuations “WalkSAT”.
  \end{itemize}
\end{frame}

\begin{frame}[t]
  {Further Reading}
  A blog post with a Python implementation: \\
  \quad \emph{Understanding SAT by Implementing a Simple SAT Solver in Python} \\
  \quad \textcolor{black!60}{Sahand Saba} \\
  \quad {\tiny \url{https://sahandsaba.com/understanding-sat-by-implementing-a-simple-sat-solver-in-python.html}}

  \bigskip

  Another blog post with more formalism: \\
  \quad \emph{A Primer on Boolean Satisfiability} \\
  \quad \textcolor{black!60}{Emina Torlak} \\
  \quad {\tiny \url{https://homes.cs.washington.edu/~emina/blog/2017-06-23-a-primer-on-sat.html}} \\
  \qquad {\footnotesize See also the links at the end for lots more detail.}
\end{frame}

\begin{frame}[t]
  {More Further Reading}

  For more breadth and detail than you could possibly imagine:\\
  \quad \emph{The Art of Computer Programming: 7.2.2.2 Satisfiability} \\
  \qquad \emph{Draft: Volume 4B, Pre-fascicle 6A} \\
  \quad \textcolor{black!60}{Donald E. Knuth} \\
  \quad {\tiny \url{https://cs.stanford.edu/~knuth/fasc6a.ps.gz}}
\end{frame}

\begin{frame}
  {Summary}

  \begin{itemize}
  \item Unit Propagation speeds up SAT Solving \\
    \sidenote{by using the structure of the problem}
  \item This makes 2-SAT very fast
  \item Real SAT Solvers are very sophisticated.
  \end{itemize}
\end{frame}


\weeksection{Decidable Theories}

\begin{frame}
  {Decidable Theories}

  TBD
\end{frame}

\weeksection{Quantifier Elimination}

\begin{frame}
  {Fourier-Motzkin Elimination}

  TBD
\end{frame}

\end{document}
