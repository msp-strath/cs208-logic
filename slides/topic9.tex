% -*- TeX-engine: xetex -*-

\documentclass[xetex,aspectratio=169,14pt,hyperref={pdfpagelabels=true,pdflang={en-GB}}]{beamer}

\usepackage[sci,noslidestrathidentity]{strathclyde}
\strathsetidentity{Department of}{Computer \& Information Sciences}

\usepackage{lmodern}
\usepackage{subscript}
\usepackage{url}
\usepackage{pifont}
\usepackage{csquotes}
\usepackage{mathpartir}
\usepackage{stmaryrd}
\usepackage{multirow}
\usepackage{euler}
\usepackage[normalem]{ulem}

% 2013-09-08 SU adding xetex specifics
% http://www.woggie.net/2008/07/16/beamer-pdftex-and-xetex/
\usepackage{xltxtra}

% 2013-09-08 SU http://robjhyndman.com/hyndsight/xelatex/
\defaultfontfeatures{Ligatures=TeX}

% 2015-12-09 SU table beautified
\renewcommand{\arraystretch}{1.2}
\usepackage{booktabs}

%MGK compatibility
\newcommand{\hh}[1]
  {\medskip\textbf{\large #1}}
\newcommand{\pdu}[3]
  {#1\rightarrow #2 : &\ & \makebox[70mm][l]{$#3$}}



\setlength{\marginparwidth}{2cm}
\usepackage{todonotes}%[disable]
\let\OldTodo\todo
\renewcommand{\todo}{\OldTodo[inline]}%
\newcommand{\todolater}[1]{}% Things to do for next year


\DeclareTextCommandDefault{\nobreakspace}{\leavevmode\nobreak\ }
\usepackage{rotating}

\usepackage{pdfcomment}% To add alt text for images using \pdftooltip{}

\usepackage{appendixnumberbeamer}

\newcommand{\messageframe}[1]{\begin{frame}\begin{center}\Huge #1\end{center}\end{frame}}
\newcommand{\sechead}[1]{{\bf #1} \\}
\newcommand{\examplehead}[1]{{\bf Example:} {\it\textcolor{red!90}{#1}} \\}
\newcommand{\eqnote}[1]{\hspace{3cm}\textit{#1}}
\newcommand{\sidenote}[1]{\qquad {\footnotesize \textcolor{black!60}{(#1)}}}
\newcommand{\sem}[1]{\llbracket #1 \rrbracket}
\newcommand{\true}{\mathsf{T}}
\newcommand{\false}{\mathsf{F}}

\def\strikeafter<#1>#2{\temporal<#1>{#2}{\sout{#2}}{\sout{#2}}}


%\newcommand{\rhighlight}{\textcolor{titlered}}
\newcommand{\rhighlight}{\textbf}
\newcommand{\highlight}{\textbf}

% \setmainfont{Linux Biolinum O}
\setmainfont{LinBiolinum}[
Path=,
UprightFont = *_R.otf ,
BoldFont = *_RB.otf ,
ItalicFont = *_RI.otf
]

\setbeamertemplate{navigation symbols}{}
%\usecolortheme[rgb={0.8,0,0}]{structure}
\usefonttheme{serif}
\usefonttheme{structurebold}
\setbeamercolor{description item}{fg=black}


\author[Atkey]{Dr.~Robert Atkey}
\institute[Strathclyde]{Computer \& Information Sciences}
\date[]{}

\newcommand{\weeksection}[1]{%
  \section{\thetitle{}, Part~\thesection : #1}
  \begin{frame}
    \begin{center}
      \textcolor{black!60}{\thetitle{}, Part \thesection}\\
      {\Huge #1}
    \end{center}
  \end{frame}}

\newcommand{\weektitle}[2]{\def\thetitle{#2}
\title[CS208 - Topic #1]{CS208 (Semester 1) Topic #1 : #2}}

\newcommand{\assigned}{:}

\newcommand{\forcedto}{\assigned_f}
\newcommand{\decideto}{\assigned_d}


\weektitle{9}{Automating Logic}

\begin{document}

\frame{\titlepage}

\weeksection{Automating Logic}

\begin{frame}
  {What to Automate?}

  Given a formula $P$, we can ask:
  \begin{enumerate}
  \item Does $P$ have a proof?
  \item Is $P$ valid?
  \item Is $P$ satisfiable?
  \item For a model $\mathcal{M}$, do we have $\mathcal{M} \models P$?
  \item Can we generate $\mathcal{M}$, such that $\mathcal{M} \models P$?
  \item How many models does $P$ have?
  \end{enumerate}

  \bigskip

  Given the expressiveness of Predicate Logic, this covers a large
  range of questions.
\end{frame}

\begin{frame}
  {The Bad News}

  Lots of things are undecidable:
  \begin{enumerate}
  \item Validity in Predicate Logic
  \item Entailment in Predicate Logic
  \item Checking models of Predicate Logic
  \item Model generation for Predicate Logic
  \end{enumerate}

  \bigskip

  \textcolor{black!60}{Undecidable: No program that can perfectly say
    ``yes'' or ``no''}.
\end{frame}

\begin{frame}
  {More Bad News}

  Or theoretically intractable:
  \begin{enumerate}
  \item Validity / Satisfiability checking in Propositional Logic
  \item Synthesis of finite models
  \item Checking of finite models
  \end{enumerate}
  \textcolor{black!60}{Intractable: there is no (known) program to
    solve it that runs in better than $O(2^n)$ on inputs of size $n$.}

\end{frame}

\begin{frame}
  {The Good News}

  For undecidable problems, there are good \emph{semi-decision}
  procedures. \\
  \textcolor{black!60}{Semi-decision: says ``yes'' exactly when the
    problem is solved. No guarantees otherwise.}

  \bigskip

  For intractable problems, there are heuristics that solve common
  cases quickly.

\end{frame}

\begin{frame}
  {Many Algorithms}

  \begin{enumerate}
  \item Proof Search \\
    \textcolor{black!60}{Try all possible proof rules, under some strategy}
  \item Resolution Provers \\
    \textcolor{black!60}{A proof system specialised to proof search}
  \item Specific tools for sub-languages of Predicate Logic \\
    \textcolor{black!60}{E.g., ``Horn'' clause provers}
  \item SAT / SMT Solvers \\
    \textcolor{black!60}{SATisfiabilty solvers} \\
    \textcolor{black!60}{SATisfiabilty Modulo Theory solvers}
  \end{enumerate}

  \pause
  \bigskip

  We will look at SAT / SMT solvers here.
\end{frame}

\begin{frame}
  {SAT / SMT}

  SAT solvers work on Propositional Logic.

  \bigskip

  SMT solvers work on a \emph{quantifier-free} fragment of Predicate Logic.

  \bigskip

  Lots of industrial strength tools: Z3, CVC5, Yices, ...

  \bigskip

  Used by (e.g.) Amazon Web Services to check access control rules, Microsoft to verify software, ...
\end{frame}

\weeksection{SAT Solvers}

\begin{frame}
  {SAT solvers}

  SATisfiability solvers.

  \bigskip

  The problem they solve:
  \begin{itemize}
  \item Given a formula $P$ (in \emph{conjunctive normal form}), find
    a valuation $v$ that makes it $\true$ and return
    $\mathsf{SAT}(v)$, or if there is no such valuation, return
    $\mathsf{UNSAT}$.
  \end{itemize}
\end{frame}

\begin{frame}
  {Solving SAT}

  \begin{itemize}
  \item In the worst case, there are $2^n$ cases to check, where $n$
    is the number of atomic propositions.
    \begin{itemize}
    \item Checking each case is quick ... but there are a lot of cases.
    \end{itemize}
  \item This is the archetypal NP problem:
    \begin{itemize}
    \item If we knew the answer, it would be easy to check \\
      \qquad (\textbf{P}olynomial time)
    \item But there are exponentially many to check \\
      \qquad (\textbf{N}ondeterminism)
    \end{itemize}
  \item It is unknown if there is a better way. Does P = NP?
  \end{itemize}
\end{frame}

\begin{frame}
  {Encoding Problems into SAT}

  In general, if we want to prove that $P$ is valid, then it suffices
  to show that $\lnot P$ is not satisifable:

  \bigskip

  \begin{enumerate}
  \item Take $P$
  \item Put $\lnot P$ into a SAT solver:
    \begin{enumerate}
    \item if $\lnot P$ is satisfiable, then we have a counter example to $P$
    \item if $\lnot P$ is not satisfiable, then $P$ is valid
    \end{enumerate}
  \end{enumerate}

  \bigskip

  Also, there are \emph{many} problems that can be encoded as
  Propositional Logic formulas.

\end{frame}

\begin{frame}
  {But SAT is useful: Solving Problems}

  \begin{enumerate}
  \item Package installations \\
    \sidenote{satisfying valuation = good package installation}
  \item Solving Sudoku \\
    \sidenote{satisfying valuation = correct solution}
  \item Solving Resource allocations \\
    \sidenote{satisfying valuation = feasible resource allocation}
  \end{enumerate}

\end{frame}

\begin{frame}
  {SAT is Useful: Finding Bugs}

%  \sidenote{Recall: $P_1 \to P_2 \to Q$ is valid if $\lnot(P_1 \to P_2 \to Q)$ is not satisfiable}
  \begin{enumerate}
  \item Finding faults in systems \\
    \sidenote{satisfying valuation = path to a bad state}
  \item Finding flaws in Access Control rules \\
    \sidenote{satisfying valuation = unexpectedly permitted request}
  \item Verifying hardware \\
    \sidenote{satisfying valuation = counterexample to correctness}
  \end{enumerate}
\end{frame}

\begin{frame}
  {An alluring proposition}

  Instead of writing custom solvers for all these problems, we:
  \begin{enumerate}
  \item translate into propositional logic; and
  \item use an off the shelf SAT solver.
  \end{enumerate}
\end{frame}

\begin{frame}
  {Solving the problem in practice}

  Despite the $2^n$ worst case time, practical SAT solvers are possible:
  \begin{enumerate}
  \item Solvers don't blindly check all cases:
    \begin{itemize}
    \item Use the formula to guide the search;
    \item Analyse dead ends to avoid finding them more than once;
    \item Very efficient data structures.
    \end{itemize}
  \item Human-made problems tend to be quite regular.
  \item Modern SAT solvers can handle
    \begin{itemize}
    \item 10s of thousands of variables
    \item millions of clauses
    \end{itemize}
  \item Practical tools for solving real-world problems.
  \end{enumerate}
\end{frame}

\begin{frame}
  {Input for SAT solvers}

  SAT solvers take input in \emph{Conjunctive Normal Form} (CNF):
  \begin{displaymath}
    \begin{array}{cl}
      &(\lnot a \lor \lnot b \lor \lnot c) \\
      \land&(\lnot b \lor \lnot c \lor \lnot d) \\
      \land&(\lnot a \lor \lnot b \lor c) \\
      \land&b
    \end{array}
  \end{displaymath}
  \begin{enumerate}
  \item Entire formula is a conjunction $C_1 \land C_2 \land \cdots \land C_n$
  \item where each \emph{clause} $C_i = L_{i,1} \lor L_{i,2} \lor \cdots \lor L_{i,k}$
  \item where each \emph{literal} $L_{i,j} = x_{i,j}$ or $L_{i,j} = \lnot x_{i,j}$
  \end{enumerate}

  \raggedleft
  {\footnotesize \textcolor{black!60}{Every formula can be put into CNF (later)}}

  \raggedright

\end{frame}

\begin{frame}
  {Conjunctive Normal Form}

  The restriction to CNF may seem like a massive restriction.

  \bigskip

  Every Propositional Logic formula can be translated into CNF.

  \bigskip

  \begin{enumerate}
  \item Slow way: ``multiply out the brackets'' \\
    \textcolor{black!60}{Resulting formula might be exponentially larger}
  \item Fast way: ``Tseytin translation'' \\
    \textcolor{black!60}{Resulting formulas is at most 3 times larger}
  \end{enumerate}
  We'll just assume this can be done for now.
\end{frame}

% \begin{frame}
%   {Conjunctive Normal Form}
%   For the package installation problems, we already have CNF:
%   \begin{displaymath}
%     \begin{array}{ll}
%       \left.
%       \begin{array}{cl}
%         &(\lnot \mathit{libD}_1 \lor \lnot \mathit{libD}_2) \\
%         \land&(\lnot \mathit{libC}_1 \lor \lnot \mathit{libC}_2) \\
%         \land&(\lnot \mathit{progA}_1 \lor \lnot \mathit{progA}_2)
%       \end{array}
%       \right\}
%       &
%         \textsc{Incompat}
%       \\
%       \left.
%       \begin{array}{cl}
%         \land&(\lnot \mathit{progA}_1 \lor \mathit{libC}_1) \\
%         \land&(\lnot \mathit{progA}_2 \lor \mathit{libC}_2) \\
%         \land&(\lnot \mathit{libC}_1 \lor \mathit{libD}_2) \\
%         \land&(\lnot \mathit{libC}_2 \lor \mathit{libD}_2)
%       \end{array}
%                \right\} & \textsc{Dep}
%       \\
%       \left.\begin{array}{cl}
%         \land &(\mathit{progA}_1 \lor \mathit{progA}_2)
%       \end{array}\right.
%     \end{array}
%   \end{displaymath}

% \end{frame}

\begin{frame}
  {A SAT Solver's job}

  Given clauses that look like:
  \begin{displaymath}
    \begin{array}{cl}
      &(\lnot a \lor \lnot b \lor \lnot c) \\
      \land&(\lnot b \lor \lnot c \lor \lnot d) \\
      \land&(\lnot a \lor \lnot b \lor c) \\
      \land&b
    \end{array}
  \end{displaymath}

  To find a valuation $v$ for the $a, ...$ such that at least one
  literal in every clause is true.
  \begin{displaymath}
    \textrm{Returns either:} \quad \mathsf{SAT}(v) \qquad \textrm{or} \qquad \mathsf{UNSAT}.
  \end{displaymath}
\end{frame}

\begin{frame}
  {Basic idea of the algorithm}

  \begin{enumerate}
  \item The clauses $C_1, \dots, C_n$ to be satisfied are fixed;
  \item The state is a partial valuation \textcolor{black!60}{(next slide)};
  \item At each step we pick a way to modify the current partial
    valuation by choosing from a collection of rules;
  \item Algorithm terminates when either a satisfying valuation is
    constructed, or it is clear that this is not possible.
  \end{enumerate}

  \bigskip

  This is known as the \emph{DPLL Algorithm}.
\end{frame}

\begin{frame}
  {Partial Valuations}

  To describe what a SAT solver does, we need \emph{partial valuations}.

  \bigskip

  A \rhighlight{partial valuation} $v^?$ is a:
  \begin{itemize}
  \item \emph{sequence} of assignments to atoms; with each one marked
    \begin{enumerate}
    \item $d$ecision point, if we guessed this value.
    \item $f$orced, if we were forced to have this value.
    \end{enumerate}
  \end{itemize}

\medskip

  Examples:
  $\begin{array}[t]{l}
      v_1^? = [ a \decideto \true, b \decideto \false, c \forcedto \true ] \\
      v_2^? = [ a \forcedto \false, b \decideto \false ] \\
    \end{array}$
\end{frame}

\begin{frame}
  {Differences with Valuations}

  \begin{enumerate}
  \item The order matters\\
    \sidenote{we keep track of what decisions we make during the search}
  \item Not all atoms need an assignment\\
    \sidenote{we want to represent partial solutions during the search}
  \item We mark decision points and forced decisions.
  \end{enumerate}
\end{frame}

\begin{frame}
  {Notation}
  We write
  \begin{displaymath}
    v_1^?, a \decideto x, v_2^?
  \end{displaymath}
  for a partial valuation with $a \decideto x$ somewhere in
  the middle.

  \bigskip

  We write
  \begin{displaymath}
    \mathit{decisionfree}(v^?)
  \end{displaymath}
  if none of the assignments in $v^?$ are marked $d$ \\
  \sidenote{i.e., all decisions in $v^?$ are $f$orced}
\end{frame}

\begin{frame}
  {1. Initialisation}

  We start with the \emph{empty partial valuation} $v^? = []$.\\
  \sidenote{We make no commitments}

  \medskip

  We must extend this guess to a valuation that satisfies all the clauses.

\end{frame}

\begin{frame}
  {2. Guessing}

  If there is an atom $a$ in the clauses that is not in the current
  partial valuation $v^?$, then we can make a guess. We pick one of:
  \begin{displaymath}
    v^?, a \decideto \true \qquad \textrm{or} \qquad v^?, a \decideto \false
  \end{displaymath}
  \sidenote{Note: we have marked this as a $d$ecision point}

\end{frame}

\begin{frame}
  {3. Success}

  If the current $v^?$ makes all the clauses true (for all $i$,
  $\sem{C_i}v^? = \true$), then stop with $\mathsf{SAT}(v^?)$.
\end{frame}

\def\doneafter<#1>#2{%
  \temporal<#1>%
  {\colorbox{white}{\color{black}$\displaystyle#2$}}%
  {\colorbox{cyan!50}{\color{black}$\displaystyle#2$}}%
  {\colorbox{cyan!50}{\color{black}$\displaystyle#2$}}}

\def\stuck<#1>#2{%
  \temporal<#1>%
  {\colorbox{white}{\color{black}$\displaystyle#2$}}%
  {\colorbox{magenta!50}{\color{black}$\displaystyle#2$}}%
  {\colorbox{white}{\color{black}$\displaystyle#2$}}}

\def\done<#1>#2{%
  \temporal<#1>%
  {\colorbox{white}{\color{black}$\displaystyle#2$}}%
  {\colorbox{cyan!50}{\color{black}$\displaystyle#2$}}%
  {\colorbox{white}{\color{black}$\displaystyle#2$}}}

\newcommand{\undet}[1]{\stackrel{\textcolor{white}{\checkmark}}{\colorbox{white}{$#1$}}}
\newcommand{\failed}[1]{\stackrel{\times}{\colorbox{magenta!90}{\textcolor{white}{$#1$}}}}
\newcommand{\satd}[1]{\stackrel{\checkmark}{\colorbox{cyan!50}{{$#1$}}}}

\begin{frame}
  {Example}
  \begin{displaymath}
    \only<1>{(\undet{\lnot a} \lor \undet{\lnot b} \lor \undet{\lnot c}) \land (\undet{\lnot b} \lor \undet{\lnot c} \lor \undet{\lnot d}) \land (\undet{\lnot a} \lor \undet{\lnot b} \lor \undet{c}) \land \undet{b}}%
    \only<2>{(\satd{\lnot a} \lor \undet{\lnot b} \lor \undet{\lnot c}) \land (\undet{\lnot b} \lor \undet{\lnot c} \lor \undet{\lnot d}) \land (\satd{\lnot a} \lor \undet{\lnot b} \lor \undet{c}) \land \undet{b}}%
    \only<3>{(\satd{\lnot a} \lor \failed{\lnot b} \lor \undet{\lnot c}) \land (\failed{\lnot b} \lor \undet{\lnot c} \lor \undet{\lnot d}) \land (\satd{\lnot a} \lor \failed{\lnot b} \lor \undet{c}) \land \satd{b}}%
    \only<4>{(\satd{\lnot a} \lor \failed{\lnot b} \lor \satd{\lnot c}) \land (\failed{\lnot b} \lor \satd{\lnot c} \lor \undet{\lnot d}) \land (\satd{\lnot a} \lor \failed{\lnot b} \lor \failed{c}) \land \satd{b}}%
    \only<5->{(\satd{\lnot a} \lor \failed{\lnot b} \lor \satd{\lnot c}) \land (\failed{\lnot b} \lor \satd{\lnot c} \lor \satd{\lnot d}) \land (\satd{\lnot a} \lor \failed{\lnot b} \lor \failed{c}) \land \satd{b}}%
  \end{displaymath}
  \sidenote{Need at least one cyan in every clause}

  \sechead{Sequence of (lucky) guesses}
  \begin{enumerate}
  \item $[]$
  \item<2-> $[a \decideto \false]$
  \item<3-> $[a \decideto \false, b \decideto \true]$
  \item<4-> $[a \decideto \false, b \decideto \true, c \decideto \false]$
  \item<5-> $[a \decideto \false, b \decideto \true, c \decideto \false, d \decideto \false]$, a satisfying valuation.
  \end{enumerate}

  % \pause\pause\pause\pause\pause
  % Line 5 is a satisfying valuation.

  % \pause
  % \bigskip

  % But we can't program ``luck''!
\end{frame}

\begin{frame}
  But we can't program ``luck''!
\end{frame}

\begin{frame}
  {4. Backtracking}

  If we have a partial valuation:
  \begin{displaymath}
    v_1^?, a \decideto x, v_2^?
  \end{displaymath}
  and $\mathit{decisionfree}(v_2^?)$ (so $a \assigned x$ was our most recent guess).

  \medskip

  Then we backtrack (throw away $v_2^?$) and change our mind:
  \begin{displaymath}
    v_1^?, a \forcedto \lnot x
  \end{displaymath}
  marking the assignment as $f$orced.

\end{frame}

\begin{frame}
  {5. Failure}

  If all decisions are forced ($\mathit{decisionfree}(v^?)$), and
  there is at least one clause $C_i$ such that $\sem{C}v^? = \false$,
  then return $\mathsf{UNSAT}$.

\end{frame}

\begin{frame}[t]
  {\footnotesize
    $\only<1>{(\undet{\lnot a} \lor \undet{\lnot b} \lor \undet{\lnot c}) \land (\undet{\lnot b} \lor \undet{\lnot c} \lor \undet{\lnot d}) \land (\undet{\lnot a} \lor \undet{\lnot b} \lor \undet{c}) \land \undet{b}}%
    \only<2>{(\failed{\lnot a} \lor \undet{\lnot b} \lor \undet{\lnot c}) \land (\undet{\lnot b} \lor \undet{\lnot c} \lor \undet{\lnot d}) \land (\failed{\lnot a} \lor \undet{\lnot b} \lor \undet{c}) \land \undet{b}}%
    \only<3>{(\failed{\lnot a} \lor \failed{\lnot b} \lor \undet{\lnot c}) \land (\failed{\lnot b} \lor \undet{\lnot c} \lor \undet{\lnot d}) \land (\failed{\lnot a} \lor \failed{\lnot b} \lor \undet{c}) \land \satd{b}}%
    \only<4>{(\failed{\lnot a} \lor \failed{\lnot b} \lor \failed{\lnot c}) \land (\failed{\lnot b} \lor \failed{\lnot c} \lor \undet{\lnot d}) \land (\failed{\lnot a} \lor \failed{\lnot b} \lor \satd{c}) \land \satd{b}}%
    \only<5>{(\failed{\lnot a} \lor \failed{\lnot b} \lor \satd{\lnot c}) \land (\failed{\lnot b} \lor \satd{\lnot c} \lor \undet{\lnot d}) \land (\failed{\lnot a} \lor \failed{\lnot b} \lor \failed{c}) \land \satd{b}}%
    \only<6>{(\failed{\lnot a} \lor \satd{\lnot b} \lor \undet{\lnot c}) \land (\satd{\lnot b} \lor \undet{\lnot c} \lor \undet{\lnot d}) \land (\failed{\lnot a} \lor \satd{\lnot b} \lor \undet{c}) \land \failed{b}}%
    \only<7>{(\satd{\lnot a} \lor \undet{\lnot b} \lor \undet{\lnot c}) \land (\undet{\lnot b} \lor \undet{\lnot c} \lor \undet{\lnot d}) \land (\satd{\lnot a} \lor \undet{\lnot b} \lor \undet{c}) \land \undet{b}}%
    \only<8>{(\satd{\lnot a} \lor \failed{\lnot b} \lor \undet{\lnot c}) \land (\failed{\lnot b} \lor \undet{\lnot c} \lor \undet{\lnot d}) \land (\satd{\lnot a} \lor \failed{\lnot b} \lor \undet{c}) \land \satd{b}}%
    \only<9>{(\satd{\lnot a} \lor \failed{\lnot b} \lor \failed{\lnot c}) \land (\failed{\lnot b} \lor \failed{\lnot c} \lor \undet{\lnot d}) \land (\satd{\lnot a} \lor \failed{\lnot b} \lor \satd{c}) \land \satd{b}}%
    \only<10>{(\satd{\lnot a} \lor \failed{\lnot b} \lor \failed{\lnot c}) \land (\failed{\lnot b} \lor \failed{\lnot c} \lor \failed{\lnot d}) \land (\satd{\lnot a} \lor \failed{\lnot b} \lor \satd{c}) \land \satd{b}}%
    \only<11>{(\satd{\lnot a} \lor \failed{\lnot b} \lor \failed{\lnot c}) \land (\failed{\lnot b} \lor \failed{\lnot c} \lor \satd{\lnot d}) \land (\satd{\lnot a} \lor \failed{\lnot b} \lor \satd{c}) \land \satd{b}}%
  $}

{\footnotesize
  \begin{enumerate}
  \item $[]$
    \vspace{-0.5em}
  \item<2-> $[a \decideto \true]$
    \vspace{-0.5em}
  \item<3-> $[a \decideto \true, b \decideto \true]$
    \vspace{-0.5em}
  \item<4-> $[a \decideto \true, b \decideto \true, c \decideto \true]$ \qquad \emph{clause 1 failed, backtrack...}
    \vspace{-0.5em}
  \item<5-> $[a \decideto \true, b \decideto \true, c \forcedto \false]$ \qquad \emph{clause 3 failed, backtrack...}
    \vspace{-0.5em}
  \item<6-> $[a \decideto \true, b \forcedto \false]$ \qquad \emph{clause 4 failed, backtrack...}
    \vspace{-0.5em}
  \item<7-> $[a \forcedto \false]$
    \vspace{-0.5em}
  \item<8-> $[a \forcedto \false, b \decideto \true]$
    \vspace{-0.5em}
  \item<9-> $[a \forcedto \false, b \decideto \true, c \decideto \true]$
    \vspace{-0.5em}
  \item<10-> $[a \forcedto \false, b \decideto \true, c \decideto \true, d \decideto \true]$ \qquad \emph{clause 2 failed, backtrack}
    \vspace{-0.5em}
  \item<11-> $[a \forcedto \false, b \decideto \true, c \decideto \true, d \decideto \false]$ \qquad $\mathsf{SAT}$
  \end{enumerate}
  ~}
\end{frame}

\begin{frame}
  {Summary}

  \begin{enumerate}
  \item SAT solvers are tools that find satisfying valuations for formulas in CNF.
  \item Having a SAT solver enables solving of problems modelled using logic.
  \item The core algorithm is a backtracking search.
  \end{enumerate}
\end{frame}

\weeksection{Faster SAT by Unit Propagation}

% \begin{frame}
%   \begin{center}
%     {\Huge \textcolor{black!60}{Part 4 : }Faster SAT with Unit Propagation}
%   \end{center}
% \end{frame}

\begin{frame}
  {Backtracking is Oblivious}

  The example:
  \begin{displaymath}
    ({\lnot a} \lor {\lnot b} \lor {\lnot c}) \land ({\lnot b} \lor {\lnot c} \lor {\lnot d}) \land ({\lnot a} \lor {\lnot b} \lor {c}) \land {b}
  \end{displaymath}

  Backtracking tries the atoms in some order.

  \bigskip

  But we can see immediately that $b$ must be true.

  \bigskip

  Other forced assignments occur during the search.
\end{frame}

\begin{frame}
  {Making the Search less naive}

  If we are in a situation like:
  \begin{displaymath}
    (\failed{\lnot b} \lor \failed{\lnot c} \lor \undet{\lnot d})
  \end{displaymath}
  then if the current valuation is to succeed in any way, it must be
  the case that $d \assigned \false$.\\
  \sidenote{because we need at least one literal in every clause to be true.}

  % \bigskip

  % Similarly, in a situation like:
  % \begin{displaymath}
  %   (\failed{\lnot b} \lor \failed{\lnot c} \lor \undet{d})
  % \end{displaymath}
  % then if the current valuation is to succeed in any way, it must be
  % the case that $d \assigned \true$.\\

  \bigskip

  Using this, we can make the search a little less naive.
\end{frame}

\begin{frame}
  {6. Unit Propagation Step}

  (a) If there is a clause $C \lor a$and $\sem{C}v^? = \false$, then we
  extend $v^?$ to:
  \begin{displaymath}
    v^?, a \forcedto \true
  \end{displaymath}

  \medskip

  (b) If there is a clause $C \lor \lnot a$ and $\sem{C}v^? = \false$,
  then we extend $v^?$ to:
  \begin{displaymath}
    v^?, a \forcedto \false
  \end{displaymath}

  \sidenote{Note: the $a$ needn't necessarily appear at the end of the clause}

\end{frame}

\begin{frame}
  {\footnotesize
  \begin{displaymath}
    \only<1>{(\undet{\lnot a} \lor \undet{\lnot b} \lor \undet{\lnot c}) \land (\undet{\lnot b} \lor \undet{\lnot c} \lor \undet{\lnot d}) \land (\undet{\lnot a} \lor \undet{\lnot b} \lor \undet{c}) \land \undet{b}}%
    \only<2>{(\undet{\lnot a} \lor \failed{\lnot b} \lor \undet{\lnot c}) \land (\failed{\lnot b} \lor \undet{\lnot c} \lor \undet{\lnot d}) \land (\undet{\lnot a} \lor \failed{\lnot b} \lor \undet{c}) \land \satd{b}}%
    \only<3>{(\failed{\lnot a} \lor \failed{\lnot b} \lor \undet{\lnot c}) \land (\failed{\lnot b} \lor \undet{\lnot c} \lor \undet{\lnot d}) \land (\failed{\lnot a} \lor \failed{\lnot b} \lor \undet{c}) \land \satd{b}}%
    \only<4>{(\failed{\lnot a} \lor \failed{\lnot b} \lor \satd{\lnot c}) \land (\failed{\lnot b} \lor \satd{\lnot c} \lor \undet{\lnot d}) \land (\failed{\lnot a} \lor \failed{\lnot b} \lor \failed{c}) \land \satd{b}}%
    \only<5>{(\satd{\lnot a} \lor \failed{\lnot b} \lor \undet{\lnot c}) \land (\failed{\lnot b} \lor \undet{\lnot c} \lor \undet{\lnot d}) \land (\satd{\lnot a} \lor \failed{\lnot b} \lor \undet{c}) \land \satd{b}}%
    \only<6>{(\satd{\lnot a} \lor \failed{\lnot b} \lor \failed{\lnot c}) \land (\failed{\lnot b} \lor \failed{\lnot c} \lor \undet{\lnot d}) \land (\satd{\lnot a} \lor \failed{\lnot b} \lor \satd{c}) \land \satd{b}}%
    \only<7-8>{(\satd{\lnot a} \lor \failed{\lnot b} \lor \failed{\lnot c}) \land (\failed{\lnot b} \lor \failed{\lnot c} \lor \satd{\lnot d}) \land (\satd{\lnot a} \lor \failed{\lnot b} \lor \satd{c}) \land \satd{b}}%
  \end{displaymath}}

  \begin{enumerate}
  \item $[]$ \qquad \emph{do unit propagation...}
  \item<2-> $[b \forcedto \true]$
  \item<3-> $[b \forcedto \true, a \decideto \true]$ \qquad \emph{do unit propagation...}
  \item<4-> $[b \forcedto \true, a \decideto \true, c \forcedto \false]$ \qquad \emph{clause 3 failed, backtrack...}
  \item<5-> $[b \forcedto \true, a \forcedto \false]$
  \item<6-> $[b \forcedto \true, a \forcedto \false, c \decideto \true]$ \qquad \emph{do unit propagation...}
  \item<7-> $[b \forcedto \true, a \forcedto \false, c \decideto \true, d \forcedto \false]$ \qquad $\mathsf{SAT}$
  \end{enumerate}

  \pause\pause\pause\pause\pause\pause\pause

  \bigskip

  One backtrack vs. four without unit propagation.
\end{frame}

\begin{frame}[t]
  {2-SAT}
  If every clause has at most two literals, UP means less backtracking:

  \begin{displaymath}
    \only<1>{
      \begin{array}{clcl}
        &(\undet{\lnot \mathit{libD}_1} \lor \undet{\lnot \mathit{libD}_2})
        &\land& (\undet{\lnot \mathit{libC}_1} \lor \undet{\lnot \mathit{libC}_2}) \\
        \land&(\undet{\lnot \mathit{progA}_1} \lor \undet{\lnot \mathit{progA}_2})
        &\land& (\undet{\lnot \mathit{progA}_1} \lor \undet{\mathit{libC}_1}) \\
        \land&(\undet{\lnot \mathit{progA}_2} \lor \undet{\mathit{libC}_2})
        &\land& (\undet{\lnot \mathit{libC}_1} \lor \undet{\mathit{libD}_2}) \\
        \land&(\undet{\lnot \mathit{libC}_2} \lor \undet{\mathit{libD}_2})
        &\land&(\undet{\mathit{progA}_1} \lor \undet{\mathit{progA}_2})
      \end{array}}
    \only<2>{
      \begin{array}{clcl}
        &(\undet{\lnot \mathit{libD}_1} \lor \undet{\lnot \mathit{libD}_2})
        &\land& (\undet{\lnot \mathit{libC}_1} \lor \undet{\lnot \mathit{libC}_2}) \\
        \land&(\failed{\lnot \mathit{progA}_1} \lor \undet{\lnot \mathit{progA}_2})
        &\land& (\failed{\lnot \mathit{progA}_1} \lor \undet{\mathit{libC}_1}) \\
        \land&(\undet{\lnot \mathit{progA}_2} \lor \undet{\mathit{libC}_2})
        &\land& (\undet{\lnot \mathit{libC}_1} \lor \undet{\mathit{libD}_2}) \\
        \land&(\undet{\lnot \mathit{libC}_2} \lor \undet{\mathit{libD}_2})
        &\land& (\satd{\mathit{progA}_1} \lor \undet{\mathit{progA}_2})
      \end{array}}
    \only<3>{
      \begin{array}{clcl}
        &(\undet{\lnot \mathit{libD}_1} \lor \undet{\lnot \mathit{libD}_2})
        &\land& (\undet{\lnot \mathit{libC}_1} \lor \undet{\lnot \mathit{libC}_2}) \\
        \land&(\failed{\lnot \mathit{progA}_1} \lor \satd{\lnot \mathit{progA}_2})
        &\land& (\failed{\lnot \mathit{progA}_1} \lor \undet{\mathit{libC}_1}) \\
        \land&(\satd{\lnot \mathit{progA}_2} \lor \undet{\mathit{libC}_2})
        &\land& (\undet{\lnot \mathit{libC}_1} \lor \undet{\mathit{libD}_2}) \\
        \land&(\undet{\lnot \mathit{libC}_2} \lor \undet{\mathit{libD}_2})
        &\land& (\satd{\mathit{progA}_1} \lor \failed{\mathit{progA}_2})
      \end{array}}
    \only<4>{
      \begin{array}{clcl}
        &(\undet{\lnot \mathit{libD}_1} \lor \undet{\lnot \mathit{libD}_2})
        &\land& (\failed{\lnot \mathit{libC}_1} \lor \undet{\lnot \mathit{libC}_2}) \\
        \land&(\failed{\lnot \mathit{progA}_1} \lor \satd{\lnot \mathit{progA}_2})
        &\land& (\failed{\lnot \mathit{progA}_1} \lor \satd{\mathit{libC}_1}) \\
        \land&(\satd{\lnot \mathit{progA}_2} \lor \undet{\mathit{libC}_2})
        &\land& (\failed{\lnot \mathit{libC}_1} \lor \undet{\mathit{libD}_2}) \\
        \land&(\undet{\lnot \mathit{libC}_2} \lor \undet{\mathit{libD}_2})
        &\land& (\satd{\mathit{progA}_1} \lor \failed{\mathit{progA}_2})
      \end{array}}
    \only<5>{
      \begin{array}{clcl}
        &(\undet{\lnot \mathit{libD}_1} \lor \undet{\lnot \mathit{libD}_2})
        &\land& (\failed{\lnot \mathit{libC}_1} \lor \satd{\lnot \mathit{libC}_2}) \\
        \land&(\failed{\lnot \mathit{progA}_1} \lor \satd{\lnot \mathit{progA}_2})
        &\land& (\failed{\lnot \mathit{progA}_1} \lor \satd{\mathit{libC}_1}) \\
        \land&(\satd{\lnot \mathit{progA}_2} \lor \failed{\mathit{libC}_2})
        &\land& (\failed{\lnot \mathit{libC}_1} \lor \undet{\mathit{libD}_2}) \\
        \land&(\satd{\lnot \mathit{libC}_2} \lor \undet{\mathit{libD}_2})
        &\land& (\satd{\mathit{progA}_1} \lor \failed{\mathit{progA}_2})
      \end{array}}
    \only<6>{
      \begin{array}{clcl}
        &(\undet{\lnot \mathit{libD}_1} \lor \failed{\lnot \mathit{libD}_2})
        &\land& (\failed{\lnot \mathit{libC}_1} \lor \satd{\lnot \mathit{libC}_2}) \\
        \land&(\failed{\lnot \mathit{progA}_1} \lor \satd{\lnot \mathit{progA}_2})
        &\land& (\failed{\lnot \mathit{progA}_1} \lor \satd{\mathit{libC}_1}) \\
        \land&(\satd{\lnot \mathit{progA}_2} \lor \failed{\mathit{libC}_2})
        &\land& (\failed{\lnot \mathit{libC}_1} \lor \satd{\mathit{libD}_2}) \\
        \land&(\satd{\lnot \mathit{libC}_2} \lor \satd{\mathit{libD}_2})
        &\land& (\satd{\mathit{progA}_1} \lor \failed{\mathit{progA}_2})
      \end{array}}
    \only<7>{
      \begin{array}{clcl}
        &(\satd{\lnot \mathit{libD}_1} \lor \failed{\lnot \mathit{libD}_2})
        &\land& (\failed{\lnot \mathit{libC}_1} \lor \satd{\lnot \mathit{libC}_2}) \\
        \land&(\failed{\lnot \mathit{progA}_1} \lor \satd{\lnot \mathit{progA}_2})
        &\land& (\failed{\lnot \mathit{progA}_1} \lor \satd{\mathit{libC}_1}) \\
        \land&(\satd{\lnot \mathit{progA}_2} \lor \failed{\mathit{libC}_2})
        &\land& (\failed{\lnot \mathit{libC}_1} \lor \satd{\mathit{libD}_2}) \\
        \land&(\satd{\lnot \mathit{libC}_2} \lor \satd{\mathit{libD}_2})
        &\land& (\satd{\mathit{progA}_1} \lor \failed{\mathit{progA}_2})
      \end{array}}
  \end{displaymath}

  \begin{displaymath}
    \only<1>{[]}
    \only<2>{[\mathit{progA}_1 \decideto \true]}
    \only<3>{[\mathit{progA}_1 \decideto \true, \mathit{progA}_2 \forcedto \false]}
    \only<4>{[\mathit{progA}_1 \decideto \true, \mathit{progA}_2 \forcedto \false, \mathit{libC}_1 \forcedto \true]}
    \only<5>{[\mathit{progA}_1 \decideto \true, \mathit{progA}_2 \forcedto \false, \mathit{libC}_1 \forcedto \true, \mathit{libC}_2 \forcedto \false]}
    \only<6>{[\mathit{progA}_1 \decideto \true, \mathit{progA}_2 \forcedto \false, \mathit{libC}_1 \forcedto \true, \mathit{libC}_2 \forcedto \false, \mathit{libD}_2 \forcedto \true]}
    \only<7>{[\mathit{progA}_1 \decideto \true, \mathit{progA}_2 \forcedto \false, \mathit{libC}_1 \forcedto \true, \mathit{libC}_2 \forcedto \false, \mathit{libD}_2 \forcedto \true, \mathit{libD}_1 \forcedto \false]}
  \end{displaymath}
\end{frame}

\begin{frame}
  {2-SAT}

  If every clause has at most two literals,

  \begin{itemize}
  \item UP means at most one backtrack
  \item Means that we can solve the problem in polynomial time
  \item So for the $n$-SAT problem:
    \begin{itemize}
    \item If $n \leq 2$, there is a fast polynomial time algorithm
    \item If $n \geq 3$, no known general fast algorithm
    \end{itemize}
  \end{itemize}
\end{frame}

\begin{frame}
  {Summary of the Rules 1}

  \begin{displaymath}
    \begin{array}{@{}llcl@{\qquad}l}
      \TirName{DecideTrue}    & v^? & \Longrightarrow & v^?, a \decideto \true & \textit{if $a$ is not assigned in $v^?$} \\
      \\
      \TirName{DecideFalse}   & v^? & \Longrightarrow & v^?, a \decideto \false & \textit{if $a$ is not assigned in $v^?$} \\
      \\
      \TirName{Success}       & v^? & \Longrightarrow & \mathsf{SAT}(v^?) & \textit{if $v^?$ makes all the} \\
      &&&&\textit{clauses true.} \\
  \end{array}
\end{displaymath}
\end{frame}

\begin{frame}
  {Summary of the Rules 2}

  \begin{displaymath}
    \begin{array}{@{}llcl@{\qquad}l}
      \TirName{BackTrack}     & v_1^?, a \decideto x, v_2^? & \Longrightarrow & v_1^?, a \forcedto \lnot x & \textit{if $v_2^?$ is decision free} \\
      \\
      \TirName{Fail}          & v^? & \Longrightarrow & \mathsf{UNSAT} & \textit{if $v^?$ is decision free, and}\\
                              &&&&\textit{makes at least one clause}\\
      &&&&\textit{false.} \\
  \end{array}
\end{displaymath}
\end{frame}

\begin{frame}
  {Summary of the Rules 3}

  \begin{displaymath}
    \begin{array}{@{}llcl@{\qquad}l}
      \TirName{UnitPropTrue}  & v^? & \Longrightarrow & v^?, a \forcedto \true & \textit{if there is a clause $C \lor a$}\\
      &&&&\textit{and $\sem{C}(v^?) = \false$} \\
      \\
      \TirName{UnitPropFalse} & v^? & \Longrightarrow & v^?, a \forcedto \false & \textit{if there is a clause $C \lor \lnot a$}\\
      &&&&\textit{and $\sem{C}(v^?) = \false$} \\
  \end{array}
\end{displaymath}
\end{frame}

\begin{frame}[t]
  {Real SAT solvers}
  Use very efficient data structures.
  \sidenote{Key is very fast unit propagation}

  \medskip

  Use heuristics to guide the search:
  \begin{itemize}
  \item Which atom to try next? (not just $a, b, c, ...$)
  \item Whether to try $\true$ or $\false$ first?
  \end{itemize}

  \bigskip

  Incorporate additional rules:
  \begin{itemize}
  \item Non-chronological backjumping \\
    \sidenote{skip several decision points by analysing conflicts}
  \item Clause learning to avoid doing the same work over again.
  \item “CDCL” (Conflict Driven Clause Learning)
  \item Random walk between possible valuations “WalkSAT”.
  \end{itemize}
\end{frame}

% \begin{frame}[t]
%   {Further Reading}
%   A blog post with a Python implementation: \\
%   \quad \emph{Understanding SAT by Implementing a Simple SAT Solver in Python} \\
%   \quad \textcolor{black!60}{Sahand Saba} \\
%   \quad {\tiny \url{https://sahandsaba.com/understanding-sat-by-implementing-a-simple-sat-solver-in-python.html}}

%   \bigskip

%   Another blog post with more formalism: \\
%   \quad \emph{A Primer on Boolean Satisfiability} \\
%   \quad \textcolor{black!60}{Emina Torlak} \\
%   \quad {\tiny \url{https://homes.cs.washington.edu/~emina/blog/2017-06-23-a-primer-on-sat.html}} \\
%   \qquad {\footnotesize See also the links at the end for lots more detail.}
% \end{frame}

% \begin{frame}[t]
%   {More Further Reading}

%   For more breadth and detail than you could possibly imagine:\\
%   \quad \emph{The Art of Computer Programming: 7.2.2.2 Satisfiability} \\
%   \qquad \emph{Draft: Volume 4B, Pre-fascicle 6A} \\
%   \quad \textcolor{black!60}{Donald E. Knuth} \\
%   \quad {\tiny \url{https://cs.stanford.edu/~knuth/fasc6a.ps.gz}}
% \end{frame}

\begin{frame}
  {Summary}

  \begin{itemize}
  \item Unit Propagation speeds up SAT Solving \\
    \sidenote{by using the structure of the problem}
  \item This makes 2-SAT very fast
  \item Real SAT Solvers are very sophisticated.
  \end{itemize}
\end{frame}

\weeksection{Conversion to CNF}

\begin{frame}[t]
  {Conjunctive Normal Form (CNF)}

  \begin{displaymath}
    \begin{array}{cl}
      &(\lnot a \lor \lnot b \lor \lnot c) \\
      \land&(\lnot b \lor \lnot c \lor \lnot d) \\
      \land&(\lnot a \lor \lnot b \lor c) \\
      \land&b
    \end{array}
  \end{displaymath}
  \begin{enumerate}
  \item Entire formula is a conjunction $C_1 \land C_2 \land \cdots \land C_n$
  \item where each \emph{clause} $C_i = L_{i,1} \lor L_{i,2} \lor \cdots \lor L_{i,k}$
  \item where each \emph{literal} $L_{i,j} = x_{i,j}$ or $L_{i,j} = \lnot x_{i,j}$
  \end{enumerate}
\end{frame}

\begin{frame}[t]
  {Disjunctive Normal Form (DNF)}

  \emph{Disjunctive Normal Form} (DNF) is similar, but swaps $\land$
  and $\lor$.

  \begin{displaymath}
    \begin{array}{cl}
      &(\lnot a \land \lnot b \land \lnot c) \\
      \lor&(\lnot b \land \lnot c \land \lnot d) \\
      \lor&(\lnot a \land \lnot b \land c) \\
      \lor&b
    \end{array}
  \end{displaymath}
  \begin{enumerate}
  \item Entire formula is a \emph{dis}junction $D_1 \lor D_2 \lor \cdots \lor D_n$
  \item where each \emph{disjunct} $D_i = L_{i,1} \land L_{i,2} \land \cdots \land L_{i,k}$
  \item where each \emph{literal} $L_{i,j} = x_{i,j}$ or $L_{i,j} = \lnot x_{i,j}$
  \end{enumerate}
\end{frame}

\begin{frame}
  {Normal Forms and Satisfiability}

  \textbf{CNF} \\ Each clause is a \emph{constraint} and all constraints
  must be satisfied.

  \bigskip

  \textbf{DNF} \\ At least one of the disjuncts must be satisfied.

  \bigskip

  \emph{Exercise:} How would you write a SAT
  Solver for formulas in DNF? Why don't we do this instead of CNF?
\end{frame}

\begin{frame}
  {Conversion to CNF}

  Not every formula is in CNF, e.g.,
  \begin{displaymath}
    (A \land B) \to (B \land A)
  \end{displaymath}
  What if we want to use a SAT solver to determine satisfiability?

  \bigskip

  Two ways to convert a formula to CNF that is ``the same'':
  \begin{itemize}
  \item ``Multiplying out''
  \item Tseytin transformation
  \end{itemize}

  \bigskip

  First we need to define what we mean by ``the same''.

\end{frame}

\begin{frame}
  {Equivalent Formulas}
  Define two formulas $P$ and $Q$ to be \emph{equivalent}, written
  \begin{displaymath}
    P \equiv Q
  \end{displaymath}
  exactly when, for all valuations $v$,
  \begin{displaymath}
    \sem{P}v = \sem{Q}v
  \end{displaymath}
  \textcolor{black!60}{Equivalent to both $P \models Q$ and $Q \models P$ being valid}
\end{frame}

\begin{frame}
  {Simplifying Implication}

  \begin{displaymath}
    A \to B \equiv \lnot A \lor B
  \end{displaymath}
  \medskip
  \begin{displaymath}
    \begin{array}{cc|c|c|c}
      \multicolumn{2}{c|}{\textit{valuation}}&&P&Q \\
      A & B & \lnot A & A \to B & \lnot A \lor B \\
      \hline
      \false & \false & \true  & \true  & \true \\
      \false & \true  & \true  & \true  & \true \\
      \true  & \false & \false & \false & \false \\
      \true  & \true  & \false & \true  & \true
    \end{array}
  \end{displaymath}
\end{frame}

\begin{frame}[t]
  {Double Negation}

  Negating twice is the same as doing nothing:
  \begin{displaymath}
    A \equiv \lnot\lnot A
  \end{displaymath}
  \begin{displaymath}
    \begin{array}{c|c|c|c}
      \textit{valuation} & & P & Q \\
      A&\lnot A&A&\lnot \lnot A \\
      \hline
      \false&\true&\false&\false \\
      \true&\false&\true&\true
    \end{array}
  \end{displaymath}
\end{frame}

\begin{frame}
  {de Morgan's laws}

  Negation swaps $\land$ and $\lor$:
  \begin{displaymath}
    \lnot (A \land B) \equiv \lnot A \lor \lnot B
  \end{displaymath}
  \begin{displaymath}
    \begin{array}{cc|ccc|c|c}
      \multicolumn{2}{c|}{\textit{valuation}}&&&&P&Q\\
      A & B & \lnot A & \lnot B & A \land B & \lnot (A \land B) & \lnot A \lor \lnot B \\
      \hline
      \false & \false & \true  & \true  & \false & \true & \true \\
      \false & \true  & \true  & \false & \false & \true & \true \\
      \true  & \false & \false & \true  & \false & \true & \true \\
      \true  & \true  & \false & \false & \true  & \false & \false
    \end{array}
  \end{displaymath}
  Similar for $\lnot (A \lor B) \equiv \lnot A \land \lnot B$
\end{frame}

\begin{frame}
  {Negation Normal Form (NNF)}

  Using the equivalences:
  \begin{displaymath}
    \begin{array}{rcl}
      A \to B & \equiv & \lnot A \lor B \\
      A&\equiv&\lnot \lnot A \\
      \lnot (A \land B)&\equiv&\lnot A \lor \lnot B \\
      \lnot (A \lor B)&\equiv&\lnot A \land \lnot B
    \end{array}
  \end{displaymath}

  We can \emph{rewrite} any formula into an equivalent one with
  \begin{enumerate}
  \item No implications ($\to$s)
  \item All negation signs on the atomic propositions
  \end{enumerate}

  % \bigskip

  % Formula will have only $\land$s and $\lor$s, and $\lnot$s only at
  % the ``leaves''.
\end{frame}

\begin{frame}
  {Example}

  \begin{displaymath}
    \begin{array}{cll}
             & (a \land (a \to b)) \to c \\
      \equiv & \lnot (a \land (a \to b)) \lor c & \textit{converted }\to\\
      \equiv & \lnot (a \land (\lnot a \lor b)) \lor c & \textit{converted }\to\\
      \equiv & \lnot a \lor \lnot (\lnot a \lor b) \lor c & \textit{converted }\land\textit{ to }\lor \\
      \equiv & \lnot a \lor (\lnot \lnot a \land \lnot b) \lor c & \textit{converted }\lor\textit{ to }\land \\
      \equiv & \lnot a \lor (a \land \lnot b) \lor c & \textit{converted double negation} \\
    \end{array}
  \end{displaymath}

  \bigskip

  Now in NNF, but not CNF.
\end{frame}

\begin{frame}
  {``Push'' $\lor$s into $\land$s}
  \begin{displaymath}
    A \lor (B \land C) \equiv (A \lor B) \land (A \lor C)
  \end{displaymath}
  {\footnotesize\begin{displaymath}
    \begin{array}{ccc|ccc|c|c}
      \multicolumn{3}{c|}{\textit{valuation}}&&&&P&Q\\
      A & B & C       & B \land C & A \lor B & A \lor C & A \lor (B \land C) & (A \lor B) \land (A \lor C) \\
      \hline
      \false & \false & \false & \false       & \false      & \false      & \false                & \false \\
      \false & \false & \true  & \false       & \false      & \true       & \false                & \false \\
      \false & \true  & \false & \false       & \true       & \false      & \false                & \false \\
      \false & \true  & \true  & \true        & \true       & \true       & \true                 & \true  \\
      \true  & \false & \false & \false       & \true       & \true       & \true                 & \true  \\
      \true  & \false & \true  & \false       & \true       & \true       & \true                 & \true  \\
      \true  & \true  & \false & \false       & \true       & \true       & \true                 & \true  \\
      \true  & \true  & \true  & \true        & \true       & \true       & \true                 & \true  \\
    \end{array}
  \end{displaymath}}
\end{frame}

\begin{frame}
  {Conversion to CNF}

  \begin{displaymath}
    \begin{array}{cl}
             & \lnot a \lor (a \land \lnot b) \lor c \\
      \equiv & \hspace{4cm}\textit{multiply out} \\
             & \lnot a \lor ((a \lor c) \land (\lnot b \lor c)) \\
      \equiv & \hspace{4cm}\textit{multiply out} \\
             & (\lnot a \lor a \lor c) \land (\lnot a \lor \lnot b \lor c) \\
    \end{array}
  \end{displaymath}
  Now in CNF.

  \bigskip

  \textcolor{black!60}{(Can further simplify to: $(\lnot a \lor \lnot b \lor c)$)}
\end{frame}

\begin{frame}
  {Exponential Blowup}

  If we convert
  $(a \land b \land c) \lor (d \land e \land f) \lor (g \land h \land
  i)$ to CNF, we get:

  {\footnotesize
    \begin{displaymath}
      \begin{array}{@{}l}
        (a \lor d \lor g) \land (a \lor d \lor h) \land (a \lor d \lor i) \land (a \lor e \lor g) \land (a \lor e \lor h) \land \\
        (a \lor e \lor i) \land (a \lor f \lor g) \land (a \lor f \lor h) \land (a \lor f \lor i) \land (b \lor d \lor g) \land \\
        (b \lor d \lor h) \land (b \lor d \lor i) \land (b \lor e \lor g) \land (b \lor e \lor h) \land (b \lor e \lor i) \land \\
        (b \lor f \lor g) \land (b \lor f \lor h) \land (b \lor f \lor i) \land (c \lor d \lor g) \land (c \lor d \lor h) \land \\
        (c \lor d \lor i) \land (c \lor e \lor g) \land (c \lor e \lor h) \land (c \lor e \lor i) \land (c \lor f \lor g) \land \\
        (c \lor f \lor h) \land (c \lor f \lor i)
      \end{array}
    \end{displaymath}}

  which has $27$ clauses.

\end{frame}

\begin{frame}
  {Summary}

  \begin{itemize}
  \item SAT Solvers take their input in CNF
  \item Some problems are naturally in CNF
  \item Conversion by ``multiplying out'' can generate huge formulas
  \item We need something better
  \end{itemize}
\end{frame}

\weeksection{Tseytin Transformation}

\begin{frame}
  {Tseytin Transformation}

  The Tseytin transformation converts a formula into CNF with at most
  3 times as many clauses as connectives in the original formula
  (versus potentially exponential for multiplying out the brackets).

  \bigskip

  \begin{enumerate}
  \item Convert the formula into equations \\
    \quad \textcolor{black!60}{One connective $\leadsto$ one equation}
  \item Convert each equation into clauses \\
    \quad \textcolor{black!60}{One equation $\leadsto$ 2-3 clauses}
  \end{enumerate}

  \bigskip

  Result is not equivalent, but \emph{equisatisfiable}.


  % FIXME: give the basic idea

  % \textcolor{black!60}{Assume that all $\to$s have been removed}

  % If we can find values for $a, b, c$ \emph{and} $x_1, x_2, x_3$ such
  % that all these equations are true, and $x_1$ is true, then we've
  % found values for $a, b, c$ such that the original formula is
  % satisfied.

\end{frame}

\begin{frame}[t]
  {1. Name subformulas}

  Take the formula and name all the non-atomic subformulas.

  \medskip

  Example:
  \begin{displaymath}
    \lnot(a \land (\lnot a \lor b)) \lor c
  \end{displaymath}
  becomes:
  \begin{displaymath}
    \begin{array}{lcl}
      x_1&=&x_2 \lor c \\
      x_2&=&\lnot x_3 \\
      x_3&=&a \land x_4 \\
      x_4&=&x_5 \lor b\\
      x_5&=&\lnot a
    \end{array}
  \end{displaymath}
\end{frame}

\begin{frame}
  {2. Converting Equations to Clauses}

  Given an equation like $x = y \land z$, we want some clauses that
  are true for every valuation that satisfies the equation.

  \pause
  \bigskip

  Derive by conversion to CNF:
  \begin{displaymath}
    \begin{array}{cl}
      &x = y \land z \\
      \equiv&(x \to (y \land z)) \land ((y \land z) \to x) \\
      \equiv&(\lnot x \lor (y \land z)) \land (\lnot(y \land z) \lor x)\\
      \equiv&(\lnot x \lor y) \land (\lnot x \lor z) \land (\lnot y \lor \lnot z \lor x)
    \end{array}
  \end{displaymath}

\end{frame}

\begin{frame}
  {2. Equations to Clauses}

  Take each equation $x = y \mathop\Box z$ and turn it into clauses:
  \begin{enumerate}
  \item If $x = y \land z$, add
    \begin{displaymath}
      (\lnot x \lor y) \land (\lnot x \lor z) \land (\lnot y \lor \lnot z \lor x)
    \end{displaymath}
  \item If $x = y \lor z$, add
    \begin{displaymath}
      (y \lor z \lor \lnot x) \land (\lnot y \lor x) \land (\lnot z \lor x)
    \end{displaymath}
  \item If $x = \lnot y$, add
    \begin{displaymath}
      (\lnot y \lor \lnot x) \land (y \lor x)
    \end{displaymath}
  \end{enumerate}

\end{frame}

\begin{frame}
  {3. Assert the top level variable}

  If $x$ is the name of the whole formula, add a clause with just
  $x$:

  \begin{displaymath}
    \begin{array}{cl}
      &\textit{equation 1} \\
      \land&\textit{equation 2}\\
      \land&...\\
      \land&x
    \end{array}
  \end{displaymath}

  This asserts that our original formula must be true.
\end{frame}

\begin{frame}
  {Example: $\lnot (A \land B) \lor (B \land A)$}

  \rhighlight{1.}~Name the subformulas:
  \begin{displaymath}
    \begin{array}{lcl@{\hspace{2em}}lcl}
      x_1&=&x_2 \lor x_4 &
      x_2&=&\lnot x_3 \\
      x_3&=&A \land B &
      x_4&=&B \land A
    \end{array}
  \end{displaymath}

  \pause

  \rhighlight{2+3.}~Generate clauses: \textcolor{black!60}{(One line per equation)}
  \begin{displaymath}
    \begin{array}{cl}
      &(x_2 \lor x_4 \lor \lnot x_1) \land (\lnot x_2 \lor x_1) \land (\lnot x_4 \lor x_1) \\
      \land&(\lnot x_3 \lor \lnot x_2) \land (x_3 \lor x_2) \\
      \land&(\lnot A \lor \lnot B \lor x_3) \land (A \lor \lnot x_3) \land (B \lor \lnot x_3) \\
      \land&(\lnot B \lor \lnot A \lor x_4) \land (B \lor \lnot x_4) \land (A \lor \lnot x_4) \\
      \land&x_1
    \end{array}
  \end{displaymath}
\end{frame}

\begin{frame}
  {Efficiency}

  In small examples, we get many clauses.

  \bigskip

  But we \emph{always} get $\leq 3n$ clauses, where $n$ number of
  connectives.

  \bigskip

  Multiplying out can result in exponential number of clauses.

  \bigskip

  Can also optimise (see the tutorial questions).

\end{frame}

% \begin{frame}
%   A bigger example... FIXME
% \end{frame}

\begin{frame}
  {Not Equivalent!}

  The formulas generated by the Tseytin transformation are
  \textbf{not} equivalent to the original, because they have extra
  atomic propositions.
\end{frame}

\begin{frame}
  {Example}

  If the original formula is
  \begin{displaymath}
    \lnot A
  \end{displaymath}

  the Tseytin transformed version is: \textcolor{black!60}{(assuming
    we don't optimise)}
  \begin{displaymath}
    (\lnot A \lor \lnot x) \land (A \lor x) \land x
  \end{displaymath}

  Then $\{A \assigned \false, x \assigned \false\}$ satisfies the
  original, but not the transformed formula.
\end{frame}

\begin{frame}
  {Equisatisfiable}

  If we write $\mathsf{Tseytin}(P)$ for the Tseytin translation of
  $P$, then:
  \begin{enumerate}
  \item If there exists a valuation $v_1$ such that $\sem{P}v_1 = \true$,
    then there exists a valuation $v_2$ such that
    $\sem{\mathsf{Tseytin}(P)}v_2 = \true$;
  \item If there exists a valuation $v$ such that
    $\sem{\mathsf{Tseytin}(P)}v = \true$, then the valuation $v' = v$
    without the additional $x_i$s makes $\sem{P}v' = \true$.
  \end{enumerate}

  \bigskip

  This is called ``equisatisfiability''.
\end{frame}

\begin{frame}
  {Example}

  $v = \{A \assigned \false\}$ satisfies $\lnot A$

  \bigskip

  The corresponding satisfying valuation for
  \begin{displaymath}
    (\lnot A \lor \lnot x) \land (A \lor x) \land x
  \end{displaymath}
  is $\{A \assigned \false, x \assigned \true\}$.

  \bigskip

  A corresponding satisfying assignment always exists for the Tseytin
  transformation, because it is built from equations.
\end{frame}

\begin{frame}
  {Summary}

  \begin{itemize}
  \item Tseytin transformation converts formulas to CNF
  \item Generates $\leq 3n$ clauses, where $n$ is the number of connectives
  \item Avoids exponential blowup
  \item Can be further optimised
  \item Result is \emph{equisatisfiable}
  \end{itemize}
\end{frame}



% \weeksection{Proof Search}

% \begin{frame}
%   Judgements:

%   \begin{displaymath}
%     x_1, \cdots, x_n \vdash G \Rightarrow \Sigma
%   \end{displaymath}
%   where:
%   \begin{itemize}
%   \item $x_1, \cdots, x_n$ is the list of variable in scope
%   \item $G$ is a list of \emph{goals}
%   \item $\Sigma$ is a list of literals
%   \end{itemize}
% \end{frame}

% \begin{frame}
%   \begin{mathpar}
%     \inferrule*
%     {+P, G \\ +Q, G}
%     {+(P \land Q), G}

%     \inferrule*
%     {-P, -Q, G}
%     {-(P \land Q), G}

%     \inferrule*
%     {+P, +Q, G}
%     {+(P \lor Q), G}

%     \inferrule*
%     {-P, G \\ -Q, G}
%     {-(P \lor Q), G}
%   \end{mathpar}
% \end{frame}

% \begin{frame}
%   Negation:
%   \begin{mathpar}
%     \inferrule*
%     {-P, G}
%     {+(\lnot P), G}

%     \inferrule*
%     {+P, G}
%     {-(\lnot P), G}
%   \end{mathpar}

%   Implication:
%   \begin{mathpar}
%     \inferrule*
%     {-P, +Q, G}
%     {+(P \to Q), G}

%     \inferrule*
%     {+P, G \\ -Q, G}
%     {-(P \to Q), G}
%   \end{mathpar}
% \end{frame}

% \begin{frame}
%   Universal quantifier:
%   \begin{mathpar}
%     \inferrule*
%     {x_0 \vdash +(P[x := x_0]), G}
%     {+(\forall x. P), G}

%     \inferrule*
%     {G}
%     {-(\forall x. P), G}
%   \end{mathpar}

%   Existential quantifier:
%   \begin{mathpar}
%     \inferrule*
%     {G}
%     {+(\exists x. P)}

%     \inferrule*
%     {x_0 \vdash -(P[x := x_0]), G}
%     {-(\exists x. P), G}
%   \end{mathpar}
% \end{frame}

% \begin{frame}
%   \begin{mathpar}
%     \inferrule*
%     {G \Rightarrow +- A, \Sigma}
%     {+- A, G \Rightarrow \Sigma}
%   \end{mathpar}

%   Axiom:
%   \begin{mathpar}
%     \inferrule*
%     {+A \in \Sigma \\ -A \in \Sigma}
%     {G \Rightarrow \Sigma}
%   \end{mathpar}
% \end{frame}



% \weeksection{Decidable Theories}

% \begin{frame}
%   {Decidable Theories}

%   TBD
% \end{frame}

% \weeksection{Quantifier Elimination}

% \begin{frame}
%   {Fourier-Motzkin Elimination}

%   TBD
% \end{frame}

\end{document}
