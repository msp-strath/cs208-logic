% -*- TeX-engine: xetex -*-

\documentclass[xetex,aspectratio=169,14pt,hyperref={pdfpagelabels=true,pdflang={en-GB}}]{beamer}

\usepackage[sci,noslidestrathidentity]{strathclyde}
\strathsetidentity{Department of}{Computer \& Information Sciences}

\usepackage{lmodern}
\usepackage{subscript}
\usepackage{url}
\usepackage{pifont}
\usepackage{csquotes}
\usepackage{mathpartir}
\usepackage{stmaryrd}
\usepackage{multirow}
\usepackage{euler}
\usepackage[normalem]{ulem}

% 2013-09-08 SU adding xetex specifics
% http://www.woggie.net/2008/07/16/beamer-pdftex-and-xetex/
\usepackage{xltxtra}

% 2013-09-08 SU http://robjhyndman.com/hyndsight/xelatex/
\defaultfontfeatures{Ligatures=TeX}

% 2015-12-09 SU table beautified
\renewcommand{\arraystretch}{1.2}
\usepackage{booktabs}

%MGK compatibility
\newcommand{\hh}[1]
  {\medskip\textbf{\large #1}}
\newcommand{\pdu}[3]
  {#1\rightarrow #2 : &\ & \makebox[70mm][l]{$#3$}}



\setlength{\marginparwidth}{2cm}
\usepackage{todonotes}%[disable]
\let\OldTodo\todo
\renewcommand{\todo}{\OldTodo[inline]}%
\newcommand{\todolater}[1]{}% Things to do for next year


\DeclareTextCommandDefault{\nobreakspace}{\leavevmode\nobreak\ }
\usepackage{rotating}

\usepackage{pdfcomment}% To add alt text for images using \pdftooltip{}

\usepackage{appendixnumberbeamer}

\newcommand{\messageframe}[1]{\begin{frame}\begin{center}\Huge #1\end{center}\end{frame}}
\newcommand{\sechead}[1]{{\bf #1} \\}
\newcommand{\examplehead}[1]{{\bf Example:} {\it\textcolor{red!90}{#1}} \\}
\newcommand{\eqnote}[1]{\hspace{3cm}\textit{#1}}
\newcommand{\sidenote}[1]{\qquad {\footnotesize \textcolor{black!60}{(#1)}}}
\newcommand{\sem}[1]{\llbracket #1 \rrbracket}
\newcommand{\true}{\mathsf{T}}
\newcommand{\false}{\mathsf{F}}

\def\strikeafter<#1>#2{\temporal<#1>{#2}{\sout{#2}}{\sout{#2}}}


%\newcommand{\rhighlight}{\textcolor{titlered}}
\newcommand{\rhighlight}{\textbf}
\newcommand{\highlight}{\textbf}

% \setmainfont{Linux Biolinum O}
\setmainfont{LinBiolinum}[
Path=,
UprightFont = *_R.otf ,
BoldFont = *_RB.otf ,
ItalicFont = *_RI.otf
]

\setbeamertemplate{navigation symbols}{}
%\usecolortheme[rgb={0.8,0,0}]{structure}
\usefonttheme{serif}
\usefonttheme{structurebold}
\setbeamercolor{description item}{fg=black}


\author[Atkey]{Dr.~Robert Atkey}
\institute[Strathclyde]{Computer \& Information Sciences}
\date[]{}

\newcommand{\weeksection}[1]{%
  \section{\thetitle{}, Part~\thesection : #1}
  \begin{frame}
    \begin{center}
      \textcolor{black!60}{\thetitle{}, Part \thesection}\\
      {\Huge #1}
    \end{center}
  \end{frame}}

\newcommand{\weektitle}[2]{\def\thetitle{#2}
\title[CS208 - Topic #1]{CS208 (Semester 1) Topic #1 : #2}}

\newcommand{\assigned}{:}

\newcommand{\forcedto}{\assigned_f}
\newcommand{\decideto}{\assigned_d}


\title[CS208 - Introduction]{CS208 (Semester 1) : Introduction}

\begin{document}

\maketitle

\section{Introduction}

\begin{frame}
  {Why study Logic?}

  Logic is \emph{extremely useful} in Computer Science.

  \bigskip

  For example...
  \begin{enumerate}
  \item Digital circuits
  \item Specifying systems' behaviour
  \item Proving that systems satisfy their specifications
  \item Automated Reasoning, rule-based AI
  \item Designing Programming Languages
  \item Database Query Languages
  \end{enumerate}
\end{frame}

\begin{frame}
  {Logic in Your Degree}

  Behind you (unless you are a 2nd year direct entrant):
  \begin{enumerate}
  \item CS103 \emph{Machines, Languages, and Computation}
  \item CS106 \emph{Computer Systems and Organisation}
  \end{enumerate}
\end{frame}

\begin{frame}
  {Logic in Your Degree}

  Beside you
  \begin{enumerate}
  \item CS207 \emph{Advanced Programming}
  \item CS209 \emph{User and Data Modelling}
  \item CS260 \emph{Functional Thinking}
  \end{enumerate}
\end{frame}

\begin{frame}
  {Logic in Your Degree}

  Ahead of you in 3rd year:
  \begin{enumerate}
  \item CS310 \emph{Foundations of AI}
  \item CS316 \emph{Functional Programming}
  \end{enumerate}

  \medskip

  And 4th year:
  \begin{enumerate}
  \item CS410 \emph{Advanced Functional Programming}
  \item CS411 \emph{Theory of Computation}
  \end{enumerate}
\end{frame}

\begin{frame}
  {Why Study Logic?}

  \begin{itemize}
  \item A Language for Precise Statements \\
    ... for data modelling \\
    ... for describing expected behaviours
  \item Thinking about \emph{invariants} \\
    ... what is \emph{always} true? \\
    ... when is the system \emph{going wrong?}
  \item Thinking about deduction \\
    ... systematic derivation of conclusions from assumptions
  \end{itemize}
\end{frame}

% \begin{frame}
%   {In this course}

%   We will cover two major sorts of Logic:

%   \begin{enumerate}
%   \item Propositional Logic
%   \item Predicate Logic
%   \end{enumerate}
% \end{frame}

% \begin{frame}
%   {Propositional Logic}

%   Propositional Logic talks about atomic statements:
%   \begin{mathpar}
%     \textit{``it is raining''}

%     \textit{``I am in Glasgow''}
%   \end{mathpar}
%   and compound statements made from those:
%   \begin{mathpar}
%     \textit{``it is raining''} \land \textit{``I am in Glasgow''}

%     \textit{``it is raining''} \to \textit{``I am in Glasgow''}
%   \end{mathpar}

% \end{frame}

% \begin{frame}
%   {Predicate Logic}

%   Predicate Logic adds \emph{quantifiers} and \emph{predicates}:
%   \begin{mathpar}
%     \forall \mathit{day}.~\textrm{InGlasgow}(\mathit{day}) \to (\textrm{Raining}(\mathit{day}) \lor \textrm{Drizzle}(\mathit{day}))

%     \forall x. \forall y.~x + y = y + x
%   \end{mathpar}
% \end{frame}

% \begin{frame}
%   {Semantics}

%   The \emph{Semantics} of a logic tell you what formulas \emph{mean}.

%   \bigskip

%   Roughly, we will take the view that a formula's meaning is the
%   collection of situations that make it true.
%   \begin{mathpar}
%     \sem{G \lor R} = \left\{
%       \begin{array}{@{}l}
%         \{G \mapsto \true, R \mapsto \false\}, \\
%         \{G \mapsto \false, R \mapsto \true\}, \\
%         \{G \mapsto \true, R \mapsto \true\}
%       \end{array}
%     \right\}

%     \sem{G \land R} = \left\{ [G \mapsto \true,R \mapsto \true] \right\}
%   \end{mathpar}
% \end{frame}

% \begin{frame}
%   {Logical Modelling}

%   A fundamental use for logic is to create models of situations.

%   \bigskip

%   \begin{enumerate}
%   \item Encode potential facts as atomic statements:
%     \begin{itemize}
%     \item $\mathsf{cell}((3,3), 4)$ -- cell $(3,3)$ of a sudoku board contains $4$.
%     \item $\mathsf{assigned}(A,M)$ -- task $A$ is assigned to machine $M$.
%     \item $\mathsf{installed}(progA, 1)$ -- version 1 of program A is installed
%     \end{itemize}
%   \item Encode constraints as logical statements:
%     \begin{itemize}
%     \item $\mathsf{installed}(progA,1) \to (\mathsf{installed}(libB,3) \land \mathsf{installed}(libC,2))$ \\
%       -- program A v1 requires libB v3 and libC v2.
%     \end{itemize}
%   \item Finding a true/false value for each atom that makes all the
%     constraints true allows us to solve the problem.
%   \end{enumerate}
% \end{frame}

% \begin{frame}
%   {Deductive Systems}

%   \emph{Deductive Systems} are collections of rules that allow you to
%   derive valid statements from valid statements.

%   \bigskip

%   A rule:
%   \begin{displaymath}
%     \inferrule*
%     {X\textit{ is feathery} \\ X\textit{ lays eggs}}
%     {X\textit{ is a bird}}
%   \end{displaymath}

%   Above the line are \emph{premises}, below is the \emph{conclusion}.

%   \bigskip

%   We will study rules for Propositional and Predicate Logic.
% \end{frame}

% \begin{frame}
%   {Course plan}

%   \begin{tabular}{ll}
%     Week 1 & Recap Propositional Logic \\
%     Week 2 & Logical modelling and SAT solvers \\
%     Week 3 & More examples of Logical Modelling \\
%     Week 4 & More examples of Logical Modelling \\
%     Week 5 & Deductive proof for Propositional Logic \\
%     Week 6 & Predicate Logic \\
%     Week 7 & Deductive proof for Predicate Logic \\
%     Week 8 & Semantics of Predicate Logic \\
%     Week 9 & Equality and Arithmetic in Predicate Logic \\
%     Week 10 & Metatheory \\
%   \end{tabular}
% \end{frame}

\begin{frame}
  {Course Delivery}

  For Semester 1:
  \begin{itemize}
  \item \textbf{Lectures}: two/week \\
    \quad Mondays 12:00-13:00, JA325 \\
    \quad Fridays 12:00-13:00, TG312
    % \item \textbf{Video minilectures}: roughly four/week, roughly ten minutes each.

    \medskip

  \item \textbf{Tutorials}: Mondays and Fridays, assignment to follow. \\
    \quad Mondays TG314; 11:00-12:00 \\
    \quad Fridays JA314; 15:00-16:00
  \end{itemize}
\end{frame}

\begin{frame}
  {Course Materials}

  For Semester 1:
  \begin{itemize}
  \item MyPlace:
    \begin{enumerate}
    \item General course information
    \item Quizzes
    \item Coursework submission
    \end{enumerate}
  \item Course pages: \\
    \quad {\small \url{https://personal.cis.strath.ac.uk/robert.atkey/cs208}}
    \begin{enumerate}
    \item In-depth course notes
    \item Interactive exercises
    \end{enumerate}
  \end{itemize}
\end{frame}

\begin{frame}
  {Assessment}

  The whole course (Semesters 1\&2) is 30\% coursework, 70\% exam.

  \bigskip

  \begin{itemize}
  \item MyPlace Quizzes, $0.5\%$/week, totalling $5\%$ \\
    \textcolor{black!60}{to make sure you are progressing}
  \item One coursework assignment, worth $10\%$
    \begin{itemize}
    \item Issued: Monday 27th October
    \item Deadline: Wednesday 26th November
    \end{itemize}
  \item More courseworks in Semester 2 ($15\%$)
  \item Single exam for Logic and Algorithms in April/May 2026 ($70\%$)
  \end{itemize}

  \bigskip


\end{frame}

\begin{frame}
  \begin{center}
    {\Huge Hope you enjoy the course!} \\
    Check MyPlace for course arrangements and updates.
  \end{center}
\end{frame}

\end{document}
