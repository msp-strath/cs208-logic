% -*- TeX-engine: xetex -*-

\documentclass[xetex,aspectratio=169,14pt,hyperref={pdfpagelabels=true,pdflang={en-GB}}]{beamer}

\usepackage[sci,noslidestrathidentity]{strathclyde}
\strathsetidentity{Department of}{Computer \& Information Sciences}

\usepackage{lmodern}
\usepackage{subscript}
\usepackage{url}
\usepackage{pifont}
\usepackage{csquotes}
\usepackage{mathpartir}
\usepackage{stmaryrd}
\usepackage{multirow}
\usepackage{euler}
\usepackage[normalem]{ulem}

% 2013-09-08 SU adding xetex specifics
% http://www.woggie.net/2008/07/16/beamer-pdftex-and-xetex/
\usepackage{xltxtra}

% 2013-09-08 SU http://robjhyndman.com/hyndsight/xelatex/
\defaultfontfeatures{Ligatures=TeX}

% 2015-12-09 SU table beautified
\renewcommand{\arraystretch}{1.2}
\usepackage{booktabs}

%MGK compatibility
\newcommand{\hh}[1]
  {\medskip\textbf{\large #1}}
\newcommand{\pdu}[3]
  {#1\rightarrow #2 : &\ & \makebox[70mm][l]{$#3$}}



\setlength{\marginparwidth}{2cm}
\usepackage{todonotes}%[disable]
\let\OldTodo\todo
\renewcommand{\todo}{\OldTodo[inline]}%
\newcommand{\todolater}[1]{}% Things to do for next year


\DeclareTextCommandDefault{\nobreakspace}{\leavevmode\nobreak\ }
\usepackage{rotating}

\usepackage{pdfcomment}% To add alt text for images using \pdftooltip{}

\usepackage{appendixnumberbeamer}

\newcommand{\messageframe}[1]{\begin{frame}\begin{center}\Huge #1\end{center}\end{frame}}
\newcommand{\sechead}[1]{{\bf #1} \\}
\newcommand{\examplehead}[1]{{\bf Example:} {\it\textcolor{red!90}{#1}} \\}
\newcommand{\eqnote}[1]{\hspace{3cm}\textit{#1}}
\newcommand{\sidenote}[1]{\qquad {\footnotesize \textcolor{black!60}{(#1)}}}
\newcommand{\sem}[1]{\llbracket #1 \rrbracket}
\newcommand{\true}{\mathsf{T}}
\newcommand{\false}{\mathsf{F}}

\def\strikeafter<#1>#2{\temporal<#1>{#2}{\sout{#2}}{\sout{#2}}}


%\newcommand{\rhighlight}{\textcolor{titlered}}
\newcommand{\rhighlight}{\textbf}
\newcommand{\highlight}{\textbf}

% \setmainfont{Linux Biolinum O}
\setmainfont{LinBiolinum}[
Path=,
UprightFont = *_R.otf ,
BoldFont = *_RB.otf ,
ItalicFont = *_RI.otf
]

\setbeamertemplate{navigation symbols}{}
%\usecolortheme[rgb={0.8,0,0}]{structure}
\usefonttheme{serif}
\usefonttheme{structurebold}
\setbeamercolor{description item}{fg=black}


\author[Atkey]{Dr.~Robert Atkey}
\institute[Strathclyde]{Computer \& Information Sciences}
\date[]{}

\newcommand{\weeksection}[1]{%
  \section{\thetitle{}, Part~\thesection : #1}
  \begin{frame}
    \begin{center}
      \textcolor{black!60}{\thetitle{}, Part \thesection}\\
      {\Huge #1}
    \end{center}
  \end{frame}}

\newcommand{\weektitle}[2]{\def\thetitle{#2}
\title[CS208 - Topic #1]{CS208 (Semester 1) Topic #1 : #2}}

\newcommand{\assigned}{:}

\newcommand{\forcedto}{\assigned_f}
\newcommand{\decideto}{\assigned_d}


\title[CS208 - Introduction]{CS208 (Semester 1) : Introduction}

\begin{document}

\maketitle

\section{Introduction}

\begin{frame}{CS208 : Logic and Algorithms}

  In this semester we will study \emph{Symbolic Logic}.

  \bigskip

  \begin{enumerate}
  \item What is a good language for formal statements?
  \item What are valid rules of inference?
  \item How can we relate situations and statements about them?
  \end{enumerate}
\end{frame}

\begin{frame}
  {Very rough History of Logic}

  (with a Western bias)

  \bigskip

  \begin{tabular}{ll}
    \textcolor{black!60}{$\sim$350BCE}& Aristotle's \emph{Organon} (``\emph{logos}'' = ``the word'') \\
    \textcolor{black!60}{300BCE-1847CE}& Lots of interesting but scattered stuff... \\
    \textcolor{black!60}{1847} & George Boole's \emph{Mathematical Analysis of Logic} \\
%                & \quad later \emph{An Investigation of the Laws of Thought...} \\
    \textcolor{black!60}{1879} & Gottlob Frege's \emph{Begriffsschrift} \\
 %               & \quad invention of \emph{Quantifiers} \\
    \textcolor{black!60}{1880s to 1930s} & Logic\textbf{s} as an object of study in their own right\\
  %              & \quad Kurt G{\"o}del, Alfred Tarski, Alan Turing, ...\\
    \textcolor{black!60}{1940s to date} & Explosion, especially driven by CS.
  \end{tabular}
\end{frame}

\begin{frame}
  {Why study Logic?}

  Logic is \emph{extremely useful} in Computer Science.

  \bigskip

  For example...
  \begin{enumerate}
  \item Digital circuits
  \item Specifying systems' behaviour
  \item Proving that systems satisfy their specifications
  \item Automated Reasoning, rule-based AI
  \item Designing Programming Languages
  \item Database Query Languages
  \end{enumerate}
\end{frame}

\begin{frame}
  {In this course}

  We will cover two major sorts of Logic:

  \begin{enumerate}
  \item Propositional Logic
  \item Predicate Logic
  \end{enumerate}
\end{frame}

\begin{frame}
  {Propositional Logic}

  Propositional Logic talks about atomic statements:
  \begin{mathpar}
    \textit{``it is raining''}

    \textit{``I am in Glasgow''}
  \end{mathpar}
  and compound statements made from those:
  \begin{mathpar}
    \textit{``it is raining''} \land \textit{``I am in Glasgow''}

    \textit{``it is raining''} \to \textit{``I am in Glasgow''}
  \end{mathpar}

\end{frame}

\begin{frame}
  {Predicate Logic}

  Predicate Logic adds \emph{quantifiers} and \emph{predicates}:
  \begin{mathpar}
    \forall \mathit{day}.~\textrm{InGlasgow}(\mathit{day}) \to (\textrm{Raining}(\mathit{day}) \lor \textrm{Drizzle}(\mathit{day}))

    \forall x. \forall y.~x + y = y + x
  \end{mathpar}
\end{frame}

\begin{frame}
  {Semantics}

  The \emph{Semantics} of a logic tell you what formulas \emph{mean}.

  \bigskip

  Roughly, we will take the view that a formula's meaning is the
  collection of situations that make it true.
  \begin{mathpar}
    \sem{G \lor R} = \left\{
      \begin{array}{@{}l}
        \{G \mapsto \true, R \mapsto \false\}, \\
        \{G \mapsto \false, R \mapsto \true\}, \\
        \{G \mapsto \true, R \mapsto \true\}
      \end{array}
    \right\}

    \sem{G \land R} = \left\{ [G \mapsto \true,R \mapsto \true] \right\}
  \end{mathpar}
\end{frame}

\begin{frame}
  {Logical Modelling}

  A fundamental use for logic is to create models of situations.

  \bigskip

  \begin{enumerate}
  \item Encode potential facts as atomic statements:
    \begin{itemize}
    \item $\mathsf{cell}((3,3), 4)$ -- cell $(3,3)$ of a sudoku board contains $4$.
    \item $\mathsf{assigned}(A,M)$ -- task $A$ is assigned to machine $M$.
    \item $\mathsf{installed}(progA, 1)$ -- version 1 of program A is installed
    \end{itemize}
  \item Encode constraints as logical statements:
    \begin{itemize}
    \item $\mathsf{installed}(progA,1) \to (\mathsf{installed}(libB,3) \land \mathsf{installed}(libC,2))$ \\
      -- program A v1 requires libB v3 and libC v2.
    \end{itemize}
  \item Finding a true/false value for each atom that makes all the
    constraints true allows us to solve the problem.
  \end{enumerate}
\end{frame}

\begin{frame}
  {Deductive Systems}

  \emph{Deductive Systems} are collections of rules that allow you to
  derive valid statements from valid statements.

  \bigskip

  A rule:
  \begin{displaymath}
    \inferrule*
    {X\textit{ is feathery} \\ X\textit{ lays eggs}}
    {X\textit{ is a bird}}
  \end{displaymath}

  Above the line are \emph{premises}, below is the \emph{conclusion}.

  \bigskip

  We will study rules for Propositional and Predicate Logic.
\end{frame}

\begin{frame}
  {Course plan}

  \begin{tabular}{ll}
    Week 1 & Recap Propositional Logic \\
    Week 2 & Logical modelling and SAT solvers \\
    Week 3 & More examples of Logical Modelling \\
    Week 4 & More examples of Logical Modelling \\
    Week 5 & Deductive proof for Propositional Logic \\
    Week 6 & Predicate Logic \\
    Week 7 & Deductive proof for Predicate Logic \\
    Week 8 & Semantics of Predicate Logic \\
    Week 9 & Equality and Arithmetic in Predicate Logic \\
    Week 10 & Metatheory \\
  \end{tabular}
\end{frame}

\begin{frame}
  {Course Delivery}

  For Semester 1:
  \begin{itemize}
  \item \textbf{Lectures}: two/week \\
    Mondays 12:00-13:00, GH514 \\
    Fridays 12:00-13:00, RC512
  \item \textbf{Video minilectures}: roughly four/week, roughly ten minutes each.
  \item \textbf{Notes and Tutorial Exercises}: linked from MyPlace.
  \item \textbf{Tutorials}: Mondays and Fridays, assignment to follow.
  \item \textbf{Mattermost chat}: use the Mattermost channel for this
    course to discuss the course and ask questions.
  \item \textbf{Individual questions}: contact the course lecturer.
  \end{itemize}
\end{frame}

\begin{frame}
  {Assessment}

  The whole course (Semesters 1\&2) is 30\% coursework, 70\% exam.

  \bigskip

  This semester there are two courseworks:
  \begin{center}
    \begin{tabular}{llll}
      Name & Release & Submission & Weighting\\
      \hline
      1 : Logical Modelling &Wed 4th Oct &Wed 18th Oct &7.5 \% \\
      2 : Deductive Proof   &Wed 2nd Nov &Wed 30th Nov &7.5 \%\\
    \end{tabular}
  \end{center}

  \bigskip

  Single exam for Logic and Algorithms in April/May.
\end{frame}

\begin{frame}
  \begin{center}
    {\Huge Hope you enjoy the course!} \\
    Check MyPlace for course arrangements and updates.
  \end{center}
\end{frame}

\end{document}
