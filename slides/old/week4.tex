% -*- TeX-engine: xetex -*-

\documentclass[xetex,aspectratio=169,14pt,hyperref={pdfpagelabels=true,pdflang={en-GB}}]{beamer}

\usepackage[sci,noslidestrathidentity]{strathclyde}
\strathsetidentity{Department of}{Computer \& Information Sciences}

\usepackage{lmodern}
\usepackage{subscript}
\usepackage{url}
\usepackage{pifont}
\usepackage{csquotes}
\usepackage{mathpartir}
\usepackage{stmaryrd}
\usepackage{multirow}
\usepackage{euler}
\usepackage[normalem]{ulem}

% 2013-09-08 SU adding xetex specifics
% http://www.woggie.net/2008/07/16/beamer-pdftex-and-xetex/
\usepackage{xltxtra}

% 2013-09-08 SU http://robjhyndman.com/hyndsight/xelatex/
\defaultfontfeatures{Ligatures=TeX}

% 2015-12-09 SU table beautified
\renewcommand{\arraystretch}{1.2}
\usepackage{booktabs}

%MGK compatibility
\newcommand{\hh}[1]
  {\medskip\textbf{\large #1}}
\newcommand{\pdu}[3]
  {#1\rightarrow #2 : &\ & \makebox[70mm][l]{$#3$}}



\setlength{\marginparwidth}{2cm}
\usepackage{todonotes}%[disable]
\let\OldTodo\todo
\renewcommand{\todo}{\OldTodo[inline]}%
\newcommand{\todolater}[1]{}% Things to do for next year


\DeclareTextCommandDefault{\nobreakspace}{\leavevmode\nobreak\ }
\usepackage{rotating}

\usepackage{pdfcomment}% To add alt text for images using \pdftooltip{}

\usepackage{appendixnumberbeamer}

\newcommand{\messageframe}[1]{\begin{frame}\begin{center}\Huge #1\end{center}\end{frame}}
\newcommand{\sechead}[1]{{\bf #1} \\}
\newcommand{\examplehead}[1]{{\bf Example:} {\it\textcolor{red!90}{#1}} \\}
\newcommand{\eqnote}[1]{\hspace{3cm}\textit{#1}}
\newcommand{\sidenote}[1]{\qquad {\footnotesize \textcolor{black!60}{(#1)}}}
\newcommand{\sem}[1]{\llbracket #1 \rrbracket}
\newcommand{\true}{\mathsf{T}}
\newcommand{\false}{\mathsf{F}}

\def\strikeafter<#1>#2{\temporal<#1>{#2}{\sout{#2}}{\sout{#2}}}


%\newcommand{\rhighlight}{\textcolor{titlered}}
\newcommand{\rhighlight}{\textbf}
\newcommand{\highlight}{\textbf}

% \setmainfont{Linux Biolinum O}
\setmainfont{LinBiolinum}[
Path=,
UprightFont = *_R.otf ,
BoldFont = *_RB.otf ,
ItalicFont = *_RI.otf
]

\setbeamertemplate{navigation symbols}{}
%\usecolortheme[rgb={0.8,0,0}]{structure}
\usefonttheme{serif}
\usefonttheme{structurebold}
\setbeamercolor{description item}{fg=black}


\author[Atkey]{Dr.~Robert Atkey}
\institute[Strathclyde]{Computer \& Information Sciences}
\date[]{}

\newcommand{\weeksection}[1]{%
  \section{\thetitle{}, Part~\thesection : #1}
  \begin{frame}
    \begin{center}
      \textcolor{black!60}{\thetitle{}, Part \thesection}\\
      {\Huge #1}
    \end{center}
  \end{frame}}

\newcommand{\weektitle}[2]{\def\thetitle{#2}
\title[CS208 - Topic #1]{CS208 (Semester 1) Topic #1 : #2}}

\newcommand{\assigned}{:}

\newcommand{\forcedto}{\assigned_f}
\newcommand{\decideto}{\assigned_d}


\weektitle{4}{Natural Deduction I}

\begin{document}

\frame{\titlepage}

\weeksection{Deductive Reasoning}
% Subtitle:     \textcolor{black!60}{To Truth Through Proof!}\footnote{Apologies to Peter B. Andrews: \url{http://gtps.math.cmu.edu/tttp.html}}

\begin{frame}
  \sechead{Why have logic(s)?}
  One reason is to study ``arguments''.
  \begin{itemize}
  \item To separate valid and invalid reasoning.
  \item If we assume $P_1, P_2, P_3$, then when is it valid to conclude $Q$?
  \end{itemize}

  \pause
  \bigskip

  \sechead{One answer is ``entailment''}
  \begin{itemize}
  \item $P_1, \dots \models Q$ ``is'' valid reasoning from assumptions to a conclusion.
  \end{itemize}
  \emph{Entailment} is defined in terms of the semantics of formulas
  \begin{itemize}
  \item $P_1, ... \models Q$ if \highlight<3->{for all valuations $v$}, $\sem{P}v = \true$ implies $\sem{Q}v \kern-3pt= \true$
  \end{itemize}

  \pause
  \pause
  \bigskip

  \sechead{This doesn't match how we reason normally.}
  If we are trying to convince someone, we don't (usually) say: \\
  \qquad \emph{``let's go through all the combinations of truth values and test each one.''}
\end{frame}

\begin{frame}
  \sechead{Chains of Inference}
  Usually, we might say things like:
  \begin{enumerate}
  \item Let's assume that $A, B, C$ are true.
  \item If we assume $A$ and $B$ imply $D$, then $D$ is true.
  \item If we assume $C$ and $D$ imply $E$, then $E$ is true.
  \item So, we can conclude $E$, under the assumptions.
  \end{enumerate}
  If our reasoning is sound, then we ought to be able to conclude
  \begin{displaymath}
    A, B, C, (A \land B) \to D, (C \land D) \to E \models E
  \end{displaymath}

  \pause
  \sechead{We have a form of \emph{modularity}}
  \begin{itemize}
  \item We don't check the entailment for every possible truth value of $A, B, C, D, E$ \qquad \textcolor{black!60}{($2^5 = 32$ combinations!)}
  \item We apply individual reasoning \emph{steps} and chain them together.
  \end{itemize}
\end{frame}

\begin{frame}
  {Semantic Reasoning doesn't scale}

  In \emph{Propositional Logic}, it is possible (though not always
  feasible) to check all cases.
  \begin{itemize}
  \item If there are $n$ atomic propositions, check $2^n$ combinations.
  \item SAT solvers are good at only checking the ones that matter.
  \item But there are still Hard Problems that take too long.
  \end{itemize}

  \pause
  \bigskip

  Also, later in the course we will study \emph{Predicate Logic}
  \begin{itemize}
  \item Predicate logic allows \emph{universal} statements:
    \begin{displaymath}
      \forall x.\forall y.~x+y = y+x
    \end{displaymath}
    \emph{``For all (numbers) $x$ and $y$, $x+y$ is equal to $y+x$''}
  \item Simply not possible to exhaustively check all numbers.
  \end{itemize}
\end{frame}

\begin{frame}
  {Deductive Systems}
  To overcome these problems, we use \emph{deductive systems}.

  \bigskip

  A \rhighlight{deductive system} is a collection of \emph{rules} for
  deriving conclusions from assumptions.
  \begin{itemize}
  \item Typically, the rules are ``finitely describable'' \\
    \sidenote{roughly: we can implement them on a computer}
  \end{itemize}

  \bigskip

  Typically (but not always), we write
  \begin{displaymath}
    P_1, \cdots, P_n \vdash Q
  \end{displaymath}
  when we can derive conclusion $Q$ from assumptions $P_1, \cdots, P_n$.

  % \bigskip There are many deductive systems. We will look at a variant
  % of Natural Deduction.

\end{frame}

\begin{frame}
  {Soundness and Completeness}

  \textbf{Soundness} : ``Everything that is provable is valid''
  \begin{displaymath}
    P_1, \cdots, P_n \vdash Q \qquad \textit{implies} \qquad P_1, \cdots, P_n \models Q
  \end{displaymath}
  \sidenote{pretty much a requirement to be useful}

  \bigskip

  \textbf{Completeness} : ``Everything that is valid is provable''
  \begin{displaymath}
    P_1, \cdots, P_n \models Q \qquad \textit{implies} \qquad P_1, \cdots, P_n \vdash Q
  \end{displaymath}
  \sidenote{not \emph{essential}, but good to have}

\end{frame}

\begin{frame}
  {Advantages of Deductive Systems}

  \bigskip \emph{1. } We can write computer programs to check our proofs, even
  when talking about infinitely many things.

  \pause
  \bigskip
  \emph{2. }If we remove or alter rules do we get an interesting new logic?

  \pause
  \bigskip
  \emph{3. }We can start to ask questions about the proofs:
  \begin{itemize}
  \item An entailment $P_1, \cdots, P_n \models Q$ is either valid or invalid. Meh.
  \item but there may be many proofs (ways of applying the rules).
  \item Questions:
    \begin{itemize}
    \item Do different proofs \emph{mean} different things?
    \item Is one proof a simplification of another?
    \item Is there information hidden in proofs that we can extract?
    \end{itemize}
  \end{itemize}
\end{frame}

\begin{frame}[t]
  {Inference Rules}

  \begin{displaymath}
    \inferrule*
    {\mathit{premise}_1 \\ \cdots \\ \mathit{premise}_n}
    {\mathit{conclusion}}
  \end{displaymath}

  The idea:
  \begin{itemize}
  \item If we can prove all of $\mathit{premise}_1$ and ... and $\mathit{premise}_n$; then
  \item we have a proof of $\mathit{conclusion}$.
  \end{itemize}

  \pause
  \bigskip

  We might have zero premises, \\
  \quad in which case the $\mathit{conclusion}$
  requires no proof (``is an axiom'').

  \pause
  \bigskip

  Rules are organised into \emph{trees} to make \emph{deductions}.
\end{frame}

\begin{frame}[t]
  {Example}
  \begin{mathpar}
    \inferrule* [right=Rule1] { } {\textrm{bears}~\textit{are furry}}

    \inferrule* [right=Rule2] { } {\textrm{bears}~\textit{make milk}}

    \inferrule* [right=Rule3]
    {X~\textit{are furry} \\ X~\textit{make milk}}
    {X~\textit{are mammals}}
  \end{mathpar}

  \pause
  \bigskip

  A deduction:
  \begin{displaymath}
    \inferrule* [right=Rule3]
    {\inferrule* [right=Rule1] { } {\textrm{bears}~\textit{are furry}} \\
      \inferrule* [Right=Rule2] { } {\textrm{bears}~\textit{make milk}}}
    {\textrm{bears}~\textit{are mammals}}
  \end{displaymath}
\end{frame}

\begin{frame}
  {Example (cont.)}

  \begin{mathpar}
    \inferrule* [right=Rule4]
    {X~\textit{are covered in fibres}}
    {X~\textit{are furry}}

    \inferrule* [right=Rule5]
    { }
    {\textrm{coconuts}~\textit{are covered in fibres}}

    \inferrule* [right=Rule6]
    { }
    {\textrm{coconuts}~\textit{make milk}}
  \end{mathpar}

\end{frame}

\begin{frame}
  {Example (cont.)}

  Another deduction:
  \begin{displaymath}
    \inferrule* [right=R3]
    {\inferrule* [right=R4]
      {\inferrule* [Right=R5] { } {\textrm{coconuts}~\textit{are covered in fibres}}}
      {\textrm{coconuts}~\textit{are furry}}
      \\
      \inferrule* [Right=R6]
      { } {\textrm{coconuts}~\textit{make milk}}}
    {\textrm{coconuts}~\textit{are mammals}}
  \end{displaymath}
\end{frame}

\begin{frame}[t]
  {Example (cont.)}
  When \emph{building} deductions, we work bottom up:

  \begin{minipage}[t][2cm][b]{\textwidth}
  \begin{displaymath}
    \only<2>{\textrm{coconuts}~\textit{are mammals}}%
    \only<3>{\inferrule* [right=R3]
      {\textrm{coconuts}~\textit{are furry} \\ \textrm{coconuts}~\textit{make milk}}
      {\textrm{coconuts}~\textit{are mammals}}}%
    \only<4>{\inferrule* [right=R3]
      {\inferrule* [right=R4] {\textrm{coconuts}~\textit{are covered in fibres}} {\textrm{coconuts}~\textit{are furry}}
        \\
        \textrm{coconuts}~\textit{make milk}}
      {\textrm{coconuts}~\textit{are mammals}}}%
    \only<5>{\inferrule* [right=R3]
      {\inferrule* [right=R4]
        {\inferrule* [Right=R5] { } {\textrm{coconuts}~\textit{are covered in fibres}}}
        {\textrm{coconuts}~\textit{are furry}}
        \\
        \textrm{coconuts}~\textit{make milk}}
      {\textrm{coconuts}~\textit{are mammals}}}%
    \only<6-7>{\inferrule* [right=R3]
    {\inferrule* [right=R4]
      {\inferrule* [Right=R5] { } {\textrm{coconuts}~\textit{are covered in fibres}}}
      {\textrm{coconuts}~\textit{are furry}}
      \\
      \inferrule* [Right=R6]
      { } {\textrm{coconuts}~\textit{make milk}}}
      {\textrm{coconuts}~\textit{are mammals}}}
  \end{displaymath}
\end{minipage}

\bigskip

  \begin{enumerate}
  \item<2-> Write down the conclusion
  \item<3-> Apply rule \TirName{Rule3} ($X$ \emph{are mammals} if $X$ \emph{are furry and make milk})
  \item<4-> Apply rule \TirName{Rule4} ($X$ \emph{are furry} if they \emph{are covered in fibres})
  \item<5-> Apply rule \TirName{Rule5} (an axiom)
  \item<6-> Apply rule \TirName{Rule6} (an axiom)
  \end{enumerate}

  % \bigskip
  % \pause\pause\pause\pause\pause\pause
  % \begin{itemize}
  % \item A statement with no line above it is an \emph{open goal}
  % \item A branch of the tree that ends with an axiom is called \emph{closed}
  % \end{itemize}
\end{frame}

\begin{frame}
  {Summary}

  \begin{itemize}
  \item The \emph{why?} of deductive systems.
  \item Inference rules.
  \item How to make chains of inference.
  \end{itemize}
\end{frame}


\weeksection{Natural Deduction}

\begin{frame}[t]
  {Judgements}

  We want to deduce \emph{judgements} of the form:
  \begin{displaymath}
    P_1, \dots, P_n \vdash Q
  \end{displaymath}
  meaning:
  \begin{center}
    From assumptions $P_1, \dots, P_n$, we can prove $Q$.
  \end{center}

  \bigskip

  {\bf Soundness}
  The system will be \emph{sound}, meaning:
  \begin{displaymath}
    P_1,\dots,P_n \vdash Q~\textrm{provable}~~\Rightarrow~~P_1, \dots, P_n \models Q
  \end{displaymath}

  We will make sure it is sound by checking each rule as we go. \\
  \quad \textcolor{black!60}{If all the premises are valid entailments, then so is the conclusion}
\end{frame}

\begin{frame}[t]
  {Judgements}

  The main judgement form is
  \begin{displaymath}
    P_1, \dots, P_n \vdash Q
  \end{displaymath}
  \qquad \textcolor{black!60}{With assumptions $P_1,\dots,P_n$, can prove $Q$}

  \pause
  \bigskip

  We will also use an auxiliary judgement:
  \begin{displaymath}
    P_1, \dots, P_n~[P] \vdash Q
  \end{displaymath}
  \quad \textcolor{black!60}{$\cdot$ With assumptions $P_1,\dots,P_n$, \emph{focusing on} $P$, can prove $Q$}\\
  \quad \textcolor{black!60}{$\cdot$ Also ``means'' $P_1,\dots,P_n,P \models Q$}\\
  \quad \textcolor{black!60}{$\cdot$ Having a focus is useful for organising proofs}
\end{frame}

\begin{frame}[t]
  {Judgements}

  The main judgement form is
  \begin{displaymath}
    P_1, \dots, P_n \vdash Q
  \end{displaymath}

  \medskip

  We will also use an auxiliary judgement:
  \begin{displaymath}
    P_1, \dots, P_n~[P] \vdash Q
  \end{displaymath}

  \pause
  \bigskip

  \textbf{Assumption lists}~The list of assumptions
  $P_1, \dots, P_n$ will appear often. So we
  will shorten it to $\Gamma$
  \textcolor{black!60}{$ = P_1, \dots, P_n$}.
\end{frame}

\newcommand{\pUse}[1]{\texttt{use}~#1\texttt{,}~}
\newcommand{\pDone}{\texttt{done}}
\newcommand{\pSplit}[2]{\texttt{split(}#1~\texttt{|}~#2\texttt{)}}
\newcommand{\pFst}{\texttt{first,}~}
\newcommand{\pSnd}{\texttt{second,}~}
\newcommand{\pTrue}{\texttt{true}}

%\titlecard{Basic Rules}

\begin{frame}
  {Basic Rules}

  \begin{displaymath}
    \inferrule* [right=Done]
    { }
    {\Gamma~[P] \vdash P}
  \end{displaymath}

  \bigskip
  \bigskip
  \pause

  \begin{itemize}
  \item If we have a focused assumption $P$, then we can prove $P$
  \item (Remember $\Gamma$ stands for a list of other assumptions)
  \end{itemize}
\end{frame}

\begin{frame}
  {Basic Rules}

  \begin{displaymath}
    \inferrule* [right=Use]
    {P \in \Gamma \\ \Gamma~[P] \vdash Q}
    {\Gamma \vdash Q}
  \end{displaymath}

  \bigskip
  \bigskip
  \pause

  \begin{itemize}
  \item $P \in \Gamma$ means ``$P$ is in $\Gamma$''.
  \item If we have a $P$ in our current assumptions, we can focus on it.
  \item $P \in \Gamma$ is a \emph{side condition}: it is something we
    check when we apply the rule, not another judgement to be proved.
  \end{itemize}
\end{frame}

\begin{frame}[t]
  {A first proof}

  \bigskip

  \begin{minipage}[t][2cm][b]{\textwidth}
    \begin{displaymath}
      \only<1>{A \vdash A\quad\quad}%
      \only<2>{\inferrule* [right=Use] {A~[A] \vdash A} {A \vdash A}}%
      \only<3>{\inferrule* [right=Use] {\inferrule* [Right=Done] { } {A~[A] \vdash A}} {A \vdash A}}
    \end{displaymath}
  \end{minipage}

  \bigskip

  \begin{itemize}
  \item<2-> First \TirName{Use} the $A$ assumption.
  \item<3-> Then we are \TirName{Done}.
  \end{itemize}
\end{frame}

\begin{frame}
  {Soundness}

  \begin{mathpar}
    \inferrule* [right=Done]
    { }
    {\Gamma~[P] \vdash P}

    \inferrule* [right=Use]
    {P \in \Gamma \\ \Gamma~[P] \vdash Q}
    {\Gamma \vdash Q}
  \end{mathpar}

  \pause
  \bigskip

  \TirName{Done} is sound because assuming $P$ entails $P$, and extra
  assumptions make no difference.

  \pause
  \bigskip

  \TirName{Use} is sound because if we assuming $P$ twice entails $Q$,
  then it is okay to assume it once.
\end{frame}

\begin{frame}
  {Rules for connectives}

  \bigskip

  The rule \TirName{Done} and \TirName{Use} do not mention the connectives.

  \bigskip

  In Natural Deduction, rules for connectives come in two kinds:

  \bigskip

  \begin{enumerate}
  \item {\bf Introduction} rules\\
    \textcolor{black!60}{How to \emph{construct} a proof with the connective}
  \item {\bf Elimination} rules \\
    \textcolor{black!60}{How to \emph{use} an assumption with this connective}
  \end{enumerate}

  \bigskip
  \pause

  Very rough analogy:\quad \textcolor{black!60}{but can be made very precise}
  \begin{enumerate}
  \item Introduction rules are like \emph{constructors} for building objects
  \item Elimination rules are like \emph{methods} for taking apart objects
  \end{enumerate}
\end{frame}

%\titlecard{Rules for Conjunction}

\begin{frame}
  {``And'' Introduction}


  \begin{displaymath}
    \inferrule* [right=Split]
    {\Gamma \vdash Q_1 \\ \Gamma \vdash Q_2}
    {\Gamma \vdash Q_1 \land Q_2}
  \end{displaymath}

  \bigskip
  \pause

  \begin{itemize}
  \item To prove $P_1 \land P_2$ we have to prove $P_1$ and $P_2$
  \item This rule is often called \TirName{$\land$-Introduction}
  \end{itemize}
\end{frame}

\begin{frame}
  \sechead{An example proof}

  \bigskip

  \begin{displaymath}
    \inferrule* [Right=Split]
    { \inferrule* [right=Use]
      {\inferrule* [Right=Done]
        { } {A, B~[A] \vdash A}}
      {A, B \vdash A}
      \\
      \inferrule* [Right=Use]
      {\inferrule* [Right=Done]
        { } {A, B~[B] \vdash B}
      }
      {A, B \vdash B}
    }
    {A, B \vdash A \land B}
  \end{displaymath}

  \pause
  \bigskip

  To prove $A \land B$, we \TirName{Split} into proofs of $A$ and
  $B$. \\
  In each case, we \TirName{Use} the corresponding assumption.
\end{frame}

\begin{frame}
  {``And'' Elimination}

  \begin{mathpar}
    \inferrule* [right=First]
    {\Gamma~[P_1] \vdash Q}
    {\Gamma~[P_1 \land P_2] \vdash Q}

    \inferrule* [right=Second]
    {\Gamma~[P_2] \vdash Q}
    {\Gamma~[P_1 \land P_2] \vdash Q}
  \end{mathpar}

  \bigskip
  \pause

  If we are focused on an formula $P_1 \land P_2$, we can select
  either the \TirName{First} or \TirName{Second} component to focus
  on.
\end{frame}

\begin{frame}
  \sechead{Example proof}

  \begin{displaymath}
    \inferrule* [Right=Use]
    {\inferrule* [Right=Second]
      {\inferrule* [Right=Done]
        { }
        {A \land B~[B] \vdash B}
      }
      {A \land B~[A \land B] \vdash B}
    }
    {A \land B \vdash B}
  \end{displaymath}
\end{frame}

%\titlecard{Rules for Truth}

\begin{frame}
  {``True'' Introduction}

  \bigskip

  \begin{displaymath}
    \inferrule* [right=True]
    { }
    {\Gamma \vdash \true}
  \end{displaymath}

  \bigskip
  \pause

  \begin{itemize}
  \item $\true$ is always provable.
  \end{itemize}
\end{frame}

\begin{frame}
  {``True'' Elimination}

  \pause
  \bigskip

  \begin{center}
    \emph{No elimination rule!}
  \end{center}
\end{frame}

\begin{frame}
  {Summary}

  \begin{itemize}
  \item The judgement forms for (focused) Natural Deduction:
    \begin{mathpar}
      P_1, \dots, P_n \vdash Q

      P_1, \dots, P_n~[P] \vdash Q
    \end{mathpar}
  \item Rules for \TirName{Use} and \TirName{Done}
  \item Rules for introducing and eliminating $\land$.
  \end{itemize}
\end{frame}

\weeksection{Rules for ``Implies''}

\begin{frame}
  {``Implies'' Introduction}

  \begin{displaymath}
    \inferrule* [right=Introduce]
    {\Gamma, P \vdash Q}
    {\Gamma \vdash P \to Q}
  \end{displaymath}

  \pause
  \bigskip

  To prove $P \to Q$, we prove $Q$ under the assumption $P$.
\end{frame}

\begin{frame}
  {Example: $A \to A$}

  \begin{displaymath}
    \inferrule* [right=Introduce]
    {\inferrule* [Right=Use]
      {\inferrule* [Right=Done] { } {A~[A] \vdash A}}
      {A \vdash A}
    }
    {\vdash A \to A}
  \end{displaymath}

\end{frame}

\begin{frame}
  {Example : $(A \land B) \to A$}

  \begin{displaymath}
    \inferrule* [right=Introduce]
    {\inferrule* [Right=Use]
      {\inferrule* [Right=First]
        {\inferrule* [Right=Done]
          { }
          {\mathrm{A} \land \mathrm{B}~[\mathrm{A}] \vdash \mathrm{A}}
        }
        {\mathrm{A} \land \mathrm{B}~[\mathrm{A} \land \mathrm{B}] \vdash \mathrm{A}}
      }
      {\mathrm{A} \land \mathrm{B} \vdash \mathrm{A}}
    }
    { \vdash (\mathrm{A} \land \mathrm{B}) \to \mathrm{A}}
  \end{displaymath}
\end{frame}

\begin{frame}
  {``Implies'' Elimination}

  \begin{displaymath}
    \inferrule* [right=Apply]
    {\Gamma \vdash P_1 \\ \Gamma~[P_2] \vdash Q}
    {\Gamma~[P_1 \to P_2] \vdash Q}
  \end{displaymath}

  \bigskip
  \pause

  If we have $P_1 \to P_2$ and we can prove $P_1$, then we have $P_2$.
\end{frame}

\begin{frame}
  {Example: $A \to (A \to B) \to B$}

  \begin{displaymath}
    \inferrule* [right=Introduce]
    {\inferrule* [Right=Introduce]
      {\inferrule* [Right=Use]
        {\inferrule* [Right=Apply]
          {\inferrule* [right=Use]
            {\inferrule* [Right=Done]
              { }
              {\mathrm{A}, \mathrm{A} \to \mathrm{B}~[\mathrm{A}] \vdash \mathrm{A}}
            }
            {\mathrm{A}, \mathrm{A} \to \mathrm{B} \vdash \mathrm{A}}
            \\
            \inferrule* [Right=Done]
            { }
            {\mathrm{A}, \mathrm{A} \to \mathrm{B}~[\mathrm{B}] \vdash \mathrm{B}}
          }
          {\mathrm{A}, \mathrm{A} \to \mathrm{B}~[\mathrm{A} \to \mathrm{B}] \vdash \mathrm{B}}
        }
        {\mathrm{A}, \mathrm{A} \to \mathrm{B} \vdash \mathrm{B}}
      }
      {\mathrm{A} \vdash (\mathrm{A} \to \mathrm{B}) \to \mathrm{B}}
    }
    { \vdash \mathrm{A} \to (\mathrm{A} \to \mathrm{B}) \to \mathrm{B}}
  \end{displaymath}
\end{frame}


% \begin{frame}

%   Examples (see scans):
%   \begin{enumerate}
%   \item $\vdash A \to (A \lor B)$
%   \item $\vdash A \to B \to (A \land B)$
%   \end{enumerate}
% \end{frame}

% \begin{frame}
%   Examples (see scans):
%   \begin{enumerate}
%   \item $A \to B, A \vdash B$
%   \item $A \lor B, B \to C \vdash A \lor C$
%   \item $A \to B \to C \vdash (A \land B) \to C$
%   \end{enumerate}
% \end{frame}

\begin{frame}
  {The Rules so far}

  \begin{mathpar}
    \inferrule* [right=Done]
    { }
    {\Gamma~[P] \vdash P}

    \inferrule* [right=Use]
    {P \in \Gamma \\ \Gamma~[P] \vdash Q}
    {\Gamma \vdash Q}

    \inferrule* [right=Split]
    {\Gamma \vdash Q_1 ~~~~~~ \Gamma \vdash Q_2}
    {\Gamma \vdash Q_1 \land Q_2}
    \quad
    \inferrule* [right=First]
    {\Gamma~[P_1] \vdash Q}
    {\Gamma~[P_1 \land P_2] \vdash Q}
    ~
    \inferrule* [right=Second]
    {\Gamma~[P_2] \vdash Q}
    {\Gamma~[P_1 \land P_2] \vdash Q}

    \inferrule* [right=Introduce]
    {\Gamma, P \vdash Q}
    {\Gamma \vdash P \to Q}

    \inferrule* [right=Apply]
    {\Gamma \vdash P_1 \\ \Gamma~[P_2] \vdash Q}
    {\Gamma~[P_1 \to P_2] \vdash Q}
  \end{mathpar}
\end{frame}

\begin{frame}
  {Summary}

  \begin{itemize}
  \item The rules for Implication
    \begin{mathpar}
      \inferrule* [right=Introduce]
      {\Gamma, P \vdash Q}
      {\Gamma \vdash P \to Q}

      \inferrule* [right=Apply]
      {\Gamma \vdash P_1 \\ \Gamma~[P_2] \vdash Q}
      {\Gamma~[P_1 \to P_2] \vdash Q}
    \end{mathpar}
  \end{itemize}
\end{frame}

%%%%%%%%%%%%%%%%%%%%%%%%%%%%%%%%%%%%%%%%%%%%%%%%%%%%%%%%%%%%%%%%%%%%%%%%%%%%%%
%%%%%%%%%%%%%%%%%%%%%%%%%%%%%%%%%%%%%%%%%%%%%%%%%%%%%%%%%%%%%%%%%%%%%%%%%%%%%%
%%%%%%%%%%%%%%%%%%%%%%%%%%%%%%%%%%%%%%%%%%%%%%%%%%%%%%%%%%%%%%%%%%%%%%%%%%%%%%

\weeksection{Using the Interactive Editor}


\end{document}
