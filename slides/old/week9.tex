% -*- TeX-engine: xetex -*-

\documentclass[xetex,aspectratio=169,14pt,hyperref={pdfpagelabels=true,pdflang={en-GB}}]{beamer}

\usepackage[sci,noslidestrathidentity]{strathclyde}
\strathsetidentity{Department of}{Computer \& Information Sciences}

\usepackage{lmodern}
\usepackage{subscript}
\usepackage{url}
\usepackage{pifont}
\usepackage{csquotes}
\usepackage{mathpartir}
\usepackage{stmaryrd}
\usepackage{multirow}
\usepackage{euler}
\usepackage[normalem]{ulem}

% 2013-09-08 SU adding xetex specifics
% http://www.woggie.net/2008/07/16/beamer-pdftex-and-xetex/
\usepackage{xltxtra}

% 2013-09-08 SU http://robjhyndman.com/hyndsight/xelatex/
\defaultfontfeatures{Ligatures=TeX}

% 2015-12-09 SU table beautified
\renewcommand{\arraystretch}{1.2}
\usepackage{booktabs}

%MGK compatibility
\newcommand{\hh}[1]
  {\medskip\textbf{\large #1}}
\newcommand{\pdu}[3]
  {#1\rightarrow #2 : &\ & \makebox[70mm][l]{$#3$}}



\setlength{\marginparwidth}{2cm}
\usepackage{todonotes}%[disable]
\let\OldTodo\todo
\renewcommand{\todo}{\OldTodo[inline]}%
\newcommand{\todolater}[1]{}% Things to do for next year


\DeclareTextCommandDefault{\nobreakspace}{\leavevmode\nobreak\ }
\usepackage{rotating}

\usepackage{pdfcomment}% To add alt text for images using \pdftooltip{}

\usepackage{appendixnumberbeamer}

\newcommand{\messageframe}[1]{\begin{frame}\begin{center}\Huge #1\end{center}\end{frame}}
\newcommand{\sechead}[1]{{\bf #1} \\}
\newcommand{\examplehead}[1]{{\bf Example:} {\it\textcolor{red!90}{#1}} \\}
\newcommand{\eqnote}[1]{\hspace{3cm}\textit{#1}}
\newcommand{\sidenote}[1]{\qquad {\footnotesize \textcolor{black!60}{(#1)}}}
\newcommand{\sem}[1]{\llbracket #1 \rrbracket}
\newcommand{\true}{\mathsf{T}}
\newcommand{\false}{\mathsf{F}}

\def\strikeafter<#1>#2{\temporal<#1>{#2}{\sout{#2}}{\sout{#2}}}


%\newcommand{\rhighlight}{\textcolor{titlered}}
\newcommand{\rhighlight}{\textbf}
\newcommand{\highlight}{\textbf}

% \setmainfont{Linux Biolinum O}
\setmainfont{LinBiolinum}[
Path=,
UprightFont = *_R.otf ,
BoldFont = *_RB.otf ,
ItalicFont = *_RI.otf
]

\setbeamertemplate{navigation symbols}{}
%\usecolortheme[rgb={0.8,0,0}]{structure}
\usefonttheme{serif}
\usefonttheme{structurebold}
\setbeamercolor{description item}{fg=black}


\author[Atkey]{Dr.~Robert Atkey}
\institute[Strathclyde]{Computer \& Information Sciences}
\date[]{}

\newcommand{\weeksection}[1]{%
  \section{\thetitle{}, Part~\thesection : #1}
  \begin{frame}
    \begin{center}
      \textcolor{black!60}{\thetitle{}, Part \thesection}\\
      {\Huge #1}
    \end{center}
  \end{frame}}

\newcommand{\weektitle}[2]{\def\thetitle{#2}
\title[CS208 - Topic #1]{CS208 (Semester 1) Topic #1 : #2}}

\newcommand{\assigned}{:}

\newcommand{\forcedto}{\assigned_f}
\newcommand{\decideto}{\assigned_d}


\weektitle{9}{Equality and Arithmetic}

\begin{document}

\frame{\titlepage}

\weeksection{Rules for Equality}

\begin{frame}
  {What is Equality?}

  \begin{displaymath}
    t_1 = t_2
  \end{displaymath}
\end{frame}

\begin{frame}
  {What is Equality?}

  Some properties:
  \begin{enumerate}
  \item \emph{Reflexivity: } for all $x$, $x = x$
  \item \emph{Symmetry: } for all $x$ and $y$, if $x = y$ then $y = x$
  \item \emph{Transitivity: } for all $x$, $y$ and $z$, if $x = y$ and $y = z$, then $x = z$
  \end{enumerate}

  \pause
  \bigskip

  Any binary relation that satisfies these properties is called an
  \emph{equivalence relation}.
\end{frame}

\begin{frame}
  {What is Equality?}

  The \textbf{special} property of equality is the following:

  \bigskip

  \begin{center}
    If $s = t$, then everything that is true about $s$ is true about $t$.
  \end{center}

  \pause
  \bigskip

  \textcolor{black!60}{(and vice versa. but do we need to say this?)}

  \pause
  \bigskip

  Gottfried Leibniz (co-inventor of Calculus) took this as the
  \emph{definition} of equality.
\end{frame}

\begin{frame}
  {What is Equality?}

  With more symbols:
  \begin{center}
    If $t_1 = t_2$, then for all $P$, if $P[x \mapsto t_1]$ then $P[x \mapsto t_2]$
  \end{center}

\end{frame}

\begin{frame}
  {What is Equality?}

  All we will need is:
  \begin{enumerate}
  \item Reflexivity: for every term $t$, $t = t$
  \item Substitution: $t_1 = t_2$ and $P[x \mapsto t_1]$ implies $P[x \mapsto t_2]$
  \end{enumerate}

  \bigskip

  Amazingly, this is enough!
\end{frame}

\begin{frame}
  {Symmetry}

  To prove that $x = y$ implies $y = x$:
  \begin{enumerate}
  \item We know that $x = x$ by reflexivity
  \item So we use our assumption to replace the first $x$ by $y$ to get $y = x$.
  \end{enumerate}
\end{frame}

\begin{frame}
  {Transitivity}

  To prove that $x = y$ and $y = z$ implies $x = z$:
  \begin{enumerate}
  \item Substitute the second assumption in the first to get $x = z$.
  \end{enumerate}
\end{frame}

\begin{frame}
  {Rules for Equality: Introduction}

  \begin{displaymath}
    \inferrule* [right=Reflexivity]
    { }
    {\Gamma \vdash t = t}
  \end{displaymath}

  \bigskip

  Every term is equal to itself.
\end{frame}

\begin{frame}
  {Rules for Equality: Elimination}

  \begin{displaymath}
    \inferrule* [right=Subst]
    {\Gamma \vdash P[x \mapsto t_2]}
    {\Gamma~[t_1 = t_2] \vdash P[x \mapsto t_1]}
  \end{displaymath}

  If we know that $t_1 = t_2$ then we can replace $t_1$ with $t_2$ in
  the goal. This is substitution backwards: if we know
  $P[x \mapsto t_2]$ and $t_1 = t_2$, then we know $P[x \mapsto t_1]$.
\end{frame}

\begin{frame}
  {Example: Symmetry}

  \begin{displaymath}
    \inferrule* [right=Introduce]
    {\inferrule* [Right=Introduce]
      {\inferrule* [Right=Introduce]
        {\inferrule* [Right=Use]
          {\inferrule* [Right=Subst]
            {\inferrule* [Right=Reflexivity]
              { }
              {\mathit{x}, \mathit{y}, x = y \vdash y = y}
            }
            {\mathit{x}, \mathit{y}, x = y~[x = y] \vdash y = x}
          }
          {\mathit{x}, \mathit{y}, x = y \vdash y = x}
        }
        {\mathit{x}, \mathit{y} \vdash x = y \to y = x}
      }
      {\mathit{x} \vdash \forall \mathit{y}. x = y \to y = x}
    }
    { \vdash \forall \mathit{x}. \forall \mathit{y}. x = y \to y = x}
  \end{displaymath}
\end{frame}

\begin{frame}
  {Example: Transitivity}

  {\small
  \begin{displaymath}
    \inferrule* [right=Introduce]
    {\inferrule* [Right=Introduce]
      {\inferrule* [Right=Introduce]
        {\inferrule* [Right=Introduce]
          {\inferrule* [Right=Introduce]
            {\inferrule* [Right=Use]
              {\inferrule* [Right=Subst]
                {\inferrule* [Right=Use]
                  {\inferrule* [Right=Done]
                    { }
                    {\mathit{x}, \mathit{y}, \mathit{z}, x = y, y = z~[y = z] \vdash y = z}
                  }
                  {\mathit{x}, \mathit{y}, \mathit{z}, x = y, y = z \vdash y = z}
                }
                {\mathit{x}, \mathit{y}, \mathit{z}, x = y, y = z~[x = y] \vdash x = z}
              }
              {\mathit{x}, \mathit{y}, \mathit{z}, x = y, y = z \vdash x = z}
            }
            {\mathit{x}, \mathit{y}, \mathit{z}, x = y \vdash y = z \to x = z}
          }
          {\mathit{x}, \mathit{y}, \mathit{z} \vdash x = y \to y = z \to x = z}
        }
        {\mathit{x}, \mathit{y} \vdash \forall \mathit{z}. x = y \to y = z \to x = z}
      }
      {\mathit{x} \vdash \forall \mathit{y}. \forall \mathit{z}. x = y \to y = z \to x = z}
    }
    { \vdash \forall \mathit{x}. \forall \mathit{y}. \forall \mathit{z}. x = y \to y = z \to x = z}
  \end{displaymath}}
\end{frame}

\begin{frame}
  {Rewriting}

  \begin{enumerate}
  \item ~\TirName{Subst} can be quite tricky to use because we have to
    give a formula $P$ such that $P[x \mapsto t_1]$ is the one we
    start with, and $P[x \mapsto t_2]$ is the one we end up with.
  \item Usually, we want to replace \emph{every} occurrence of $t_1$
    with $t_2$. We write this as:
    \begin{displaymath}
      P\{t_1 \mapsto t_2\}
    \end{displaymath}
  \end{enumerate}
\end{frame}

\begin{frame}
  {Rewriting}

  \begin{displaymath}
    \inferrule* [right=Rewrite$\to$]
    {\Gamma \vdash P\{t_1 \mapsto t_2\}}
    {\Gamma~[t_1 = t_2] \vdash P}
  \end{displaymath}

  \bigskip

  If we have $t_1 = t_2$ then we can replace $t_1$ with $t_2$
  everywhere.
\end{frame}

\begin{frame}
  {Rewriting}

  For convenience:
  \begin{displaymath}
    \inferrule* [right=Rewrite$\leftarrow$]
    {\Gamma \vdash P\{t_2 \mapsto t_1\}}
    {\Gamma~[t_1 = t_2] \vdash P}
  \end{displaymath}

  \bigskip

  If we have $t_1 = t_2$ then we can replace $t_1$ with $t_2$
  everywhere.
\end{frame}

\begin{frame}
  {Summary}

  Equality is characterised by two principles:
  \begin{enumerate}
  \item Everything is equal to itself (\emph{reflexivity})
  \item If $s = t$, then everything that is true about $s$ is true
    about $t$.
  \end{enumerate}
\end{frame}

\weeksection{Arithmetic and Induction}

\begin{frame}
  {Arithmetic}

  One thing we might want to do with Predicate Logic is reason about
  numbers:

  \bigskip

  \begin{itemize}
  \item $\forall x. \forall y.~x + y = y + x$
  \item $\forall x. \forall y.\forall z.~x + (y + z) = (x+y) + z$
  \item $\forall x. \forall y. \forall z.~x \times (y+z) = (x \times y) + (x \times z)$
  \item $\forall n.~n > 2 \to \lnot(\exists x. \exists y. \exists z.~x^n + y^n = z^n)$
  \end{itemize}
\end{frame}

\begin{frame}
  {Representation of Numbers}

  To make thing easier, we use a \emph{unary} representation of numbers:

  \bigskip

  A number is either:
  \begin{enumerate}
  \item $0$ or
  \item $S(n)$, where $n$ is a number.
  \end{enumerate}

  \bigskip

  For example, $5$ is represented as $S(S(S(S(S(0)))))$.

  \bigskip

  This is \emph{massively} inefficient, but makes reasoning easier.
\end{frame}

\begin{frame}
  {Axioms}

  We can now write down some plausible axioms for arithmetic.
\end{frame}

\begin{frame}
  {Axiom 1}

  \begin{displaymath}
    \forall x.~\lnot(0 = S(x))
  \end{displaymath}

  \bigskip

  $0$ is not the successor of any number.
\end{frame}

\begin{frame}
  {Axiom 2}

  \begin{displaymath}
    \forall x. \forall y. S(x) = S(y) \to x = y
  \end{displaymath}

  \bigskip

  If two successors are equal, the things they are successors of must
  be equal.

  \bigskip

  This is a way of saying ``successor goes on forever''. If we had a
  loop such that $S(x)$ was equal to some number $y$ less than $x$,
  then this axiom would not hold.
\end{frame}

\begin{frame}
  {Axioms 3 and 4}

  \begin{displaymath}
    \begin{array}{l}
      \forall x.~add(0,x) = x \\
      \forall x. \forall y.~add(S(x),y) = S(add(x,y))
    \end{array}
  \end{displaymath}

  \bigskip

  \begin{enumerate}
  \item Adding $0$ to $x$ gives $x$, i.e., $0+x=x$
  \item $(1+x)+y = 1+(x+y)$.
  \end{enumerate}

  \bigskip

  These axioms define addition.
\end{frame}

\begin{frame}
  {Axioms 5 and 6}

  \begin{displaymath}
    \begin{array}{l}
      \forall x.~mul(0,x) = 0 \\
      \forall x. \forall y.~mul(S(x),y) = add(y,mul(x,y))
    \end{array}
  \end{displaymath}

  \bigskip

  \begin{enumerate}
  \item $0 \times x = 0$
  \item $(1+x) \times y = y + (x \times y)$
  \end{enumerate}

  \bigskip

  These axioms define multiplication.
\end{frame}

\begin{frame}
  {Peano's Axioms (not all of them)}

  \begin{enumerate}
  \item $\forall x.~\lnot(0 = S(x))$
  \item $\forall x. \forall y.~S(x) = S(y) \to x = y$
  \item $\forall x.~add(0,x) = x$
  \item $\forall x.\forall y.~add(S(x),y) = S(add(x,y))$
  \item $\forall x.~mul(0,x) = 0$
  \item $\forall x. \forall y.~mul(S(x),y) = add(y,mul(x,y))$
  \end{enumerate}

  \bigskip

  (named after Guiseppe Peano)
\end{frame}

\begin{frame}
  {What can we prove?}

  Can do computation:
  \begin{displaymath}
    add(S(S(0)),S(S(S(0)))) = S(S(S(S(S(0)))))
  \end{displaymath}
  \qquad \qquad \textcolor{black!60}{(2+3=5)}

  \bigskip

  But can't prove anything for all numbers:
  \begin{displaymath}
    \forall x. \forall y. x + y = y + x
  \end{displaymath}
  is not provable yet.
\end{frame}

\begin{frame}
  {Induction}

  To prove things for all numbers, we use the principle of \emph{induction}:

  \bigskip

  To prove for all $x$, $P(x)$, we must:
  \begin{itemize}
  \item Prove $P(0)$
  \item For all $n$, prove that $P(n)$ implies $P(S(n))$
  \end{itemize}
  \textcolor{black!60}{(this is the main reason we use the unary representation)}

  \bigskip

  The assumption $P(n)$ in part 2 is called the \emph{induction hypothesis}.
\end{frame}

\begin{frame}
  {Example (informal)}

  To prove $add(x,0) = x$ by induction on $x$:
  \begin{enumerate}
  \item Need to prove that $add(0,0) = 0$, which is axiom 3.
  \item Need to prove if $add(n,0) = n$, then $add(S(n),0) = S(n)$:
    \begin{enumerate}
    \item We have $add(S(n),0) = S(add(n,0))$ by axiom 4; and
    \item so $S(add(n,0)) = S(n)$ by the induction hypothesis
    \end{enumerate}
  \end{enumerate}
\end{frame}

\begin{frame}
  {Induction as a Rule}

  \begin{displaymath}
    \inferrule* [right=Induction]
    {\Gamma \vdash P[x \mapsto 0] \\
      \Gamma, x_1, P[x \mapsto x_1] \vdash P[x \mapsto S(x_1)]}
    {\Gamma \vdash P}
  \end{displaymath}
  where $x$ is declared in $\Gamma$.
\end{frame}

\begin{frame}
  {Peano's Axioms}

  \begin{enumerate}
  \item $\forall x.~\lnot(0 = S(x))$
  \item $\forall x. \forall y.~S(x) = S(y) \to x = y$
  \item $\forall x.~add(0,x) = x$
  \item $\forall x.\forall y.~add(S(x),y) = S(add(x,y))$
  \item $\forall x.~mul(0,x) = 0$
  \item $\forall x. \forall y.~mul(S(x),y) = add(y,mul(x,y))$
  \end{enumerate}
  $+$ induction
\end{frame}

\begin{frame}
  {Summary}

  We can reason about arithmetic using Peano's Axioms.
  \begin{itemize}
  \item 6 axioms defining $0$, $S$, $add$ and $mul$
  \item and induction
  \end{itemize}
  This system is surprisingly powerful.

  \pause
  \bigskip

  In fact, it is \emph{too} powerful, as we shall see next week.
\end{frame}


\end{document}
