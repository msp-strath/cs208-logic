% -*- TeX-engine: xetex -*-

\documentclass[xetex,aspectratio=169,14pt,hyperref={pdfpagelabels=true,pdflang={en-GB}}]{beamer}

\usepackage[sci,noslidestrathidentity]{strathclyde}
\strathsetidentity{Department of}{Computer \& Information Sciences}

\usepackage{lmodern}
\usepackage{subscript}
\usepackage{url}
\usepackage{pifont}
\usepackage{csquotes}
\usepackage{mathpartir}
\usepackage{stmaryrd}
\usepackage{multirow}
\usepackage{euler}
\usepackage[normalem]{ulem}

% 2013-09-08 SU adding xetex specifics
% http://www.woggie.net/2008/07/16/beamer-pdftex-and-xetex/
\usepackage{xltxtra}

% 2013-09-08 SU http://robjhyndman.com/hyndsight/xelatex/
\defaultfontfeatures{Ligatures=TeX}

% 2015-12-09 SU table beautified
\renewcommand{\arraystretch}{1.2}
\usepackage{booktabs}

%MGK compatibility
\newcommand{\hh}[1]
  {\medskip\textbf{\large #1}}
\newcommand{\pdu}[3]
  {#1\rightarrow #2 : &\ & \makebox[70mm][l]{$#3$}}



\setlength{\marginparwidth}{2cm}
\usepackage{todonotes}%[disable]
\let\OldTodo\todo
\renewcommand{\todo}{\OldTodo[inline]}%
\newcommand{\todolater}[1]{}% Things to do for next year


\DeclareTextCommandDefault{\nobreakspace}{\leavevmode\nobreak\ }
\usepackage{rotating}

\usepackage{pdfcomment}% To add alt text for images using \pdftooltip{}

\usepackage{appendixnumberbeamer}

\newcommand{\messageframe}[1]{\begin{frame}\begin{center}\Huge #1\end{center}\end{frame}}
\newcommand{\sechead}[1]{{\bf #1} \\}
\newcommand{\examplehead}[1]{{\bf Example:} {\it\textcolor{red!90}{#1}} \\}
\newcommand{\eqnote}[1]{\hspace{3cm}\textit{#1}}
\newcommand{\sidenote}[1]{\qquad {\footnotesize \textcolor{black!60}{(#1)}}}
\newcommand{\sem}[1]{\llbracket #1 \rrbracket}
\newcommand{\true}{\mathsf{T}}
\newcommand{\false}{\mathsf{F}}

\def\strikeafter<#1>#2{\temporal<#1>{#2}{\sout{#2}}{\sout{#2}}}


%\newcommand{\rhighlight}{\textcolor{titlered}}
\newcommand{\rhighlight}{\textbf}
\newcommand{\highlight}{\textbf}

% \setmainfont{Linux Biolinum O}
\setmainfont{LinBiolinum}[
Path=,
UprightFont = *_R.otf ,
BoldFont = *_RB.otf ,
ItalicFont = *_RI.otf
]

\setbeamertemplate{navigation symbols}{}
%\usecolortheme[rgb={0.8,0,0}]{structure}
\usefonttheme{serif}
\usefonttheme{structurebold}
\setbeamercolor{description item}{fg=black}


\author[Atkey]{Dr.~Robert Atkey}
\institute[Strathclyde]{Computer \& Information Sciences}
\date[]{}

\newcommand{\weeksection}[1]{%
  \section{\thetitle{}, Part~\thesection : #1}
  \begin{frame}
    \begin{center}
      \textcolor{black!60}{\thetitle{}, Part \thesection}\\
      {\Huge #1}
    \end{center}
  \end{frame}}

\newcommand{\weektitle}[2]{\def\thetitle{#2}
\title[CS208 - Topic #1]{CS208 (Semester 1) Topic #1 : #2}}

\newcommand{\assigned}{:}

\newcommand{\forcedto}{\assigned_f}
\newcommand{\decideto}{\assigned_d}


\weektitle{2}{Proof for Propositional Logic}

\begin{document}

\frame{\titlepage}

\begin{frame}
  \includegraphics[width=1.0\textwidth]{yard.jpg}
\end{frame}

\weeksection{Natural Deduction}

\begin{frame}[t]
  {Judgements}

  We want to deduce \emph{judgements} of the form:
  \begin{displaymath}
    P_1, \dots, P_n \vdash Q
  \end{displaymath}
  meaning:
  \begin{center}
    From assumptions $P_1, \dots, P_n$, we can prove $Q$.
  \end{center}

  \bigskip

  {\bf Soundness}
  The system will be \emph{sound}, meaning:
  \begin{displaymath}
    P_1,\dots,P_n \vdash Q~\textrm{provable}~~\Rightarrow~~P_1, \dots, P_n \models Q
  \end{displaymath}

  We will make sure it is sound by checking each rule as we go. \\
  \quad \textcolor{black!60}{If all the premises are valid entailments, then so is the conclusion}
\end{frame}

\begin{frame}[t]
  {Judgements}

  The main judgement form is
  \begin{displaymath}
    P_1, \dots, P_n \vdash Q
  \end{displaymath}
  \qquad \textcolor{black!60}{With assumptions $P_1,\dots,P_n$, can prove $Q$}

  \pause
  \bigskip

  We will also use an auxiliary judgement:
  \begin{displaymath}
    P_1, \dots, P_n~[P] \vdash Q
  \end{displaymath}
  \quad \textcolor{black!60}{$\cdot$ With assumptions $P_1,\dots,P_n$, \emph{focusing on} $P$, can prove $Q$}\\
  \quad \textcolor{black!60}{$\cdot$ Also ``means'' $P_1,\dots,P_n,P \models Q$}\\
  \quad \textcolor{black!60}{$\cdot$ Having a focus is useful for organising proofs}
\end{frame}

\begin{frame}[t]
  {Judgements}

  The main judgement form is
  \begin{displaymath}
    P_1, \dots, P_n \vdash Q
  \end{displaymath}

  \medskip

  We will also use an auxiliary judgement:
  \begin{displaymath}
    P_1, \dots, P_n~[P] \vdash Q
  \end{displaymath}

  \pause
  \bigskip

  \textbf{Assumption lists}~The list of assumptions
  $P_1, \dots, P_n$ will appear often. So we
  will shorten it to $\Gamma$
  \textcolor{black!60}{$ = P_1, \dots, P_n$}.
\end{frame}

\newcommand{\pUse}[1]{\texttt{use}~#1\texttt{,}~}
\newcommand{\pDone}{\texttt{done}}
\newcommand{\pSplit}[2]{\texttt{split(}#1~\texttt{|}~#2\texttt{)}}
\newcommand{\pFst}{\texttt{first,}~}
\newcommand{\pSnd}{\texttt{second,}~}
\newcommand{\pTrue}{\texttt{true}}

%\titlecard{Basic Rules}

\begin{frame}
  {Basic Rules}

  \begin{displaymath}
    \inferrule* [right=Done]
    { }
    {\Gamma~[P] \vdash P}
  \end{displaymath}

  \bigskip
  \bigskip
  \pause

  \begin{itemize}
  \item If we have a focused assumption $P$, then we can prove $P$
  \item (Remember $\Gamma$ stands for a list of other assumptions)
  \end{itemize}
\end{frame}

\begin{frame}
  {Basic Rules}

  \begin{displaymath}
    \inferrule* [right=Use]
    {P \in \Gamma \\ \Gamma~[P] \vdash Q}
    {\Gamma \vdash Q}
  \end{displaymath}

  \bigskip
  \bigskip
  \pause

  \begin{itemize}
  \item $P \in \Gamma$ means ``$P$ is in $\Gamma$''.
  \item If we have a $P$ in our current assumptions, we can focus on it.
  \item $P \in \Gamma$ is a \emph{side condition}: it is something we
    check when we apply the rule, not another judgement to be proved.
  \end{itemize}
\end{frame}

\begin{frame}[t]
  {A first proof}

  \bigskip

  \begin{minipage}[t][2cm][b]{\textwidth}
    \begin{displaymath}
      \only<1>{A \vdash A\quad\quad}%
      \only<2>{\inferrule* [right=Use] {A~[A] \vdash A} {A \vdash A}}%
      \only<3>{\inferrule* [right=Use] {\inferrule* [Right=Done] { } {A~[A] \vdash A}} {A \vdash A}}
    \end{displaymath}
  \end{minipage}

  \bigskip

  \begin{itemize}
  \item<2-> First \TirName{Use} the $A$ assumption.
  \item<3-> Then we are \TirName{Done}.
  \end{itemize}
\end{frame}

\begin{frame}
  {Soundness}

  \begin{mathpar}
    \inferrule* [right=Done]
    { }
    {\Gamma~[P] \vdash P}

    \inferrule* [right=Use]
    {P \in \Gamma \\ \Gamma~[P] \vdash Q}
    {\Gamma \vdash Q}
  \end{mathpar}

  \pause
  \bigskip

  \TirName{Done} is sound because assuming $P$ entails $P$, and extra
  assumptions make no difference.

  \pause
  \bigskip

  \TirName{Use} is sound because if we assuming $P$ twice entails $Q$,
  then it is okay to assume it once.
\end{frame}

\begin{frame}
  {Rules for connectives}

  \bigskip

  The rule \TirName{Done} and \TirName{Use} do not mention the connectives.

  \bigskip

  In Natural Deduction, rules for connectives come in two kinds:

  \bigskip

  \begin{enumerate}
  \item {\bf Introduction} rules\\
    \textcolor{black!60}{How to \emph{construct} a proof with the connective}
  \item {\bf Elimination} rules \\
    \textcolor{black!60}{How to \emph{use} an assumption with this connective}
  \end{enumerate}

  \bigskip
  \pause

  Very rough analogy:\quad \textcolor{black!60}{but can be made very precise}
  \begin{enumerate}
  \item Introduction rules are like \emph{constructors} for building objects
  \item Elimination rules are like \emph{methods} for taking apart objects
  \end{enumerate}
\end{frame}

%\titlecard{Rules for Conjunction}

\begin{frame}
  {``And'' Introduction}


  \begin{displaymath}
    \inferrule* [right=Split]
    {\Gamma \vdash Q_1 \\ \Gamma \vdash Q_2}
    {\Gamma \vdash Q_1 \land Q_2}
  \end{displaymath}

  \bigskip
  \pause

  \begin{itemize}
  \item To prove $P_1 \land P_2$ we have to prove $P_1$ and $P_2$
  \item This rule is often called \TirName{$\land$-Introduction}
  \end{itemize}
\end{frame}

\begin{frame}
  \sechead{An example proof}

  \bigskip

  \begin{displaymath}
    \inferrule* [Right=Split]
    { \inferrule* [right=Use]
      {\inferrule* [Right=Done]
        { } {A, B~[A] \vdash A}}
      {A, B \vdash A}
      \\
      \inferrule* [Right=Use]
      {\inferrule* [Right=Done]
        { } {A, B~[B] \vdash B}
      }
      {A, B \vdash B}
    }
    {A, B \vdash A \land B}
  \end{displaymath}

  \pause
  \bigskip

  To prove $A \land B$, we \TirName{Split} into proofs of $A$ and
  $B$. \\
  In each case, we \TirName{Use} the corresponding assumption.
\end{frame}

\begin{frame}
  {``And'' Elimination}

  \begin{mathpar}
    \inferrule* [right=First]
    {\Gamma~[P_1] \vdash Q}
    {\Gamma~[P_1 \land P_2] \vdash Q}

    \inferrule* [right=Second]
    {\Gamma~[P_2] \vdash Q}
    {\Gamma~[P_1 \land P_2] \vdash Q}
  \end{mathpar}

  \bigskip
  \pause

  If we are focused on an formula $P_1 \land P_2$, we can select
  either the \TirName{First} or \TirName{Second} component to focus
  on.
\end{frame}

\begin{frame}
  \sechead{Example proof}

  \begin{displaymath}
    \inferrule* [Right=Use]
    {\inferrule* [Right=Second]
      {\inferrule* [Right=Done]
        { }
        {A \land B~[B] \vdash B}
      }
      {A \land B~[A \land B] \vdash B}
    }
    {A \land B \vdash B}
  \end{displaymath}
\end{frame}

%\titlecard{Rules for Truth}

\begin{frame}
  {``True'' Introduction}

  \bigskip

  \begin{displaymath}
    \inferrule* [right=True]
    { }
    {\Gamma \vdash \true}
  \end{displaymath}

  \bigskip
  \pause

  \begin{itemize}
  \item $\true$ is always provable.
  \end{itemize}
\end{frame}

\begin{frame}
  {``True'' Elimination}

  \pause
  \bigskip

  \begin{center}
    \emph{No elimination rule!}
  \end{center}
\end{frame}

\begin{frame}
  {Summary}

  \begin{itemize}
  \item The judgement forms for (focused) Natural Deduction:
    \begin{mathpar}
      P_1, \dots, P_n \vdash Q

      P_1, \dots, P_n~[P] \vdash Q
    \end{mathpar}
  \item Rules for \TirName{Use} and \TirName{Done}
  \item Rules for introducing and eliminating $\land$.
  \end{itemize}
\end{frame}

\weeksection{Rules for ``Implies''}

\begin{frame}
  {``Implies'' Introduction}

  \begin{displaymath}
    \inferrule* [right=Introduce]
    {\Gamma, P \vdash Q}
    {\Gamma \vdash P \to Q}
  \end{displaymath}

  \pause
  \bigskip

  To prove $P \to Q$, we prove $Q$ under the assumption $P$.
\end{frame}

\begin{frame}
  {Example: $A \to A$}

  \begin{displaymath}
    \inferrule* [right=Introduce]
    {\inferrule* [Right=Use]
      {\inferrule* [Right=Done] { } {A~[A] \vdash A}}
      {A \vdash A}
    }
    {\vdash A \to A}
  \end{displaymath}

\end{frame}

\begin{frame}
  {Example : $(A \land B) \to A$}

  \begin{displaymath}
    \inferrule* [right=Introduce]
    {\inferrule* [Right=Use]
      {\inferrule* [Right=First]
        {\inferrule* [Right=Done]
          { }
          {\mathrm{A} \land \mathrm{B}~[\mathrm{A}] \vdash \mathrm{A}}
        }
        {\mathrm{A} \land \mathrm{B}~[\mathrm{A} \land \mathrm{B}] \vdash \mathrm{A}}
      }
      {\mathrm{A} \land \mathrm{B} \vdash \mathrm{A}}
    }
    { \vdash (\mathrm{A} \land \mathrm{B}) \to \mathrm{A}}
  \end{displaymath}
\end{frame}

\begin{frame}
  {``Implies'' Elimination}

  \begin{displaymath}
    \inferrule* [right=Apply]
    {\Gamma \vdash P_1 \\ \Gamma~[P_2] \vdash Q}
    {\Gamma~[P_1 \to P_2] \vdash Q}
  \end{displaymath}

  \bigskip
  \pause

  If we have $P_1 \to P_2$ and we can prove $P_1$, then we have $P_2$.
\end{frame}

\begin{frame}
  {Example: $A \to (A \to B) \to B$}

  \begin{displaymath}
    \inferrule* [right=Introduce]
    {\inferrule* [Right=Introduce]
      {\inferrule* [Right=Use]
        {\inferrule* [Right=Apply]
          {\inferrule* [right=Use]
            {\inferrule* [Right=Done]
              { }
              {\mathrm{A}, \mathrm{A} \to \mathrm{B}~[\mathrm{A}] \vdash \mathrm{A}}
            }
            {\mathrm{A}, \mathrm{A} \to \mathrm{B} \vdash \mathrm{A}}
            \\
            \inferrule* [Right=Done]
            { }
            {\mathrm{A}, \mathrm{A} \to \mathrm{B}~[\mathrm{B}] \vdash \mathrm{B}}
          }
          {\mathrm{A}, \mathrm{A} \to \mathrm{B}~[\mathrm{A} \to \mathrm{B}] \vdash \mathrm{B}}
        }
        {\mathrm{A}, \mathrm{A} \to \mathrm{B} \vdash \mathrm{B}}
      }
      {\mathrm{A} \vdash (\mathrm{A} \to \mathrm{B}) \to \mathrm{B}}
    }
    { \vdash \mathrm{A} \to (\mathrm{A} \to \mathrm{B}) \to \mathrm{B}}
  \end{displaymath}
\end{frame}


% \begin{frame}

%   Examples (see scans):
%   \begin{enumerate}
%   \item $\vdash A \to (A \lor B)$
%   \item $\vdash A \to B \to (A \land B)$
%   \end{enumerate}
% \end{frame}

% \begin{frame}
%   Examples (see scans):
%   \begin{enumerate}
%   \item $A \to B, A \vdash B$
%   \item $A \lor B, B \to C \vdash A \lor C$
%   \item $A \to B \to C \vdash (A \land B) \to C$
%   \end{enumerate}
% \end{frame}

\begin{frame}
  {The Rules so far}

  \begin{mathpar}
    \inferrule* [right=Done]
    { }
    {\Gamma~[P] \vdash P}

    \inferrule* [right=Use]
    {P \in \Gamma \\ \Gamma~[P] \vdash Q}
    {\Gamma \vdash Q}

    \inferrule* [right=Split]
    {\Gamma \vdash Q_1 ~~~~~~ \Gamma \vdash Q_2}
    {\Gamma \vdash Q_1 \land Q_2}
    \quad
    \inferrule* [right=First]
    {\Gamma~[P_1] \vdash Q}
    {\Gamma~[P_1 \land P_2] \vdash Q}
    ~
    \inferrule* [right=Second]
    {\Gamma~[P_2] \vdash Q}
    {\Gamma~[P_1 \land P_2] \vdash Q}

    \inferrule* [right=Introduce]
    {\Gamma, P \vdash Q}
    {\Gamma \vdash P \to Q}

    \inferrule* [right=Apply]
    {\Gamma \vdash P_1 \\ \Gamma~[P_2] \vdash Q}
    {\Gamma~[P_1 \to P_2] \vdash Q}
  \end{mathpar}
\end{frame}

\begin{frame}
  {Summary}

  \begin{itemize}
  \item The rules for Implication
    \begin{mathpar}
      \inferrule* [right=Introduce]
      {\Gamma, P \vdash Q}
      {\Gamma \vdash P \to Q}

      \inferrule* [right=Apply]
      {\Gamma \vdash P_1 \\ \Gamma~[P_2] \vdash Q}
      {\Gamma~[P_1 \to P_2] \vdash Q}
    \end{mathpar}
  \end{itemize}
\end{frame}

\weeksection{Rules for ``Or''}

\begin{frame}
  {``Or'' Introduction}

  \bigskip

  \begin{mathpar}
    \inferrule* [right=Left]
    {\Gamma \vdash Q_1}
    {\Gamma \vdash Q_1 \lor Q_2}

    \inferrule* [right=Right]
    {\Gamma \vdash Q_2}
    {\Gamma \vdash Q_1 \lor Q_2}
  \end{mathpar}

  \bigskip
  \pause

  To prove $Q_1 \lor Q_2$, \emph{either} we:
  \begin{enumerate}
  \item prove $Q_1$, \emph{or}
  \item prove $Q_2$.
  \end{enumerate}
\end{frame}

\begin{frame}
  \sechead{Example}

  \bigskip

  \begin{displaymath}
    \inferrule* [Right=Left]
    {\inferrule* [Right=Use]
      {\inferrule* [Right=Done]
        { } {A~[A] \vdash A}
      }
      {A \vdash A}
    }
    {A \vdash A \lor B}
  \end{displaymath}
\end{frame}

\begin{frame}[t]
  {``Or'' Elimination}

  \begin{displaymath}
    \inferrule* [right=Cases]
    {\Gamma, P_1 \vdash Q \\ \Gamma, P_2 \vdash Q}
    {\Gamma~[P_1 \lor P_2] \vdash Q}
  \end{displaymath}

  \textcolor{black!60}{$\Gamma, P$ means all the assumptions in $\Gamma$, and $P$}

  \bigskip
  \pause

  If we are focused on $P_1 \lor P_2$, then:
  \begin{enumerate}
  \item Either $P_1$ holds, so we have to prove $Q$ assuming $P_1$; or
  \item Either $P_2$ holds, so we have to prove $Q$ assuming $P_2$
  \end{enumerate}
\end{frame}

\begin{frame}[t]
  {``Or'' Elimination}

  \begin{displaymath}
    \inferrule* [right=Cases]
    {\Gamma, P_1 \vdash Q \\ \Gamma, P_2 \vdash Q}
    {\Gamma~[P_1 \lor P_2] \vdash Q}
  \end{displaymath}

  \pause
  \bigskip

  \emph{We} (the provers) don't know which of $P_1$ or $P_2$ is true,
  so we need to write proofs for both eventualities.

  \pause
  \bigskip

  This is dual to the case for conjunction: for $P_1 \land P_2$
  \emph{we} had to provide both sides in the introduction rule, but
  got to choose in the elimination rule.
\end{frame}

\begin{frame}
  \sechead{Example}

  \begin{displaymath}
    \inferrule* [right=Use]
    {\inferrule* [Right=Cases]
      {\inferrule* [right=Right]
        {\inferrule* [Right=Use]
          {\inferrule* [Right=Done] { } {A \lor B, A~[A] \vdash A}}
          {A \lor B, A \vdash A}
        }
        {A \lor B, A \vdash B \lor A}
        \\
        \inferrule* [Right=Left]
        {\inferrule* [Right=Use]
          {\inferrule* [Right=Done] { } {A \lor B, B~[B] \vdash B}}
          {A \lor B, B \vdash B}
        }
        {A \lor B, B \vdash B \lor A}
      }
      {A \lor B~[A \lor B] \vdash B \lor A}
    }
    {A \lor B \vdash B \lor A}
  \end{displaymath}
\end{frame}

\begin{frame}
  {``False'' Introduction}

  \bigskip

  \begin{center}
    \emph{No introduction rule!}
  \end{center}
\end{frame}

\begin{frame}
  {``False'' Elimination}

  \bigskip

  \begin{displaymath}
    \inferrule* [right=False]
    { }
    {\Gamma~[\false] \vdash Q}
  \end{displaymath}

  \bigskip
  \pause

  If we have a false assumption, we can prove anything.
\end{frame}

\begin{frame}
  \sechead{Example}

  \begin{displaymath}
    \inferrule* [right=Introduce]
    {\inferrule* [Right=Use]
      {\inferrule* [Right=False]
        { }
        {\mathsf{F}~[\mathsf{F}] \vdash \mathrm{A} \land \mathrm{B} \land \mathrm{C}}
      }
      {\mathsf{F} \vdash \mathrm{A} \land \mathrm{B} \land \mathrm{C}}
    }
    { \vdash \mathsf{F} \to (\mathrm{A} \land \mathrm{B} \land \mathrm{C})}
  \end{displaymath}
\end{frame}

\begin{frame}
  \sechead{Example}

  \begin{displaymath}
    \inferrule* [right=Introduce]
    {\inferrule* [Right=Use]
      {\inferrule* [Right=Cases]
        {\inferrule* [right=Use]
          {\inferrule* [Right=Done]
            { }
            {\mathrm{A} \lor \mathsf{F}, \mathrm{A}~[\mathrm{A}] \vdash \mathrm{A}}
          }
          {\mathrm{A} \lor \mathsf{F}, \mathrm{A} \vdash \mathrm{A}}
          \\
          \inferrule* [Right=Use]
          {\inferrule* [Right=False]
            { }
            {\mathrm{A} \lor \mathsf{F}, \mathsf{F}~[\mathsf{F}] \vdash \mathrm{A}}
          }
          {\mathrm{A} \lor \mathsf{F}, \mathsf{F} \vdash \mathrm{A}}
        }
        {\mathrm{A} \lor \mathsf{F}~[\mathrm{A} \lor \mathsf{F}] \vdash \mathrm{A}}
      }
      {\mathrm{A} \lor \mathsf{F} \vdash \mathrm{A}}
    }
    { \vdash (\mathrm{A} \lor \mathsf{F}) \to \mathrm{A}}
  \end{displaymath}
\end{frame}

\begin{frame}
  {Summary}

  \begin{itemize}
  \item Rules for ``Or'':
    \begin{mathpar}
      \inferrule* [right=Left]
      {\Gamma \vdash Q_1}
      {\Gamma \vdash Q_1 \lor Q_2}

      \inferrule* [right=Right]
      {\Gamma \vdash Q_2}
      {\Gamma \vdash Q_1 \lor Q_2}

      \inferrule* [right=Cases]
      {\Gamma, P_1 \vdash Q \\ \Gamma, P_2 \vdash Q}
      {\Gamma~[P_1 \lor P_2] \vdash Q}
    \end{mathpar}
  \item ``False'' lets us prove anything:
    \begin{displaymath}
      \inferrule* [right=False]
      { }
      {\Gamma~[\false] \vdash Q}
    \end{displaymath}
  \end{itemize}
\end{frame}

\weeksection{Rules for ``Not''}

\begin{frame}
  {Negation}

  We could \emph{define} negation:
  \begin{displaymath}
    \lnot P \equiv P \to \false
  \end{displaymath}

  \bigskip

  Then we wouldn't need any rules for it.
\end{frame}

\begin{frame}
  {Rules for Negation: Introduction}

  \textcolor{black!60}{($\lnot P \equiv P \to \false$)}

  \medskip

  \begin{displaymath}
    \inferrule* [right=Introduce]
    {\Gamma, P \vdash \false}
    {\Gamma \vdash P \to \false}
  \end{displaymath}

  \bigskip

  To prove $\lnot P$, we prove that $P$ proves false.
\end{frame}

\begin{frame}
  {Rules for Negation: Elimination}

  \textcolor{black!60}{($\lnot P \equiv P \to \false$)}

  \medskip

  \begin{displaymath}
    \inferrule* [right=Apply]
    {\Gamma \vdash P \\ \inferrule* [Right=False] { } {\Gamma~[\false] \vdash Q}}
    {\Gamma~[P \to \false] \vdash Q}
  \end{displaymath}

  \bigskip

  If we know that $\lnot P$ is true, and we can prove $P$, then we get
  a contradiction which allows us to prove anything.
\end{frame}

\begin{frame}
  {Specialised Rules for Negation}

  Introduction:
  \begin{displaymath}
    \inferrule* [right=Not-Intro]
    {\Gamma, P \vdash \false}
    {\Gamma \vdash \lnot P}
  \end{displaymath}

  \bigskip

  Elimination:
  \begin{displaymath}
    \inferrule* [right=Not-Elim]
    {\Gamma \vdash P}
    {\Gamma~[\lnot P] \vdash Q}
  \end{displaymath}
\end{frame}

\begin{frame}
  {Example: $(A \lor B) \to \lnot A \to B$}

  {\footnotesize
  \begin{displaymath}
    \inferrule* [right=Introduce]
    {\inferrule* [Right=Introduce]
      {\inferrule* [Right=Use]
        {\inferrule* [Right=Cases]
          {\inferrule* [right=Use]
            {\inferrule* [Right=¬-Elim]
              {\inferrule* [Right=Use]
                {\inferrule* [Right=Done]
                  { }
                  {\mathrm{A} \lor \mathrm{B}, \lnot \mathrm{A}, \mathrm{A}~[\mathrm{A}] \vdash \mathrm{A}}
                }
                {\mathrm{A} \lor \mathrm{B}, \lnot \mathrm{A}, \mathrm{A} \vdash \mathrm{A}}
              }
              {\mathrm{A} \lor \mathrm{B}, \lnot \mathrm{A}, \mathrm{A}~[\lnot \mathrm{A}] \vdash \mathrm{B}}
            }
            {\mathrm{A} \lor \mathrm{B}, \lnot \mathrm{A}, \mathrm{A} \vdash \mathrm{B}}
            \\
            \inferrule* [Right=Use]
            {\inferrule* [Right=Done]
              { }
              {\mathrm{A} \lor \mathrm{B}, \lnot \mathrm{A}, \mathrm{B}~[\mathrm{B}] \vdash \mathrm{B}}
            }
            {\mathrm{A} \lor \mathrm{B}, \lnot \mathrm{A}, \mathrm{B} \vdash \mathrm{B}}
          }
          {\mathrm{A} \lor \mathrm{B}, \lnot \mathrm{A}~[\mathrm{A} \lor \mathrm{B}] \vdash \mathrm{B}}
        }
        {\mathrm{A} \lor \mathrm{B}, \lnot \mathrm{A} \vdash \mathrm{B}}
      }
      {\mathrm{A} \lor \mathrm{B} \vdash \lnot \mathrm{A} \to \mathrm{B}}
    }
    { \vdash (\mathrm{A} \lor \mathrm{B}) \to \lnot \mathrm{A} \to \mathrm{B}}
  \end{displaymath}}
\end{frame}

\begin{frame}
  {Summary}

  \begin{itemize}
  \item Negation can be defined in terms of Implication and False
  \item Nicer to have specific rules:
    \begin{mathpar}
      \inferrule*
      {\Gamma, P \vdash \false}
      {\Gamma \vdash \lnot P}

      \inferrule*
      {\Gamma \vdash P}
      {\Gamma~[\lnot P] \vdash Q}
    \end{mathpar}
  \end{itemize}
\end{frame}

\weeksection{Soundness \& Completeness \& Philosophy}
% Subtitle: And a little bit of philosophy

\begin{frame}
  {Soundness and Completeness}

  \textbf{Soundness} : ``Everything that is provable is valid'':
  \begin{displaymath}
    P_1, \dots, P_n \vdash Q \quad \Rightarrow P_1, \dots, P_n \models Q
  \end{displaymath}
  I've tried, informally, to convince you of this for each rule. If
  each rule is sound, then the whole system is sound.

  \bigskip
  \pause

  \textbf{Completeness} : ``Everything that is provable is valid'':
  \begin{displaymath}
    P_1, \dots, P_n \models Q \quad \Rightarrow P_1, \dots, P_n \vdash Q
  \end{displaymath}
  Does this property hold of the system so far?
\end{frame}

\begin{frame}
  {Failure of Completeness}
  Recall that this entailment is valid:
  \begin{displaymath}
    \models A \lor \lnot A
  \end{displaymath}

  \bigskip

  Can we prove this? \pause Is there a proof of
  $\vdash A \lor \lnot A$? \\ \pause
  Have three options:
  \begin{enumerate}
  \item Apply \TirName{Use} to use an assumption. \pause\emph{No assumptions!}\pause
  \item Apply \TirName{Left} and try to prove $\vdash A$, \pause\emph{but
      this can't be provable, by soundness!}\pause
  \item Apply \TirName{Right} and try to prove $\vdash \lnot A$,
    \pause\emph{but this can't be provable, by soundness!}\pause
  \end{enumerate}
  So the system so far is {\bf not} complete, with respect to our
  semantics.
\end{frame}

\begin{frame}
  {Fixing completeness}
  We could add the following rule:
  \begin{displaymath}
    \inferrule* [right=ExcludedMiddle]
    {\Gamma, P \vdash Q \\ \Gamma, \lnot P \vdash Q}
    {\Gamma \vdash Q}
  \end{displaymath}

  \pause
  \bigskip

  To prove $Q$, pick any proposition $P$ and say ``either $P$
  or $\lnot P$''.

  \bigskip
  \pause

  This lets us prove $\vdash A \lor \lnot A$.

  \bigskip
  \pause

  It is \emph{sound}, but is it a good idea?
\end{frame}

\begin{frame}
  {Some Philosophy of Mathematics}
  \emph{Where do mathematical objects live?}\\
  \sidenote{objects include numbers, shapes, functions, propositions, proofs, ...}
\end{frame}

\begin{frame}
  {``Platonism''}

  \begin{center}
    \includegraphics[width=1.5cm]{800px-Plato_Silanion_Musei_Capitolini_MC1377.jpg}
  \end{center}

  \begin{itemize}
  \item Objects exist ``out there'', independently of us.
  \item There is a universal notion of ``truth''.
    \begin{itemize}
    \item Every proposition is either true or false, even if \emph{we}
      can't see why.
    \end{itemize}
  \end{itemize}

  {\tiny
    \emph{Image: }By Copy of Silanion, Public Domain, \url{https://commons.wikimedia.org/w/index.php?curid=7831217}}
\end{frame}

\begin{frame}
  {``Intuitionism''}

  \begin{center}
    \includegraphics[width=1.5cm]{Luitzen_Egbertus_Jan_Brouwer.jpeg}
  \end{center}

  \sidenote{L.E.J. Brouwer, ~1900/10/20s}
  \begin{itemize}
  \item Objects exist as constructions within our heads.
  \item Including proofs of propositions
    \begin{itemize}
    \item We convince ourselves of the truth of a proposition by
      constructing evidence for it.
    \end{itemize}
  \end{itemize}

  {\tiny
  \emph{Image: }By Source (WP:NFCC\#4), Fair use, \url{https://en.wikipedia.org/w/index.php?curid=39567913}}
\end{frame}

\begin{frame}
  {Evidence based Interpretation}

  \sidenote{Instead of saying $P \mathop\Box Q$ is true when...}
  \begin{displaymath}
    \begin{array}{cl}
      \textrm{Evidence of...}&\textrm{is}\\
      \hline
      \true &\textrm{there always evidence of $\true$} \\
      \false&\textrm{there is no evidence of $\false$} \\
      P \land Q&\textrm{evidence of $P$ and evidence of $Q$} \\
      P \lor Q &\textrm{evidence of $P$ or  evidence of $Q$} \\
      P \to Q  &\textrm{a process converting evidence of $P$ into evidence of $Q$}
    \end{array}
  \end{displaymath}
\end{frame}

\begin{frame}
  {Evidence for Negation}

  We define $\lnot P = P \to \false$.
  \begin{itemize}
  \item evidence of $\lnot P$ is a process converting evidence of $P$ to evidence of $\false$
  \item but there is no evidence of $\false$
  \item so there can be no evidence of $P$.
  \end{itemize}
\end{frame}

\begin{frame}
  {Excluded Middle?}

  In two valued ($\true, \false$) logic, \emph{excluded middle} is valid for any $P$:
  \begin{displaymath}
    P \lor \lnot P
  \end{displaymath}
  The proof of validity (via truth tables) makes no commitment to
  which one is actually true.

  \pause
  \bigskip

  However, in terms of evidence, we have to \emph{construct} either
  \begin{enumerate}
  \item evidence of $P$, or
  \item evidence of $\lnot P$.
  \end{enumerate}
  For an arbitrary proposition $P$, this seems unlikely.
\end{frame}

\begin{frame}
  {\textcolor{red}{Failure} of Excluded Middle}

  For instance, if $x$ is a real number (has an arbitrarily long
  decimal expansion), then, in terms of evidence
  \begin{displaymath}
    (x = 0) \lor \lnot (x = 0)
  \end{displaymath}
  asks us to determine whether $x$ is $0$.

  \medskip

  But there is no process to do this in finite time.\\
  \sidenote{Another example: does this Turing Machine halt?}
\end{frame}

\begin{frame}
  {Intuitionistic Logic}

  Intuitionistic Logic is the similar to ``Classical'' Logic, except
  that it does not include the Law of Excluded Middle $P \lor \lnot P$
  for all propositions $P$.

  \medskip

  \rhighlight{Note:} this does not mean that $\lnot (P \lor \lnot P)$ is
  provable. There may be some $P$s for which $P \lor \lnot P$ holds.\\
  \sidenote{For example, $(x = 0) \lor \lnot (x=0)$ when $x$ is an integer}
\end{frame}

\begin{frame}
  {Summary}

  \begin{itemize}
  \item The system was have so far is \emph{sound} but not \emph{complete}
  \item We can make it complete by adding a rule for \emph{excluded middle}:
    \begin{displaymath}
      P \lor \lnot P
    \end{displaymath}
  \item But should we? What does Logic mean?
  \end{itemize}
\end{frame}

\end{document}
