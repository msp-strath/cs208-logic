% -*- TeX-engine: xetex -*-

\documentclass[xetex,aspectratio=169,14pt,hyperref={pdfpagelabels=true,pdflang={en-GB}}]{beamer}

\usepackage[sci,noslidestrathidentity]{strathclyde}
\strathsetidentity{Department of}{Computer \& Information Sciences}

\usepackage{lmodern}
\usepackage{subscript}
\usepackage{url}
\usepackage{pifont}
\usepackage{csquotes}
\usepackage{mathpartir}
\usepackage{stmaryrd}
\usepackage{multirow}
\usepackage{euler}
\usepackage[normalem]{ulem}

% 2013-09-08 SU adding xetex specifics
% http://www.woggie.net/2008/07/16/beamer-pdftex-and-xetex/
\usepackage{xltxtra}

% 2013-09-08 SU http://robjhyndman.com/hyndsight/xelatex/
\defaultfontfeatures{Ligatures=TeX}

% 2015-12-09 SU table beautified
\renewcommand{\arraystretch}{1.2}
\usepackage{booktabs}

%MGK compatibility
\newcommand{\hh}[1]
  {\medskip\textbf{\large #1}}
\newcommand{\pdu}[3]
  {#1\rightarrow #2 : &\ & \makebox[70mm][l]{$#3$}}



\setlength{\marginparwidth}{2cm}
\usepackage{todonotes}%[disable]
\let\OldTodo\todo
\renewcommand{\todo}{\OldTodo[inline]}%
\newcommand{\todolater}[1]{}% Things to do for next year


\DeclareTextCommandDefault{\nobreakspace}{\leavevmode\nobreak\ }
\usepackage{rotating}

\usepackage{pdfcomment}% To add alt text for images using \pdftooltip{}

\usepackage{appendixnumberbeamer}

\newcommand{\messageframe}[1]{\begin{frame}\begin{center}\Huge #1\end{center}\end{frame}}
\newcommand{\sechead}[1]{{\bf #1} \\}
\newcommand{\examplehead}[1]{{\bf Example:} {\it\textcolor{red!90}{#1}} \\}
\newcommand{\eqnote}[1]{\hspace{3cm}\textit{#1}}
\newcommand{\sidenote}[1]{\qquad {\footnotesize \textcolor{black!60}{(#1)}}}
\newcommand{\sem}[1]{\llbracket #1 \rrbracket}
\newcommand{\true}{\mathsf{T}}
\newcommand{\false}{\mathsf{F}}

\def\strikeafter<#1>#2{\temporal<#1>{#2}{\sout{#2}}{\sout{#2}}}


%\newcommand{\rhighlight}{\textcolor{titlered}}
\newcommand{\rhighlight}{\textbf}
\newcommand{\highlight}{\textbf}

% \setmainfont{Linux Biolinum O}
\setmainfont{LinBiolinum}[
Path=,
UprightFont = *_R.otf ,
BoldFont = *_RB.otf ,
ItalicFont = *_RI.otf
]

\setbeamertemplate{navigation symbols}{}
%\usecolortheme[rgb={0.8,0,0}]{structure}
\usefonttheme{serif}
\usefonttheme{structurebold}
\setbeamercolor{description item}{fg=black}


\author[Atkey]{Dr.~Robert Atkey}
\institute[Strathclyde]{Computer \& Information Sciences}
\date[]{}

\newcommand{\weeksection}[1]{%
  \section{\thetitle{}, Part~\thesection : #1}
  \begin{frame}
    \begin{center}
      \textcolor{black!60}{\thetitle{}, Part \thesection}\\
      {\Huge #1}
    \end{center}
  \end{frame}}

\newcommand{\weektitle}[2]{\def\thetitle{#2}
\title[CS208 - Topic #1]{CS208 (Semester 1) Topic #1 : #2}}

\newcommand{\assigned}{:}

\newcommand{\forcedto}{\assigned_f}
\newcommand{\decideto}{\assigned_d}


\weektitle{3}{Predicate Logic}

\begin{document}

\frame{\titlepage}

\weeksection{Introduction}

\begin{frame}
  So far:
  \begin{center}
    \rhighlight{Propositional Logic}
  \end{center}

  \bigskip

  We can say things like:

  \medskip

  \quad  \emph{``If it is raining or sunny, and it is not sunny, then it is raining''}
  \begin{displaymath}
    ((R \lor S) \land \lnot S) \to R
  \end{displaymath}

  \bigskip

  \emph{``version 1 is installed, or version 2 is installed, or version 3 is installed''}
  \begin{displaymath}
    p_1 \lor p_2 \lor p_3
  \end{displaymath}
\end{frame}

\begin{frame}
  {What we can't say}

  \emph{``Every day is sunny or rainy, today is not sunny, so today is rainy''}
  \begin{itemize}
  \item No way to make \emph{universal} statements\sidenote{``Every day''}
  \end{itemize}

  \bigskip

  \emph{``Some version of the package is installed''} \\
  \begin{itemize}
  \item No way to make \emph{existential} statements\sidenote{``Some version''}
  \end{itemize}

  \pause
  \bigskip

  \sechead{Best we can do is list the possibilities}
  \begin{displaymath}
    (S_{\mathit{monday}} \lor R_{\mathit{monday}}) \land (S_{\mathit{tuesday}} \lor R_{\mathit{tuesday}}) \land ...
  \end{displaymath}
\end{frame}

\begin{frame}[t]
  {Universal statements}

  \bigskip

  \rhighlight{``Classical'' examples}: \qquad \sidenote{due to Aristole}

  \bigskip

  \begin{enumerate}
  \item All human are mortal
  \item Socrates is a human
  \item Therefore Socrates is mortal
  \end{enumerate}
  \sidenote{from the universal to the specific}

  \bigskip

  \begin{enumerate}
  \item No bird can fly in space
  \item Owls are birds
  \item Therefore owls cannot fly in space
  \end{enumerate}
\end{frame}

\begin{frame}[t]
  \sechead{Universal and Existential statements are common}

  \bigskip

  \rhighlight{Database queries}:

  \bigskip

  ``Does there exist a customer that has not paid their invoice?''

  \bigskip

  ``Does there exist a player who is within 10 metres of player 1?''

  \bigskip

  ``Are all players logged off?''

  \bigskip

  ``Do we have any customers?''
\end{frame}

\begin{frame}[t]
  \sechead{Universal and Existential statements are common}

  \bigskip

  The \rhighlight{semantics of Propositional Logic}:

  \bigskip

  ``$P$ is satisfiable if \emph{there exists} a valuation that makes it true.''

  \bigskip

  ``$P$ is valid if \emph{all} valuations make it true.''

  \bigskip

  ``$P$ entails $Q$ if \emph{for all} valuations, $P$ is true implies $Q$ is true.''

\end{frame}

\begin{frame}[t]
  \sechead{\highlight{Predicate Logic} upgrades Propositional Logic}

  \begin{enumerate}
  \item Add \emph{individuals}:
    \begin{itemize}
    \item Specific individuals (e.g., $\mathsf{socrates}$, $\mathsf{today}$, $\mathsf{player1}$, $1$, $2$, $3$) \\
      \sidenote{these ``name'' specific entities in the world}
    \item General individuals ($x$, $y$, $z$, ...) \\
      \sidenote{like variables in programming, they stand for ``some'' individual}
    \end{itemize}

    \bigskip

  \item Add \emph{function symbols}:
    \begin{itemize}
    \item $x + y$, $\mathsf{dayAfter}(\mathsf{today})$, $\mathsf{dayAfter}(x)$
    \end{itemize}

    \bigskip

  \item Add \emph{properties} and \emph{relations}:
    \begin{itemize}
    \item Properties: $\mathrm{canFlyInSpace}(\mathsf{owl})$, $\mathsf{paid}(i)$
    \item Relations: $x = y$, $x \leq 10$, $\mathrm{custInvoice}(c, i)$.
    \end{itemize}

    \bigskip

  \item Add \emph{Quantifiers}:
    \begin{itemize}
    \item Universal quantification: $\forall x. P$
      \sidenote{``for all'' $x$, it is the case that $P$}
    \item Existential quantification: $\exists x. P$
      \sidenote{``there exists'' $x$, such that $P$}
    \end{itemize}

  \end{enumerate}
\end{frame}

\begin{frame}
  \sechead{Layered Syntax}

  The syntax of Predicate Logic comes in two layers:

  \bigskip

  \rhighlight{Terms}~ Built from individuals and function symbols:
  \begin{mathpar}
    x

    \mathsf{socrates}

    \mathsf{player1}

    \mathsf{dayAfter}(\mathsf{today})

    2x + 3y

    \mathsf{nameOf}(\mathit{cust})

    \mathsf{dayAfter}(\mathsf{dayAfter}(d))
  \end{mathpar}

  \bigskip

  \rhighlight{Formulas}~ Built from properties and relations,
  connectives and quantifiers.
  \begin{displaymath}
    \begin{array}{l}
      \exists x.~\mathrm{customer}(x) \land \mathrm{loggedOff}(x)\\
      \\
      \forall x.~\mathrm{human}(x) \to \mathrm{mortal}(x)\\
      \\
      \forall d.~\mathrm{raining}(d) \to \mathrm{raining}(\mathsf{dayAfter}(d)) \\
      \\
      \forall n.\exists k. (n = k + k) \lor (n = k + k + 1)
    \end{array}
  \end{displaymath}
\end{frame}

\def\hlafter<#1>#2#3{%
  \temporal<#1>%
  {\colorbox{white}{\color{black}$\displaystyle#3$}}%
  {\colorbox{#2}{\color{black}$\displaystyle#3$}}%
  {\colorbox{#2}{\color{black}$\displaystyle#3$}}}

\begin{frame}
  {Anatomy of a Formula}

  \bigskip

  ``All humans are mortal''
  \begin{displaymath}
    \hlafter<5>{yellow}{\forall x.}~\hlafter<3>{blue!50}{\mathrm{human}}(\hlafter<2>{green!50}{x}) \hlafter<4>{pink}{\to} \hlafter<3>{blue!50}{\mathrm{mortal}}(\hlafter<2>{green!50}{x})
  \end{displaymath}
  \begin{enumerate}
  \item<2-> Variables, standing for general individuals
  \item<3-> Properties (``Predicates'') of those individuals
  \item<4-> Connectives, as in Propositional Logic
  \item<5-> Quantifiers, telling us how to interpret the general individual $x$
  \end{enumerate}
\end{frame}

\begin{frame}
  \sechead{Anatomy of a Formula}

  ``Socrates is a human''
  \begin{displaymath}
    \hlafter<3>{blue!50}{\mathrm{human}}(\hlafter<2>{green!50}{\mathsf{socrates}})
  \end{displaymath}
  \begin{enumerate}
  \item<2-> A specific individual
  \item<3-> Property of that individual
  \end{enumerate}

\end{frame}

\begin{frame}
  \sechead{Anatomy of a Formula}

  \bigskip

  ``No bird can fly in space''
  \begin{displaymath}
    \hlafter<4>{pink}{\lnot} (\hlafter<5>{yellow}{\exists x.} \hlafter<3>{blue!50}{\mathrm{bird}}(\hlafter<2>{green!50}{x}) \hlafter<4>{pink}{\land} \hlafter<3>{blue!50}{\mathrm{canFlyInSpace}}(\hlafter<2>{green!50}{x}))
  \end{displaymath}
  \begin{enumerate}
  \item<2-> Variables, standing for general individuals
  \item<3-> Properties (``Predicates'') of those individuals
  \item<4-> Connectives, as in Propositional Logic
  \item<5-> Quantifiers, telling us how to interpret the general individual $x$
  \end{enumerate}
\end{frame}

\begin{frame}
  \sechead{Anatomy of a Formula}

  \bigskip

  ``If it is raining on a day, it is raining the day after''
  \begin{displaymath}
    \hlafter<6>{yellow}{\forall d.} \hlafter<4>{blue!50}{\mathrm{raining}}(\hlafter<2>{green!50}{d}) \hlafter<5>{pink}{\to} \hlafter<4>{blue!50}{\mathrm{raining}}(\hlafter<3>{purple!50}{\mathsf{dayAfter}}(\hlafter<2>{green!50}{d}))
  \end{displaymath}
  \begin{enumerate}
  \item<2-> Variables, standing for general individuals
  \item<3-> Function symbols, performing operations on individuals
  \item<4-> Properties (``Predicates'') of those individuals
  \item<5-> Connectives, as in Propositional Logic
  \item<6-> Quantifiers,\kern-0.05pt telling us how to interpret the general individual $d$
  \end{enumerate}
\end{frame}

\begin{frame}
  \sechead{Anatomy of a Formula}

  \bigskip

  ``Every number is even or odd''
  \begin{displaymath}
    \hlafter<6>{yellow}{\forall n.} \hlafter<6>{yellow}{\exists k.} (\hlafter<2>{green!50}{n} \hlafter<4>{blue!50}{=} \hlafter<2>{green!50}{k} \hlafter<3>{purple!50}{+} \hlafter<2>{green!50}{k}) \hlafter<5>{pink}{\lor} (\hlafter<2>{green!50}{n} \hlafter<4>{blue!50}{=} \hlafter<2>{green!50}{k} \hlafter<3>{purple!50}{+} \hlafter<2>{green!50}{k} \hlafter<3>{purple!50}{+} \hlafter<2>{green!50}{1})
  \end{displaymath}
  \begin{enumerate}
  \item<2-> General ($n$, $k$) and specific ($1$) individuals
  \item<3-> Function symbols, performing operations on individuals
  \item<4-> Relations between individuals (here: equality)
  \item<5-> Connectives, as in Propositional Logic
  \item<6-> Quantifiers, telling us how to interpret the general individuals $n$ and $k$
  \end{enumerate}
\end{frame}

\begin{frame}
  \sechead{More examples}

  ``Every day is raining or sunny''
  \begin{displaymath}
    \forall d. \mathrm{raining}(d) \lor \mathrm{sunny}(d)
  \end{displaymath}

  \bigskip

  ``Does there exist a player within 10 metres of player 1?''
  \begin{displaymath}
    \exists p. \mathrm{player}(p) \land \mathsf{distance}(\mathsf{locationOf}(p), \mathsf{locationOf}(\mathsf{player1})) \leq 10
  \end{displaymath}
\end{frame}

\begin{frame}
  \sechead{Examples from Mathematics}

  \bigskip

  \emph{Fermat's Last Theorem}
  \begin{displaymath}
    \forall n. n > 2 \to \lnot (\exists a. \exists b. \exists c. a^n + b^n = c^n)
  \end{displaymath}
  \sidenote{stated in 1637, not proved until 1994}

  \bigskip

  \emph{Goldbach's Conjecture} \\
  \sidenote{Every even number greater than 2 is the sum of two primes}
  \begin{displaymath}
    \forall n. n > 2 \to \mathrm{even}(n) \to \exists p. \exists q. \mathrm{prime}(p) \land \mathrm{prime}(q) \land p+q = n
  \end{displaymath}

\end{frame}

\begin{frame}
  {Summary}

  Predicate Logic upgrades Propositional Logic, adding:
  \begin{itemize}
  \item Individuals $x$, $y$, $z$
  \item Functions $+$, $\mathsf{dayAfter}$
  \item Predicates $=$, $\mathrm{even}$, $\mathrm{odd}$
  \item Quantifiers $\forall$, $\exists$
  \end{itemize}
\end{frame}

\weeksection{Saying what you mean}

\begin{frame}[t]
  \sechead{How to say ``$x$ is a $P$''}
  \begin{displaymath}
    P(x)
  \end{displaymath}
  For example:
  \begin{displaymath}
    \begin{array}{c}
      \mathrm{human}(x) \\
      \mathrm{mortal}(x) \\
      \mathrm{swan}(x) \\
      \mathrm{golden}(x)
    \end{array}
  \end{displaymath}
\end{frame}

\begin{frame}[t]
  \sechead{How to say ``$x$ and $y$ are related by $R$''}
  \begin{displaymath}
    R(x,y)
  \end{displaymath}
  for example:
  \begin{displaymath}
    \begin{array}{c}
      \mathrm{colour}(x, \mathsf{gold}) \\
      \mathrm{species}(x, \mathsf{swan}) \\
      \mathrm{connected}(x,y) \\
      \mathrm{knows}(\mathsf{pooh},\mathsf{piglet})
    \end{array}
  \end{displaymath}
\end{frame}

\begin{frame}[t]
  \sechead{``Everything is P''}

  \begin{center}
    \emph{everything is boring} \\
    \emph{everything is wet}
  \end{center}

  \begin{displaymath}
    \begin{array}{c}
      \forall x. \mathrm{boring}(x) \\
      \forall x. \mathrm{wet}(x) \\
      \\
      \forall x. P(x)
    \end{array}
  \end{displaymath}

  \bigskip

  Usually not very \emph{useful} if $P$ is atomic, but things like
  \begin{displaymath}
    \forall x. \mathrm{even}(x) \lor \mathrm{odd}(x)
  \end{displaymath}
  are useful.

\end{frame}


\begin{frame}[t]
  \sechead{``There exists an P''}

  \begin{center}
    \emph{there is a human} \\
    \emph{there is a swan} \\
    \emph{there is an insect}
  \end{center}

  \begin{displaymath}
    \begin{array}{c}
      \exists x. \mathrm{human}(x) \\
      \exists x. \mathrm{swan}(x) \\
      \exists x. \mathrm{class}(x,\mathsf{insecta}) \\
      \\
      \exists x. P(x)
    \end{array}
  \end{displaymath}

  \bigskip

  \emph{there is at least one thing with property $P$}

\end{frame}


\begin{frame}[t]
  \sechead{``All P are Q''}

  \begin{center}
    \emph{all humans are mortal} \\
    \emph{all swans are white} \\
    \emph{all insects have 6 legs}
  \end{center}

  \begin{displaymath}
    \begin{array}{c}
      \forall x. \mathrm{human}(x) \to \mathrm{mortal}(x) \\
      \forall x. \mathrm{swan}(x) \to \mathrm{white}(x) \\
      \forall x. \mathrm{insect}(x) \to \mathrm{numLegs}(x,6) \\
      \\
      \forall x. P(x) \to Q(x)
    \end{array}
  \end{displaymath}

  \begin{center}
    \emph{for all $x$, if $x$ is $P$, then $x$ is $Q$}
  \end{center}
\end{frame}

% \begin{frame}
%   \rhighlight{To prove} $\forall x. X(x) \to Y(x)$ \\
%   \quad given an arbitrary $x_0$, we get to assume $X(x_0)$ to prove
%   $Y(y_0)$.

%   \begin{displaymath}
%     \inferrule* [right=$\forall$-I]
%     {\inferrule* [Right=$\to$-I]
%       { {\begin{array}[b]{c}
%            [X(x_0)] \\ \vdots \\ Y(x_0)
%          \end{array}}
%       }
%       {X(x_0) \to Y(x_0)}
%     }
%     {\forall x. X(x) \to Y(x)}
%   \end{displaymath}

%   \medskip

%   \rhighlight{To use:} $\forall x. X(x) \to Y(x)$ \\
%   \quad if we have a $t$ such that $X(t)$, then we
%   learn $Y(t)$.

%   \begin{displaymath}
%     \inferrule*
%     {\inferrule*
%       { {\begin{array}[b]{c} \vdots\\\forall x. X(x) \to Y(x)\end{array}} }
%       {X(t) \to Y(t)}
%       \\
%       {\begin{array}[b]{c} \vdots \\ X(t) \end{array}}}
%     {Y(t)}
%   \end{displaymath}
% \end{frame}

\begin{frame}[t]
  \sechead{``Some P is Q''}

  \begin{center}
    \emph{some human is mortal} \\
    \emph{some swan is black} \\
    \emph{some insect has 6 legs}
  \end{center}

  \begin{displaymath}
    \begin{array}{c}
      \exists x. \mathrm{human}(x) \land \mathrm{mortal}(x) \\
      \exists x. \mathrm{swan}(x) \land \mathrm{colour}(x,\mathsf{black}) \\
      \exists x. \mathrm{insect}(x) \land \mathrm{numLegs}(x,6)\\
      \\
      \exists x. P(x) \land Q(x)
    \end{array}
  \end{displaymath}

  \begin{center}
    \emph{exists $x$, such that $x$ is a $P$ and $x$ is a $Q$}
  \end{center}

\end{frame}

\begin{frame}[t]
  \sechead{``All P are Q'' vs ``Some P are Q''}

  \begin{displaymath}
    \forall x. P(x) \to Q(x)
  \end{displaymath}
  uses $\to$, but
  \begin{displaymath}
    \exists x. P(x) \land Q(x)
  \end{displaymath}
  uses $\land$.

  \pause
  \bigskip

  Tempting to write:
  \begin{displaymath}
    \forall x. P(x) \land Q(x) \quad \textrm{\textcolor{black!60}{everything is both $P$ and $Q$}}
  \end{displaymath}
  or
  \begin{displaymath}
    \exists x. P(x) \to Q(x) \quad \textrm{\textcolor{black!60}{there is some $x$, such that if $P$ then $Q$}}
  \end{displaymath}
  but {\bf almost always not what you want}.
\end{frame}

\begin{frame}[t]
  \sechead{``No P is Q''}

  \begin{center}
    \emph{no swans are blue} \\
    \emph{no bird can fly in space} \\
    \emph{no program works}
  \end{center}

  \begin{displaymath}
    \begin{array}{c}
      \forall x. \mathrm{swan}(x) \to \lnot \mathrm{blue}(x) \\
      \lnot (\exists x. \mathrm{bird}(x) \land \mathrm{canFlyInSpace}(x)) \\
      \forall x. \mathrm{program}(x) \to \lnot \mathrm{works}(x)
    \end{array}
  \end{displaymath}

  \begin{mathpar}
    \lnot (\exists x. P(x) \land Q(x))

    \textrm{or}

    \forall x. P(x) \to \lnot Q(x)
  \end{mathpar}

  % TUTORIAL QUESTION

  \bigskip

  The two statements are equivalent.
\end{frame}

\begin{frame}[t]
  \sechead{``For every P, there exists a related Q''}

  \begin{center}
    \emph{every farmer owns a donkey} \\ % ambiguous
    \emph{every day has a next day} \\
    \emph{every list has a sorted version} \\
    \emph{every position has a nearby safe position}
  \end{center}
  \begin{displaymath}
    \begin{array}{c}
      \forall f. \mathrm{farmer}(f) \to (\exists d. \mathrm{donkey}(d) \land \mathrm{owns}(f,d))\\
      \forall d. \mathrm{day}(d) \to (\exists d'. \mathrm{day}(d') \land \mathrm{next}(d,d')) \\
      \forall x. \mathrm{list}(x) \to (\exists y. \mathrm{list}(y) \land \mathrm{sorted}(y) \land \mathrm{sameElements}(x,y)) \\
      \forall p_1. \exists p_2. \mathrm{nearby}(p_1,p_2) \land \mathrm{safe}(p_2)
    \end{array}
  \end{displaymath}

  In steps:
  \begin{enumerate}
  \item For every $x$ (they choose),
  \item There is a $y$ (we choose),
  \item such that $x$ and $y$ are related.
  \end{enumerate}

\end{frame}

\begin{frame}[t]
  \sechead{``There exists an P such that every Q is related''}

  \begin{center}
    \emph{every farmer owns a donkey} (!!!) \\
    \emph{there is someone that everyone loves} \\
    \emph{there is someone that loves everyone} \\
  \end{center}

  \begin{displaymath}
    \begin{array}{c}
      \exists d. \mathrm{donkey}(d) \land (\forall f. \mathrm{farmer}(f) \to \mathrm{owns}(f,d))\\
      \exists x. \forall y. \mathrm{loves}(y, x) \\
      \exists x. \forall y. \mathrm{loves}(x, y) \\
    \end{array}
  \end{displaymath}

  In steps:
  \begin{enumerate}
  \item there exists an $x$ (we choose), such that
  \item forall $y$ (they choose),
  \item it is the case that $x$ and $y$ are related.
  \end{enumerate}
\end{frame}

\begin{frame}[t]
  \sechead{``For all P, there is a related Q, related to all R''}

  \begin{center}
%    \emph{every node is connected to a node that is connected to all nodes} \\
    \emph{everyone knows someone who knows everyone}
  \end{center}

  \begin{displaymath}
    \begin{array}{c}
%      \forall x. \exists y. \mathrm{connected}(x,y) \land (\forall z. \mathrm{connected}(y,z))\\
      \forall x. \exists y. \mathrm{knows}(x,y) \land (\forall z. \mathrm{knows}(y,z)) \\
      \\
      \forall x. P(x) \to (\exists y. Q(x,y) \land (\forall z. \mathrm{R}(x,y,z))
    \end{array}
  \end{displaymath}

  In steps:
  \begin{enumerate}
  \item for all $x$ (they choose),
  \item there exists a $y$ (we choose),
  \item for all $z$ (they choose),
  \item such that $x$, $y$, $z$ are related.
  \end{enumerate}
\end{frame}

\begin{frame}
  \sechead{``There exists exactly one X''}

  \begin{center}
    \emph{there's only one moon} \\
  \end{center}

  ``Any other individual with the same property is equal''
  \begin{displaymath}
    \exists x. \mathrm{moon}(x) \land (\forall y. \mathrm{moon}(y) \to x = y)
  \end{displaymath}
  not quite the same, but similar:
  \begin{displaymath}
    \forall x. \forall y. (\mathrm{moon}(x) \land \mathrm{moon}(y)) \to x = y
  \end{displaymath}
  this says: \emph{at most one moon}, but doesn't say one exists.

\end{frame}

\begin{frame}
  \sechead{``For every X, there exists exactly one Y''}

  \begin{center}
    \emph{every train has one driver}
  \end{center}

  \begin{displaymath}
    \forall t. \mathrm{train}(t) \to (\exists d.\mathrm{driver}(d,t) \land (\forall d'. \mathrm{driver}(d',t) \to d = d'))
  \end{displaymath}

\end{frame}

\begin{frame}
  \sechead{There exists an X such that for all Y there exists a Z}

  \begin{center}
    \emph{there is a node, such that for all reachable nodes, \\
      there is a safe node in one step}
  \end{center}

  \begin{displaymath}
    \exists a. \forall b. \mathrm{reachable}(a,b) \to (\exists c. \mathrm{safe}(c) \land \mathrm{step}(b,c))
  \end{displaymath}

  Not the same as:

  \begin{displaymath}
    \exists a. \exists c. \forall b. \mathrm{reachable}(a,b) \to (\mathrm{safe}(c) \land \mathrm{step}(b,c))
  \end{displaymath}

  \begin{enumerate}
  \item First one: $c$ can be different for each $b$.
  \item Second: the same $c$ for all $b$.
  \end{enumerate}

\end{frame}

\begin{frame}
  {Summary}

  \begin{itemize}
  \item Many of the things you want to say in Predicate Logic fall
    into one of several predefined templates.
  \item It helps to think of quantifiers as a game
    \begin{itemize}
    \item $\forall$ means ``they choose''
    \item $\exists$ means ``I choose''
    \end{itemize}
    (but they switch places under a negation or on the left of an
    implication!)
  \end{itemize}
\end{frame}

\weeksection{Syntax Details}

\begin{frame}
  {Predicate Logic}

  Predicate Logic upgrades Propositional Logic, adding:
  \begin{itemize}
  \item Individuals $x$, $y$, $z$
  \item Functions $+$, $\mathsf{dayAfter}$
  \item Predicates $=$, $\mathrm{even}$, $\mathrm{odd}$
  \item Quantifiers $\forall$, $\exists$
  \end{itemize}
\end{frame}

\begin{frame}
  {Predicate Logic is for \emph{Modelling}}

  \bigskip

  To state properties of some situation we want to model, we choose:
  \begin{enumerate}
  \item Names of specific individuals \\
    \sidenote{$\mathsf{socrates}, 1, 2, 10000, \mathsf{localhost}, \mathsf{www.strath.ac.uk}$}
  \item Function symbols \\
    \sidenote{$+$, $\times$, $\mathsf{nameOf}$}
  \item Relation symbols \\
    \sidenote{$\mathrm{human}(x)$, $x=y$, $\mathrm{linksTo}(x,y)$}
  \item Some axioms \\
    \sidenote{later ...}
  \end{enumerate}

  \bigskip

  Usually, we build a vocabulary based on what we want to do.
\end{frame}

\begin{frame}
  {Vocabulary for Arithmetic}

  Individuals:
  \begin{mathpar}
    0

    1

    2

    3

    \dots
  \end{mathpar}
  Functions:
  \begin{mathpar}
    t_1 + t_2

    t_1 - t_2

    \dots
  \end{mathpar}
  Predicates:
  \begin{mathpar}
    t_1 = t_2

    t_1 < t_2

    \dots
  \end{mathpar}
\end{frame}

\begin{frame}
  {Vocabulary for Documents}

  Individuals:
  \begin{mathpar}
    \textrm{``Frankenstein''}

    \textrm{``Dracula''}

    \textrm{``Bram Stoker''}

    \textrm{``Mary Shelley''}
  \end{mathpar}
  Predicates:
  \begin{mathpar}
    \mathrm{linksTo}(\mathit{doc}_1,\mathit{doc}_2)

    \mathrm{authorOf}(\mathit{doc},\mathit{person})

    \mathrm{ownerOf}(\mathit{doc},\mathit{person})
  \end{mathpar}
\end{frame}

\begin{frame}
  {Vocabulary for Programs}

  Individuals
  \begin{mathpar}
    \texttt{java.lang.Object}

    \texttt{j.l.String}

    \texttt{j.l.Runnable}

    \texttt{String toString()}
  \end{mathpar}
  Relations
  \begin{mathpar}
    \mathrm{extends}(\mathit{class}_1,\mathit{class}_2)

    \mathrm{implements}(\mathit{class},\mathit{interface})

    \mathrm{hasMethod}(\mathit{class},\mathit{method})

    \dots
  \end{mathpar}
\end{frame}

\begin{frame}
  {Equality}

  The equality predicate
  \begin{displaymath}
    t_1 = t_2
  \end{displaymath}
  is treated specially:
  \begin{itemize}
  \item and in proofs (Topic 4)
  \item in the semantics (Topic 8)
  \end{itemize}
\end{frame}

% \begin{frame}[t]
%   {``Universal'' vocabularies?}

%   \bigskip

%   \begin{itemize}
%   \item Zermelo-Frankel set theory
%     \begin{itemize}
%     \item one named individual $\{\}$ -- the empty set
%     \item one relation $x \in y$ -- set membership
%     \item 9 axioms (+ optional ``Choice'' axiom)
%     \item Can define ``all'' of modern mathematics in this system \\
%       e.g. Metamath \url{http://us.metamath.org/index.html}
%     \end{itemize}

%     \bigskip
%   \item DBpedia : \url{https://wiki.dbpedia.org/} \\
%     \sidenote{structured version of Wikipedia, (partly) used for Google's ``info box'' on searches}

%     \bigskip
%   \item Cyc : \url{http://www.cyc.com/} \\
%     \sidenote{building a ``knowledge base'' since 1984}
%   \end{itemize}
% \end{frame}

% \begin{frame}
%   {Existing vocabularies}

%   \begin{itemize}
%   \item OpenGraph \url{http://ogp.me/} \\
%     \sidenote{Metadata readable by Facebook}
%     \begin{displaymath}
%       \begin{array}{cl}
%         &\texttt{<meta property="og:title" content="The Rock" />} \\
%         \approx&\mathrm{title}(\mathit{currentPage}, \mathsf{"The~Rock"})
%       \end{array}
%     \end{displaymath}

%     \bigskip
%   \item DublinCore \url{http://dublincore.org/about/} \\
%     \sidenote{Standardised metadata for digital objects} \\
%   \end{itemize}
% \end{frame}

\begin{frame}
  {Formal Grammar}

  \begin{displaymath}
    \begin{array}{lcll}
      t&::= &x &\textrm{\textcolor{black!60}{variables}} \\
       &\mid&\mathsf{c} &\textrm{\textcolor{black!60}{constants}}\\
       &\mid&\mathsf{f}(t_1, \dots, t_n) &\textrm{\textcolor{black!60}{function terms}}\\
      \\
      P&::= &R(t_1,\dots,t_n) & \textrm{\textcolor{black!60}{predicates}} \\
       &\mid&P \land Q \mid P \lor Q \mid P \to Q \mid \lnot P & \textrm{\textcolor{black!60}{connectives}}\\
       &\mid&\forall x.P \mid \exists x.P & \textrm{\textcolor{black!60}{quantifiers}}
    \end{array}
  \end{displaymath}

  \emph{Propositional Logic} as special case: all relation symbols
  have arity $0$.
\end{frame}

\begin{frame}
  {When are two formulas the same?}

  Is there a difference in meaning between these two?
  \begin{mathpar}
    \forall x. P(x)

    \textrm{and}

    \forall y. P(y)
  \end{mathpar}

  \bigskip
  \pause

  No! They both mean the same thing.

  \bigskip
  \pause

  So we treat them as identical formulas.
\end{frame}

\begin{frame}
  {Free and Bound Variables}

  In the formula:
  \begin{displaymath}
    \exists y. R(x,y)
  \end{displaymath}
  \begin{enumerate}
  \item The variable $x$ is \emph{free}
  \item The variable $y$ is \emph{bound} (by the $\exists$ quantifier)
  \end{enumerate}

  \bigskip

  The quantifiers are \emph{binders}.
\end{frame}

\begin{frame}
  {Free and Bound Variables}

  Pay attention to the bracketing:
  \begin{displaymath}
    (\forall x. P(x) \to Q(x)) \land (\exists y. R(x,y))
  \end{displaymath}
  The $x$s to the left of the $\land$ are bound (by the $\forall$)

  \bigskip

  The $x$ to the right of the $\land$ is free.

  \bigskip

  When a variable is bound by quantifier, we say that it is in that
  quantifiers \emph{scope}.
\end{frame}

\begin{frame}
  {Identical Formulas, again}

  We can only rename \emph{bound} variables
  \begin{mathpar}
    \exists y. R(x,y)

    \textrm{is identical to}

    \exists z. R(x,z)
  \end{mathpar}

  \bigskip

  but
  \begin{mathpar}
    \exists y. R(x,y)

    \textrm{is not identical to}

    \exists y. R(z,y)
  \end{mathpar}

  because $x$ and $z$ do not have the same ``global'' meaning.
\end{frame}

\begin{frame}
  {Summary}

  Vocabularies define the symbols we can use in our formulas.

  \bigskip


  The formal syntax of Predicate Logic is more complex than
  Propositional Logic
  \begin{itemize}
  \item Free and Bound Variables
  \item Formulas are identical even when renaming bound variables.
  \end{itemize}
\end{frame}

\weeksection{Substitution}

\begin{frame}
  {From General to Specific}

  We will have \emph{general} assumptions like:
  \begin{displaymath}
    \forall x. \mathrm{human}(x) \to \mathrm{mortal}(x)
  \end{displaymath}

  \bigskip

  And we want to \emph{specialise} (or \emph{instantiate}) to:
  \begin{displaymath}
    \mathrm{human}(\mathsf{socrates}()) \to \mathrm{mortal}(\mathsf{socrates}())
  \end{displaymath}
\end{frame}

\begin{frame}
  {Substitution}
  The notation
  \begin{displaymath}
    P[x:=t]
  \end{displaymath}
  means ``replace all \emph{free} occurrences of $x$ in $P$ with $t$''.
  \begin{itemize}
  \item $x$ is a \emph{variable}
  \item $P$ is a \emph{formula}
  \item $t$ is a \emph{term}
  \end{itemize}


  \bigskip

  \textcolor{black!60}{But there is a subtlety...}
\end{frame}

\begin{frame}
  {Substitution Examples}

  \begin{displaymath}
    \begin{array}{cl}
      &(\mathrm{mortal}(x))[x := \mathsf{socrates}()]\\
      \Longrightarrow&\mathrm{mortal}(\mathsf{socrates}())\\
    \end{array}
  \end{displaymath}
\end{frame}

\begin{frame}
  {Substitution Examples}

  \begin{displaymath}
    \begin{array}{cl}
      &(\forall y. \mathrm{weatherIs}(d,y) \to \mathrm{weatherIs}(\mathsf{dayAfter}(d),y))[d:=\mathsf{tuesday}]\\
      \Longrightarrow&\forall y. \mathrm{weatherIs}(\mathsf{tuesday},y) \to \mathrm{weatherIs}(\mathsf{dayAfter}(\mathsf{tuesday}),y)\\
    \end{array}
  \end{displaymath}
\end{frame}

\begin{frame}
  {Substitution Examples}

  \begin{displaymath}
    \begin{array}{cl}
      &(\exists y. \mathrm{sameElements}(x,y) \land \mathrm{sorted}(y))[x:=\mathsf{cons}(z_1,\mathsf{cons}(z_2,\mathsf{nil}))]\\
      \Longrightarrow&\exists y. \mathrm{sameElements}(\mathsf{cons}(z_1,\mathsf{cons}(z_2,\mathsf{nil})),y) \land \mathrm{sorted}(y)\\
    \end{array}
  \end{displaymath}
\end{frame}

\begin{frame}
  {Substitution Examples}
  \begin{displaymath}
    \begin{array}{cl}
      &(\forall y. x+y=y+x)[x:=z-z] \\
      \Longrightarrow&\forall y. (z-z)+y = y+(z-z)
    \end{array}
  \end{displaymath}
\end{frame}

\begin{frame}[t]
  {Accidental Name Capture}

  If we substitute naively, then we produce nonsense:
  \begin{enumerate}
  \item $\exists y. \mathrm{sameElements}(x,y)$ \\
    \emph{``there exists a $y$ that has the same elements as $x$''}\\
    ~\\
  \item $(\exists y. \mathrm{sameElements}(x,y))[x:=\mathsf{append}(y,[1,2])]$\\
    \emph{``replace $x$ by the list $\mathsf{append}(y,[1,2])$''}\\
    ~\\
  \item $\exists y. \mathrm{sameElements}(\mathsf{append}(y,[1,2]),y)$\\
    \emph{``there exists a $y$ that has the same elements as $y + [1,2]$?''}
  \end{enumerate}
\end{frame}

\begin{frame}
  {Capture Avoidance}

  \bigskip

  \textbf{Solution: Rename bound variables}
  \begin{displaymath}
    \begin{array}{cl}
      &(\exists y. \mathrm{sameElements}(x,y))[x:=\mathsf{append}(y,[1,2])]\\
      \Longrightarrow&(\exists z. \mathrm{sameElements}(x,z))[x:=\mathsf{append}(y,[1,2])]\\
      \Longrightarrow&\exists z. \mathrm{sameElements}(\mathsf{append}(y,[1,2]),z)
    \end{array}
  \end{displaymath}
\end{frame}

\begin{frame}
  {Capture Avoiding Substitution}

  When working out
  \begin{displaymath}
    P[x := t]
  \end{displaymath}
  If any of the variables in $t$ are bound in $P$ then rename them
  before doing the substitution.
\end{frame}

\begin{frame}
  {Substitution Examples}

  \begin{enumerate}
  \item $P(x,y)[x:=y+y]$\pause \quad $= P(y+y,y)$
    \pause
    \bigskip
  \item $P(x,y)[y:=y+y]$\pause \quad $= P(x,y+y)$
    \pause
    \bigskip
  \item $(\forall x. P(x,y))[x:=y+y]$\pause \quad $= \forall x. P(x,y)$
  \end{enumerate}
\end{frame}


\begin{frame}
  {Substitution Examples}

  \begin{enumerate}
  \item $(\forall x. P(x,y))[y:=x+x]$\pause \quad $= \forall z. P(z,x+x)$ \\
    \qquad \textcolor{black!60}{Renaming!}
    \pause
    \bigskip
  \item $(\forall x. P(x,y) \to (\exists z. Q(y,z)))[y:=z+z]$\pause\\
    $= \forall x. P(x,z+z) \to (\exists w. Q(z+z,w))$\\
    \qquad \textcolor{black!60}{Renaming!}
    \pause
    \bigskip
  \item $(\forall x. P(x,z) \to (\exists z. Q(y,z)))[z:=x+x]$\pause\\
    $= \forall w. P(w,x+x) \to (\exists z. Q(y,z))$\\
    \qquad \textcolor{black!60}{Renaming! and no substitution of the final $z$}
  \end{enumerate}
\end{frame}

\begin{frame}
  {Summary}

  \begin{itemize}
  \item Substitution
    \begin{displaymath}
      P [x := t]
    \end{displaymath}
    is how we go from the general $x$ to the specific $t$.
  \item We need to be careful to rename bound variables to avoid
    accidental name capture.
  \end{itemize}
\end{frame}




\end{document}
