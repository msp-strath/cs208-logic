% -*- TeX-engine: xetex -*-

\documentclass[xetex,aspectratio=169,14pt,hyperref={pdfpagelabels=true,pdflang={en-GB}}]{beamer}

\usepackage[sci,noslidestrathidentity]{strathclyde}
\strathsetidentity{Department of}{Computer \& Information Sciences}

\usepackage{lmodern}
\usepackage{subscript}
\usepackage{url}
\usepackage{pifont}
\usepackage{csquotes}
\usepackage{mathpartir}
\usepackage{stmaryrd}
\usepackage{multirow}
\usepackage{euler}
\usepackage[normalem]{ulem}

% 2013-09-08 SU adding xetex specifics
% http://www.woggie.net/2008/07/16/beamer-pdftex-and-xetex/
\usepackage{xltxtra}

% 2013-09-08 SU http://robjhyndman.com/hyndsight/xelatex/
\defaultfontfeatures{Ligatures=TeX}

% 2015-12-09 SU table beautified
\renewcommand{\arraystretch}{1.2}
\usepackage{booktabs}

%MGK compatibility
\newcommand{\hh}[1]
  {\medskip\textbf{\large #1}}
\newcommand{\pdu}[3]
  {#1\rightarrow #2 : &\ & \makebox[70mm][l]{$#3$}}



\setlength{\marginparwidth}{2cm}
\usepackage{todonotes}%[disable]
\let\OldTodo\todo
\renewcommand{\todo}{\OldTodo[inline]}%
\newcommand{\todolater}[1]{}% Things to do for next year


\DeclareTextCommandDefault{\nobreakspace}{\leavevmode\nobreak\ }
\usepackage{rotating}

\usepackage{pdfcomment}% To add alt text for images using \pdftooltip{}

\usepackage{appendixnumberbeamer}

\newcommand{\messageframe}[1]{\begin{frame}\begin{center}\Huge #1\end{center}\end{frame}}
\newcommand{\sechead}[1]{{\bf #1} \\}
\newcommand{\examplehead}[1]{{\bf Example:} {\it\textcolor{red!90}{#1}} \\}
\newcommand{\eqnote}[1]{\hspace{3cm}\textit{#1}}
\newcommand{\sidenote}[1]{\qquad {\footnotesize \textcolor{black!60}{(#1)}}}
\newcommand{\sem}[1]{\llbracket #1 \rrbracket}
\newcommand{\true}{\mathsf{T}}
\newcommand{\false}{\mathsf{F}}

\def\strikeafter<#1>#2{\temporal<#1>{#2}{\sout{#2}}{\sout{#2}}}


%\newcommand{\rhighlight}{\textcolor{titlered}}
\newcommand{\rhighlight}{\textbf}
\newcommand{\highlight}{\textbf}

% \setmainfont{Linux Biolinum O}
\setmainfont{LinBiolinum}[
Path=,
UprightFont = *_R.otf ,
BoldFont = *_RB.otf ,
ItalicFont = *_RI.otf
]

\setbeamertemplate{navigation symbols}{}
%\usecolortheme[rgb={0.8,0,0}]{structure}
\usefonttheme{serif}
\usefonttheme{structurebold}
\setbeamercolor{description item}{fg=black}


\author[Atkey]{Dr.~Robert Atkey}
\institute[Strathclyde]{Computer \& Information Sciences}
\date[]{}

\newcommand{\weeksection}[1]{%
  \section{\thetitle{}, Part~\thesection : #1}
  \begin{frame}
    \begin{center}
      \textcolor{black!60}{\thetitle{}, Part \thesection}\\
      {\Huge #1}
    \end{center}
  \end{frame}}

\newcommand{\weektitle}[2]{\def\thetitle{#2}
\title[CS208 - Topic #1]{CS208 (Semester 1) Topic #1 : #2}}

\newcommand{\assigned}{:}

\newcommand{\forcedto}{\assigned_f}
\newcommand{\decideto}{\assigned_d}


\weektitle{1}{Entailment and Deduction}

%\title[CS208 - Topic 01]{CS208 Topic 01 \\ Entailment and Deduction}

\begin{document}

\frame{\titlepage}

\weeksection{Entailment}

\begin{frame}
  {Entailment}

  Entailment is a relation between some assumptions:
  \begin{displaymath}
    P_1, \dots, P_n
  \end{displaymath}
  and a conclusion:
  \begin{displaymath}
    Q
  \end{displaymath}

  \pause
  \bigskip

  What we want to capture is:

  \medskip

  \qquad \emph{If we assume $P_1$, ..., $P_n$ are all true, then it is safe to conclude $Q$.}

  % \pause
  % \bigskip

  % \sechead{Note:} Satisfiability and Validity are properties of
  % individual formulas. Entailment is a \emph{relation} between a
  % collection of formulas (the assumptions) and a single formula (the
  % conclusion).
\end{frame}

\begin{frame}[t]
  \frametitle{Is it safe?}

  If we assume
  \begin{center}
    it is sunny
  \end{center}
  then is it safe to conclude
  \begin{center}
    it is sunny
  \end{center}

  \bigskip
  \pause

  Yes! \pause There are two cases:
  \begin{enumerate}
  \item It is sunny  (i.e., $v(\mathrm{Sunny}) = \true$)
  \item \strikeafter<4>{It isn't sunny (i.e., $v(\mathrm{Sunny}) = \false$)}
  \end{enumerate}
  \pause But we are assuming ``it is sunny'', so the second case
  doesn't matter.
\end{frame}

\begin{frame}[t]
  \frametitle{Is it safe?}

  If we assume
  \begin{center}
    \emph{nothing}
  \end{center}
  then is it safe to conclude
  \begin{center}
    it is sunny
  \end{center}

  \bigskip
  \pause

  No! \pause There are two cases:
  \begin{enumerate}
  \item It is sunny  (i.e., $v(\mathrm{Sunny}) = \true$)
  \item It isn't sunny (i.e., $v(\mathrm{Sunny}) = \false$)
  \end{enumerate}
  \pause But we are making no assumptions, so either ``world'' is
  possible: it might not be sunny.
\end{frame}

\begin{frame}[t]
  \frametitle{Is it safe?}

  If we assume
  \begin{center}
    it is raining
  \end{center}
  then it is safe to conclude:
  \begin{center}
    it is not sunny
  \end{center}

  \bigskip
  \pause

  No! \pause There are four cases:
  \begin{enumerate}
  \item It is sunny and raining
  \item \strikeafter<4>{It is sunny and not raining}
  \item It is not sunny, but is raining
  \item \strikeafter<4>{It is not sunny and not raining}
  \end{enumerate}
  \pause % We are assuming ``it is raining'', so cases 2 and 4 don't
  % matter, but that still leaves case 1: it may be sunny and raining.
\end{frame}

\begin{frame}[t]
  \frametitle{Is it safe?}

  If we assume
  \begin{center}
    it is raining \qquad \emph{and} \qquad if it is raining it is not sunny
  \end{center}
  then is it safe to conclude:
  \begin{center}
    it is not sunny
  \end{center}

  \bigskip
  \pause

  Yes! \pause There are four cases:
  \begin{enumerate}
  \item \strikeafter<5>{It is sunny and raining}
  \item \strikeafter<4>{It is sunny and not raining}
  \item It is not sunny, but is raining
  \item \strikeafter<4>{It is not sunny and not raining}
  \end{enumerate}
  \pause % We are assuming ``it is raining'', so cases 2 and 4 don't
  % matter. \pause We are also assuming ``if it is raining, it is not
  % sunny'', so case 1 is removed too. The only remaining case has ``not
  % sunny''.
\end{frame}

\begin{frame}[t]
  \frametitle{Is it safe?}

  If we assume
  \begin{center}
    \emph{nothing}
  \end{center}
  then is it safe to conclude:
  \begin{center}
    it is sunny or not sunny
  \end{center}

  \pause
  \bigskip

  Yes!. There are two cases:
  \begin{enumerate}
  \item It is sunny
  \item It is not sunny
  \end{enumerate}
  \pause In either case the conclusion is true: $A \lor B$ requires at
  least one of $A$ or $B$ to be true.
\end{frame}

\begin{frame}[t]
  \frametitle{Is it safe?}

  If we assume
  \begin{center}
    it is sunny \qquad \emph{and} \qquad it is not sunny
  \end{center}
  then is it safe to conclude:
  \begin{center}
    the moon is made of spaghetti
  \end{center}

  \pause
  \bigskip

  Yes! \pause There are four cases:
  \begin{enumerate}
  \item \strikeafter<5>{it is sunny, and the moon is made of spaghetti}
  \item \strikeafter<4>{it is not sunny, and the moon is made of spaghetti}
  \item \strikeafter<5>{it is sunny, and the moon is not made of spaghetti}
  \item \strikeafter<4>{it is not sunny, and the moon is not made of spaghetti}
  \end{enumerate}
  \pause\pause
  % \pause We are assuming ``it is sunny'', so cases 2 and 4 don't
  % matter. \pause We are \emph{also} assuming ``it is not sunny'', so
  % cases 1 and 3 don't matter. \pause There are no remaining cases and
  % ``the moon is made of spaghetti'' is true in all of them.
\end{frame}


  % Idea: step through example by example
  % - example 0 : assume nothing, then it is sunny?
  % - example 1 : assume that it is sunny, then it is sunny?
  % - example 1 : if it is raining, it is not sunny?
  % - example 2 : if it is raining or sunny and it is not sunny, then it is raining?
  % - example 3 : if is is raining and (raining implies not sunny), then it is sunny?
  % - example 4 : if A and (A -> !B), then !B
  % - example 5 : if A and (A -> B), then B
  % - example 6 :

  % common feature of ones that are not valid: there is a counterexample where the assumptions are true, but the conclusion isn't
  % idea: an entailment is valid if

\begin{frame}[t]
  {Entailment}

  In general, we have $n$ assumptions $P_1, \ldots, P_n$ and
  conclusion $Q$.

  \bigskip
  \bigskip

  We are going to say: \qquad $P_1,\ldots,P_n \models Q$

  \hspace{3.5cm} \textcolor{black!60}{Read as $P_1,\ldots,P_n \textit{ entails } Q$}

  \bigskip

  if:\\
  \quad for all ``situations'' \textcolor{black!60}{(i.e., valuations)}\\
  \quad \quad that make \textbf{all} the assumptions $P_i$ true,\\
  \quad \qquad the conclusion $Q$ is true.

\end{frame}

\begin{frame}
  {Entailment}

  \textbf{With more symbols} \\\qquad for all valuations $v$, if, for
  all $i$, $\sem{P_i} v = \true$, then $\sem{Q} v = \true$.

  \bigskip

  \textbf{In terms of Semantics}\\ \qquad every valuation in all $\sem{P_i}$
  is also in $\sem{Q}$ \\ \qquad (in set theory symbols:
  $(\sem{P_1} \cap \dots \cap \sem{P_n}) \subseteq \sem{Q}$).
\end{frame}

% \begin{frame}
%   {Entailment vs Validity}

%   If we have no assumptions, then:
%   \begin{displaymath}
%     \models P
%   \end{displaymath}
%   exactly when
%   \begin{displaymath}
%     \textrm{for all }v.~\sem{P}v = \true
%   \end{displaymath}
%   exactly when
%   \begin{displaymath}
%     P\textrm{ is valid}
%   \end{displaymath}
% \end{frame}

% \begin{frame}
%   {Deduction Theorem}
%   \begin{displaymath}
%     P_1, \ldots, P_n, P \models Q
%     \qquad \textit{exactly when} \qquad
%     P_1, \ldots, P_n \models P \to Q
%   \end{displaymath}

%   \bigskip

%   All these statements are equivalent:
%   \begin{enumerate}
%   \item $P_1, \ldots, P_n, P \models Q$
%   \item for all $v$, if all $\sem{P_i} v = \true$ and $\sem{P}v = \true$, then $\sem{Q}v = \true$
%   \item for all $v$, if all $\sem{P_i} v = \true$, then (if $\sem{P} v = \true$, then $\sem{Q}v = \true$)
%   \item for all $v$, if all $\sem{P_i} v = \true$, then $\sem{P \to Q}v = \true$
%   \item $P_1, \ldots, P_n \models P \to Q$
%   \end{enumerate}
% \end{frame}

% \begin{frame}
%   {Entailment vs satisfiability}

%   So, it is the case that
%   \begin{displaymath}
%     P_1,\dots,P_n \models Q
%   \end{displaymath}
%   exactly when
%   \begin{displaymath}
%     \models P_1 \to \dots \to P_n \to Q
%   \end{displaymath}
%   exactly when
%   \begin{displaymath}
%     P_1 \to \dots \to P_n \to Q \textrm{ is valid}
%   \end{displaymath}
%   exactly when
%   \begin{displaymath}
%     \lnot (P_1 \to \dots \to P_n \to Q) \textrm{ is not satisfiable}
%   \end{displaymath}

%   % FIXME: drive the point home: if we have a satisfiability checker, we have a counterexample finder

% \end{frame}

\weeksection{Deductive Reasoning}

\begin{frame}
  \sechead{Why have logic(s)?}
  One reason is to study ``arguments''.
  \begin{itemize}
  \item To separate valid and invalid reasoning.
  \item If we assume $P_1, P_2, P_3$, then when is it valid to conclude $Q$?
  \end{itemize}

  \pause
  \bigskip

  \sechead{One answer is ``entailment''}
  \begin{itemize}
  \item $P_1, \dots \models Q$ ``is'' valid reasoning from assumptions to a conclusion.
  \end{itemize}
  \emph{Entailment} is defined in terms of the semantics of formulas
  \begin{itemize}
  \item $P_1, ... \models Q$ if \highlight<3->{for all valuations $v$}, $\sem{P}v = \true$ implies $\sem{Q}v \kern-3pt= \true$
  \end{itemize}

  \pause
  \pause
  \bigskip

  \sechead{This doesn't match how we reason normally.}
  If we are trying to convince someone, we don't (usually) say: \\
  \qquad \emph{``let's go through all the combinations of truth values and test each one.''}
\end{frame}

\begin{frame}
  \sechead{Chains of Inference}
  Usually, we might say things like:
  \begin{enumerate}
  \item Let's assume that $A, B, C$ are true.
  \item If we assume $A$ and $B$ imply $D$, then $D$ is true.
  \item If we assume $C$ and $D$ imply $E$, then $E$ is true.
  \item So, we can conclude $E$, under the assumptions.
  \end{enumerate}
  If our reasoning is sound, then we ought to be able to conclude
  \begin{displaymath}
    A, B, C, (A \land B) \to D, (C \land D) \to E \models E
  \end{displaymath}

  \pause
  \sechead{We have a form of \emph{modularity}}
  \begin{itemize}
  \item We don't check the entailment for every possible truth value of $A, B, C, D, E$ \qquad \textcolor{black!60}{($2^5 = 32$ combinations!)}
  \item We apply individual reasoning \emph{steps} and chain them together.
  \end{itemize}
\end{frame}

\begin{frame}
  {Semantic Reasoning doesn't scale}

  In \emph{Propositional Logic}, it is possible (though not always
  feasible) to check all cases.
  \begin{itemize}
  \item If there are $n$ atomic propositions, check $2^n$ combinations.
  \item SAT solvers are good at only checking the ones that matter.
  \item But there are still Hard Problems that take too long.
  \end{itemize}

  \pause
  \bigskip

  Also, later in the course we will study \emph{Predicate Logic}
  \begin{itemize}
  \item Predicate logic allows \emph{universal} statements:
    \begin{displaymath}
      \forall x.\forall y.~x+y = y+x
    \end{displaymath}
    \emph{``For all (numbers) $x$ and $y$, $x+y$ is equal to $y+x$''}
  \item Simply not possible to exhaustively check all numbers.
  \end{itemize}
\end{frame}

\begin{frame}
  {Deductive Systems}
  To overcome these problems, we use \emph{deductive systems}.

  \bigskip

  A \rhighlight{deductive system} is a collection of \emph{rules} for
  deriving conclusions from assumptions.
  \begin{itemize}
  \item Typically, the rules are ``finitely describable'' \\
    \sidenote{roughly: we can implement them on a computer}
  \end{itemize}

  \bigskip

  Typically (but not always), we write
  \begin{displaymath}
    P_1, \cdots, P_n \vdash Q
  \end{displaymath}
  when we can derive conclusion $Q$ from assumptions $P_1, \cdots, P_n$.

  % \bigskip There are many deductive systems. We will look at a variant
  % of Natural Deduction.

\end{frame}

\begin{frame}
  {Soundness and Completeness}

  \textbf{Soundness} : ``Everything that is provable is valid''
  \begin{displaymath}
    P_1, \cdots, P_n \vdash Q \qquad \textit{implies} \qquad P_1, \cdots, P_n \models Q
  \end{displaymath}
  \sidenote{pretty much a requirement to be useful}

  \bigskip

  \textbf{Completeness} : ``Everything that is valid is provable''
  \begin{displaymath}
    P_1, \cdots, P_n \models Q \qquad \textit{implies} \qquad P_1, \cdots, P_n \vdash Q
  \end{displaymath}
  \sidenote{not \emph{essential}, but good to have}

\end{frame}

\begin{frame}
  {Advantages of Deductive Systems}

  \bigskip \emph{1. } We can write computer programs to check our proofs, even
  when talking about infinitely many things.

  \pause
  \bigskip
  \emph{2. }If we remove or alter rules do we get an interesting new logic?

  \pause
  \bigskip
  \emph{3. }We can start to ask questions about the proofs:
  \begin{itemize}
  \item An entailment $P_1, \cdots, P_n \models Q$ is either valid or invalid. Meh.
  \item but there may be many proofs (ways of applying the rules).
  \item Questions:
    \begin{itemize}
    \item Do different proofs \emph{mean} different things?
    \item Is one proof a simplification of another?
    \item Is there information hidden in proofs that we can extract?
    \end{itemize}
  \end{itemize}
\end{frame}

\begin{frame}[t]
  {Inference Rules}

  \begin{displaymath}
    \inferrule*
    {\mathit{premise}_1 \\ \cdots \\ \mathit{premise}_n}
    {\mathit{conclusion}}
  \end{displaymath}

  The idea:
  \begin{itemize}
  \item If we can prove all of $\mathit{premise}_1$ and ... and $\mathit{premise}_n$; then
  \item we have a proof of $\mathit{conclusion}$.
  \end{itemize}

  \pause
  \bigskip

  We might have zero premises, \\
  \quad in which case the $\mathit{conclusion}$
  requires no proof (``is an axiom'').

  \pause
  \bigskip

  Rules are organised into \emph{trees} to make \emph{deductions}.
\end{frame}

\begin{frame}[t]
  {Example}
  \begin{mathpar}
    \inferrule* [right=Rule1] { } {\textrm{bears}~\textit{are furry}}

    \inferrule* [right=Rule2] { } {\textrm{bears}~\textit{make milk}}

    \inferrule* [right=Rule3]
    {X~\textit{are furry} \\ X~\textit{make milk}}
    {X~\textit{are mammals}}
  \end{mathpar}

  \pause
  \bigskip

  A deduction:
  \begin{displaymath}
    \inferrule* [right=Rule3]
    {\inferrule* [right=Rule1] { } {\textrm{bears}~\textit{are furry}} \\
      \inferrule* [Right=Rule2] { } {\textrm{bears}~\textit{make milk}}}
    {\textrm{bears}~\textit{are mammals}}
  \end{displaymath}
\end{frame}

\begin{frame}
  {Example (cont.)}

  \begin{mathpar}
    \inferrule* [right=Rule4]
    {X~\textit{are covered in fibres}}
    {X~\textit{are furry}}

    \inferrule* [right=Rule5]
    { }
    {\textrm{coconuts}~\textit{are covered in fibres}}

    \inferrule* [right=Rule6]
    { }
    {\textrm{coconuts}~\textit{make milk}}
  \end{mathpar}

\end{frame}

\begin{frame}
  {Example (cont.)}

  Another deduction:
  \begin{displaymath}
    \inferrule* [right=R3]
    {\inferrule* [right=R4]
      {\inferrule* [Right=R5] { } {\textrm{coconuts}~\textit{are covered in fibres}}}
      {\textrm{coconuts}~\textit{are furry}}
      \\
      \inferrule* [Right=R6]
      { } {\textrm{coconuts}~\textit{make milk}}}
    {\textrm{coconuts}~\textit{are mammals}}
  \end{displaymath}
\end{frame}

\begin{frame}[t]
  {Example (cont.)}
  When \emph{building} deductions, we work bottom up:

  \begin{minipage}[t][2cm][b]{\textwidth}
  \begin{displaymath}
    \only<2>{\textrm{coconuts}~\textit{are mammals}}%
    \only<3>{\inferrule* [right=R3]
      {\textrm{coconuts}~\textit{are furry} \\ \textrm{coconuts}~\textit{make milk}}
      {\textrm{coconuts}~\textit{are mammals}}}%
    \only<4>{\inferrule* [right=R3]
      {\inferrule* [right=R4] {\textrm{coconuts}~\textit{are covered in fibres}} {\textrm{coconuts}~\textit{are furry}}
        \\
        \textrm{coconuts}~\textit{make milk}}
      {\textrm{coconuts}~\textit{are mammals}}}%
    \only<5>{\inferrule* [right=R3]
      {\inferrule* [right=R4]
        {\inferrule* [Right=R5] { } {\textrm{coconuts}~\textit{are covered in fibres}}}
        {\textrm{coconuts}~\textit{are furry}}
        \\
        \textrm{coconuts}~\textit{make milk}}
      {\textrm{coconuts}~\textit{are mammals}}}%
    \only<6-7>{\inferrule* [right=R3]
    {\inferrule* [right=R4]
      {\inferrule* [Right=R5] { } {\textrm{coconuts}~\textit{are covered in fibres}}}
      {\textrm{coconuts}~\textit{are furry}}
      \\
      \inferrule* [Right=R6]
      { } {\textrm{coconuts}~\textit{make milk}}}
      {\textrm{coconuts}~\textit{are mammals}}}
  \end{displaymath}
\end{minipage}

\bigskip

  \begin{enumerate}
  \item<2-> Write down the conclusion
  \item<3-> Apply rule \TirName{Rule3} ($X$ \emph{are mammals} if $X$ \emph{are furry and make milk})
  \item<4-> Apply rule \TirName{Rule4} ($X$ \emph{are furry} if they \emph{are covered in fibres})
  \item<5-> Apply rule \TirName{Rule5} (an axiom)
  \item<6-> Apply rule \TirName{Rule6} (an axiom)
  \end{enumerate}

  % \bigskip
  % \pause\pause\pause\pause\pause\pause
  % \begin{itemize}
  % \item A statement with no line above it is an \emph{open goal}
  % \item A branch of the tree that ends with an axiom is called \emph{closed}
  % \end{itemize}
\end{frame}

\begin{frame}
  {Summary}

  \begin{itemize}
  \item The \emph{why?} of deductive systems.
  \item Inference rules.
  \item How to make chains of inference.
  \end{itemize}
\end{frame}



\end{document}
